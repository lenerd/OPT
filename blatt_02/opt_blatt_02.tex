\documentclass[a4paper]{scrartcl}

% font/encoding packages
\usepackage[utf8]{inputenc}
\usepackage[T1]{fontenc}
\usepackage{lmodern}
\usepackage[ngerman]{babel}

\usepackage{amsmath, amssymb, amsfonts, amsthm, mathtools}
\setcounter{MaxMatrixCols}{13}
\allowdisplaybreaks
\usepackage[output-decimal-marker={,}]{siunitx}
\usepackage{graphicx}
\usepackage{subcaption}
\usepackage{caption}
\usepackage{venndiagram}
\usepackage{enumerate}
\usepackage{polynom}
\polyset{style=C, div=:,vars=x}
\usepackage[section]{placeins}
\usepackage{tikz}
\usetikzlibrary{tikzmark}
\usetikzlibrary{calc}
\usetikzlibrary{arrows}
\usetikzlibrary{arrows.meta}
\usepackage{pgfplots}
\pgfplotsset{compat=1.5.1}
\usepgfplotslibrary{fillbetween}
\usepackage{gauss}
\usepackage{systeme}
\usepackage{cancel}

\tikzset{
    declare function={Floor(\x)=round(\x-0.5);}
}

% Matrizen
\newdimen\asep 
\asep=0.25\baselineskip 

\providecommand{\vo}{ 
    \text{\raisebox{-\asep}[0pt][0pt]{\rule[0pt]{0.8pt}{3.5\asep}}}} 
\providecommand{\vu}{ 
    \text{\raisebox{-0.5\asep}[0pt][0pt]{\rule[0pt]{0.8pt}{\baselineskip}}}} 
\providecommand{\vm}{ 
    \text{\raisebox{-\asep}[0pt][0pt]{\rule[0pt]{0.8pt}{\baselineskip}}}}

\newtheorem*{behaupt}{Behauptung}

\newcommand{\fallfac}[2]{{#1}^{\underline{#2}}}

\newcommand{\risefac}[2]{{#1}^{\overline{#2}}}

\newcommand{\dif}{\mathop{}\!\mathrm{d}}

\newcommand{\linf}[1]{\lim_{#1 \to \infty}}
\newcommand{\gdw}{\Leftrightarrow}

\renewcommand{\arraystretch}{1.3}

\title{Optimierung}
\subtitle{Blatt 2 Hausaufgaben}
\author{
	Lennart Braun (6523742, Gruppe 6) \\
    Alexander Timmermann (6524072, Gruppe 5)
}
\date{zum 27. Oktober 2014}

\begin{document}
\maketitle

\begin{enumerate}
    \item % 1.
        \begin{enumerate}
            \item
                Starttableau
                \begin{equation}
                    \begin{array}{rcrrrr}
                        x_4 & = & 7 & -x_1 & -3x_2 & -2x_3 \\
                        x_5 & = & 4 & -x_1 & -2x_2 &  -x_3 \\
                        x_6 & = & 5 &      & -3x_2 & -2x_3 \\
                        \hline
                        z   & = &   & 2x_1 & +4x_2 & +3x_3 \\
                    \end{array}
                \end{equation}
                Eingangsvariable: $x_2$ \\
                Ausgangsvariable: $x_6$ \\

                1. Iteration
                \begin{equation}
                    \begin{array}{rcrrrr}
                        x_2 & = & \frac{5}{3} & & -\frac{2}{3}x_3 & -\frac{1}{3}x_6 \\
                        x_4 & = & 2 & -x_1 & & x_6 \\
                        x_5 & = & \frac{2}{3} & -x_1 & +\frac{1}{3}x_3 & +\frac{2}{3}x_6 \\
                        \hline
                        z   & = & \frac{20}{3} & +2x_1 & +\frac{1}{3}x_3 & -\frac{4}{3}x_6 \\
                    \end{array}
                \end{equation}
                Eingangsvariable: $x_1$ \\
                Ausgangsvariable: $x_5$

                2. Iteration
                \begin{equation}
                    \begin{array}{rcrrrr}
                        x_1 & = & \frac{2}{3} & +\frac{1}{3}x_3 & +\frac{2}{3}x_6 & -x_5 \\
                        x_2 & = & \frac{5}{3} & -\frac{2}{3}x_3 & -\frac{1}{3}x_6 & \\
                        x_4 & = & \frac{4}{3} & -\frac{1}{3}x_3 & +\frac{1}{3}x_6 & +x_5 \\
                        \hline
                        z   & = & 8 & +x_3 & & -2x_5 \\
                    \end{array}
                \end{equation}
                Eingangsvariable: $x_3$ \\
                Ausgangsvariable: $x_2$

                3. Iteration
                \begin{equation}
                    \begin{array}{rcrrrr}
                        x_3 & = & \frac{5}{2} & -\frac{1}{2}x_6 & & -\frac{3}{2}x_2 \\
                        x_1 & = & \frac{3}{2} & +\frac{1}{2}x_6 & -x_5 & -\frac{1}{2}x_2 \\
                        x_4 & = & \frac{1}{2} & +\frac{1}{2}x_6 & +x_5 & +\frac{1}{2}x_2 \\
                        \hline
                        z   & = & \frac{21}{2} & -\frac{1}{2}x_6 & -2x_5 & -\frac{3}{2}x_2 \\
                    \end{array}
                \end{equation}
                Optimale Lösung:
                \begin{equation}
                    \begin{split}
                        x_1 &= \frac{3}{2} \\
                        x_2 &= 0 \\
                        x_3 &= \frac{5}{2} \\
                        x_4 &= \frac{1}{2} \\
                        x_5 &= 0 \\
                        x_6 &= 0 \\
                        z   &= \num{10,5}
                    \end{split}
                \end{equation}
                
            \item
                Starttableau
                \begin{equation}
                    \begin{array}{rcrrrr}
                        x_4 & = & 4 & -3x_1 & -3x_2 & +x_3 \\
                        x_5 & = & 6 & -5x_1 & -3x_2 & -x_3 \\
                        x_6 & = & 2 &  +x_1 & -3x_2 & -x_3 \\
                        x_7 & = & 2 & -3x_1 & +4x_2 & +x_3 \\
                        \hline
                        z   & = &   & 9x_1 & -5x_2 & -4x_3 \\
                    \end{array}
                \end{equation}
                Eingangsvariable: $x_1$ \\
                Ausgangsvariable: $x_7$ \\

                1. Iteration
                \begin{equation}
                    \begin{array}{rcrrrr}
                        x_1 & = & \frac{2}{3} & +\frac{4}{3}x_2 & +\frac{1}{3}x_3 & -\frac{1}{3}x_7 \\
                        x_4 & = & 2 & -7x_2 &  & +x_7 \\
                        x_5 & = & \frac{8}{3} & -\frac{29}{3}x_2 & -\frac{8}{3}x_3 & +\frac{5}{3}x_7 \\
                        x_6 & = & \frac{8}{3} & -\frac{5}{3}x_2 & -\frac{2}{3}x_3 & -\frac{1}{3}x_7 \\
                        \hline
                        z   & = & 6 & +7x_2 & -x_3 & -3x_7 \\
                    \end{array}
                \end{equation}
                Eingangsvariable: $x_2$ \\
                Ausgangsvariable: $x_5$ \\

                2. Iteration
                \begin{equation}
                    \begin{array}{rcrrrr}
                        x_2 & = & \frac{8}{29} & -\frac{8}{29}x_3 & +\frac{5}{29}x_7 & -\frac{3}{29}x_5 \\
                        x_1 & = & \frac{30}{29} & -\frac{1}{29}x_3 & -\frac{3}{29}x_7 & -\frac{4}{29}x_5 \\
                        x_4 & = & \frac{2}{29} & +\frac{56}{29}x_3 & -\frac{6}{29}x_7 & +\frac{21}{29}x_5 \\
                        x_6 & = & \frac{64}{29} & -\frac{6}{29}x_3 & -\frac{18}{29}x_7 & +\frac{5}{29}x_5 \\
                        \hline
                        z   & = & \frac{230}{29} & -\frac{85}{29}x_3 & -\frac{52}{29}x_7 & -\frac{21}{29}x_5 \\
                    \end{array}
                \end{equation}
                Optimale Lösung:
                \begin{equation}
                    \begin{split}
                        x_1 &= \frac{30}{29} \\
                        x_2 &= \frac{8}{29} \\
                        x_3 &= 0 \\
                        x_4 &= \frac{2}{29} \\
                        x_5 &= 0 \\
                        x_6 &= \frac{64}{29} \\
                        x_7 &= 0 \\
                        z   &= \frac{230}{29} \approx \num{7,93}
                    \end{split}
                \end{equation}
                
        \end{enumerate}

    \item % 2.
        Starttableau
        \begin{equation}
            \begin{array}{rcrrrrr}
                x_5 & = & 4 & -x_1 & -3x_2 & -x_3 & -x_4 \\
                x_6 & = & 1 & -x_1 & +7x_2 & +3x_3 & +x_4 \\
                \hline
                z   & = &   & 4x_1 & -13x_2 & -9x_3 & +x_4 \\
            \end{array}
        \end{equation}
        Eingangsvariable: $x_1$ \\
        Ausgangsvariable: $x_6$ \\

        1. Iteration
        \begin{equation}
            \begin{array}{rcrrrrr}
                x_1 & = & 1 & +7x_2 & +3x_3 & +x_4 & -x_6 \\
                x_5 & = & 3 & -10x_2 & -4x_3 & -2x_4 & +x_6 \\
                \hline
                z   & = & 4 & +15x_2 & +3x_3 & +5x_4 & -4x_6 \\
            \end{array}
        \end{equation}
        Eingangsvariable: $x_2$ \\
        Ausgangsvariable: $x_5$ \\

        2. Iteration
        \begin{equation}
            \begin{array}{rcrrrrr}
                x_2 & = & \frac{3}{10} & -\frac{2}{5}x_3 & -\frac{1}{5}x_4 & +\frac{1}{10}x_6 & -\frac{1}{10}x_5 \\
                x_1 & = & \frac{31}{10} & +\frac{1}{5}x_3 & -\frac{2}{5}x_4 & -\frac{3}{10}x_6 & -\frac{7}{10}x_5 \\
                \hline
                z   & = & \frac{17}{2} & -3x_3 & +2x_4 & -\frac{5}{2}x_6 & -\frac{3}{2}x_5 \\
            \end{array}
        \end{equation}
        Eingangsvariable: $x_4$ \\
        Ausgangsvariable: $x_2$ \\

        3. Iteration
        \begin{equation}
            \begin{array}{rcrrrrr}
                x_4 & = & \frac{3}{2} & -2x_3 & +\frac{1}{2}x_6 & -\frac{1}{2}x_5 & -5x_2 \\
                x_1 & = & \frac{5}{2} & +x_3 & -\frac{1}{2}x_6 & -\frac{1}{2}x_5 & +2x_2 \\
                \hline
                z   & = & \frac{23}{2} & -7x_3 & -\frac{3}{2}x_6 & -\frac{5}{2}x_5 & -10x_2 \\
            \end{array}
        \end{equation}
        Optimale Lösung:
        \begin{equation}
            \begin{split}
                x_1 &= \frac{5}{2} \\
                x_2 &= 0 \\
                x_3 &= 0 \\
                x_4 &= \frac{3}{2} \\
                x_5 &= 0 \\
                x_6 &= 0 \\
                z   &= \frac{23}{2} = \num{11,5}
            \end{split}
        \end{equation}


\end{enumerate}


\end{document}
