\documentclass[a4paper]{scrartcl}

% font/encoding packages
\usepackage[utf8]{inputenc}
\usepackage[T1]{fontenc}
\usepackage{lmodern}
\usepackage[ngerman]{babel}

% \usepackage[top=2.5cm,bottom=3cm,left=1.5cm,right=1.5cm]{geometry}

\usepackage{amsmath, amssymb, amsfonts, amsthm, mathtools}
\usepackage{marvosym, stmaryrd}
\setcounter{MaxMatrixCols}{13}
\allowdisplaybreaks
\usepackage[output-decimal-marker={,}]{siunitx}
\usepackage{graphicx}
\usepackage{subcaption}
\usepackage{caption}
\usepackage{venndiagram}
\usepackage[shortlabels]{enumitem}
\usepackage{polynom}
\polyset{style=C, div=:,vars=x}
\usepackage[section]{placeins}
\usepackage{tikz}
\usetikzlibrary{tikzmark}
\usetikzlibrary{calc}
\usetikzlibrary{arrows}
\usetikzlibrary{arrows.meta}
\usepackage{pgfplots}
\pgfplotsset{compat=1.5.1}
\usepgfplotslibrary{fillbetween}
\usepackage{gauss}
\usepackage{systeme}
\usepackage{cancel}

\tikzset{
    declare function={Floor(\x)=round(\x-0.5);}
}

% Matrizen
\newdimen\asep 
\asep=0.25\baselineskip 

\providecommand{\vo}{ 
    \text{\raisebox{-\asep}[0pt][0pt]{\rule[0pt]{0.8pt}{3.5\asep}}}} 
\providecommand{\vu}{ 
    \text{\raisebox{-0.5\asep}[0pt][0pt]{\rule[0pt]{0.8pt}{\baselineskip}}}} 
\providecommand{\vm}{ 
    \text{\raisebox{-\asep}[0pt][0pt]{\rule[0pt]{0.8pt}{\baselineskip}}}}

\newtheorem*{behaupt}{Behauptung}

\newcommand{\fallfac}[2]{{#1}^{\underline{#2}}}

\newcommand{\risefac}[2]{{#1}^{\overline{#2}}}

\newcommand{\dif}{\mathop{}\!\mathrm{d}}

\newcommand{\linf}[1]{\lim_{#1 \to \infty}}
\newcommand{\gdw}{\Leftrightarrow}

\renewcommand{\arraystretch}{1.3}

\title{Optimierung}
\subtitle{Blatt 7 Hausaufgaben}
\author{
	Lennart Braun (6523742, Gruppe 6) \\
    Alexander Timmermann (6524072, Gruppe 5)
}
\date{zum 1. Dezember 2014}

\begin{document}
\maketitle

\begin{enumerate}[label=\bfseries\arabic*.]
    \item % 1.
        \begin{enumerate}
            \item $(D)$
                \begin{equation}
                    \begin{gathered}
                        \text{minimiere }
                        \frac{22}{3}y_1 + \frac{4}{3}y_2 + \frac{13}{3}y_3 \\
                        \text{unter den Nebenbedingungen} \\
                        \begin{array}{rrrcr}
                            y_1 & + 4y_2 & +2y_3 & \geq & 12 \\
                            3y_1 & +2y_2 & +y_3 & \geq & 11 \\
                            2y_1 & +y_2 & +5y_3 & \geq & 7 \\
                            5y_1 & -2y_2 & +4y_3 & \geq & 5
                        \end{array} \\
                        y_1, y_2, y_3 \geq 0
                    \end{gathered}
                \end{equation}

                Da $x_1^* = x_3^* = 0$ muss Gleichheit nur für die zweite und
                die vierte Zeile von $(D)$ gelten.

                $x^*$ in $(P)$
                \begin{equation}
                    \begin{split}
                        \frac{12}{3} + \frac{10}{3} &= \frac{22}{3} \\
                        \frac{8}{3} - \frac{4}{3} &= \frac{4}{3} \\
                        \frac{4}{3} + \frac{8}{3} = \frac{12}{3} &< \frac{13}{3}
                    \end{split}
                \end{equation}
                Aus der dritten Zeile folgt, dass $y_3^* = 0$ sein muss.

                Wir erhalten folgendes Gleichungssystem.
                \begin{equation}
                    \begin{array}{rrrcr}
                        3y_1 & +2y_2 & = & 11 \\
                        5y_1 & -2y_2 & = & 5
                    \end{array}
                \end{equation}

                Daraus folgt.
                \begin{equation}
                    y^* = \left( 2, 2.5, 0 \right)
                \end{equation}
                
                $y^*$ in $(D)$
                \begin{equation}
                    \begin{split}
                        2 + 10 &\geq 12 \\
                        6 + 5 &\geq 11 \\
                        4 + 2.5 = 6.5 &< 7 \lightning \\
                        10 - 5 &\geq 5
                    \end{split}
                \end{equation}
                
                Offensichtlich erfüllt $y^*$ das Gleichungssystem nicht.
                Daher ist $x^*$ keine optimale Lösung.

            \item $(D)$
                \begin{equation}
                    \begin{gathered}
                        \text{minimiere }
                        5y_1 +2y_2 +2y_3 \\
                        \text{unter den Nebenbedingungen} \\
                        \begin{array}{rrrcr}
                            2y_1 & -y_2 &  & \geq & 1 \\
                             & 3y_2 & +y_3 & \geq & 6 \\
                            y_1 & -2y_2 & -y_3 & \geq & -4
                        \end{array} \\
                        y_1, y_2, y_3 \geq 0
                    \end{gathered}
                \end{equation}

                Da $x_3^* = 0$ muss Gleichheit nur für die erste und zweite
                Zeile von $(D)$ gelten.

                $x^*$ in $(P)$
                \begin{equation}
                    \begin{split}
                        \frac{10}{2} &= 5 \\
                        -\frac{5}{2} +\frac{9}{2} &= 2 \\
                        \frac{3}{2} &\neq 2
                    \end{split}
                \end{equation}
                Aus der dritten Zeile folgt, dass $y_3^* = 0$ sein muss.
                
                Wir erhalten folgendes Gleichungssystem.
                \begin{equation}
                    \begin{array}{rrrcr}
                        2y_1 & -y_2 & = & 1 \\
                        & 3y_2 & = & 6
                    \end{array}
                \end{equation}

                Daraus folgt.
                \begin{equation}
                    y^* = \left( 1.5, 2, 0 \right)
                \end{equation}

                $y^*$ in $(D)$
                \begin{equation}
                    \begin{split}
                        3 - 2 &\geq 1 \\
                        6 &\geq 6 \\
                        1.5 -4 &\geq -4
                    \end{split}
                \end{equation}
                
                $x^*$ ist eine optimale Lösung für $(P)$.

        \end{enumerate}

    \item % 2.
        \begin{enumerate}
            \item

            \item

        \end{enumerate}

\end{enumerate}


\end{document}
