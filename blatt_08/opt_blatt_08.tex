\documentclass[a4paper]{scrartcl}

% font/encoding packages
\usepackage[utf8]{inputenc}
\usepackage[T1]{fontenc}
\usepackage{lmodern}
\usepackage[ngerman]{babel}

% \usepackage[top=2.5cm,bottom=3cm,left=1.5cm,right=1.5cm]{geometry}

\usepackage{amsmath, amssymb, amsfonts, amsthm, mathtools}
\usepackage{stmaryrd}
\usepackage{marvosym}
\setcounter{MaxMatrixCols}{13}
\allowdisplaybreaks
\usepackage[output-decimal-marker={,}]{siunitx}
\usepackage{graphicx}
\usepackage{subcaption}
\usepackage{caption}
\usepackage{venndiagram}
\usepackage[shortlabels]{enumitem}
\usepackage{polynom}
\polyset{style=C, div=:,vars=x}
\usepackage[section]{placeins}
\usepackage{tikz}
\usetikzlibrary{tikzmark}
\usetikzlibrary{calc}
\usetikzlibrary{arrows}
\usetikzlibrary{arrows.meta}
\usepackage{pgfplots}
\pgfplotsset{compat=1.5.1}
\usepgfplotslibrary{fillbetween}
\usepackage{gauss}
\usepackage{systeme}
\usepackage{cancel}

\tikzset{
    declare function={Floor(\x)=round(\x-0.5);}
}

% Matrizen
\newdimen\asep 
\asep=0.25\baselineskip 

\providecommand{\vo}{ 
    \text{\raisebox{-\asep}[0pt][0pt]{\rule[0pt]{0.8pt}{3.5\asep}}}} 
\providecommand{\vu}{ 
    \text{\raisebox{-0.5\asep}[0pt][0pt]{\rule[0pt]{0.8pt}{\baselineskip}}}} 
\providecommand{\vm}{ 
    \text{\raisebox{-\asep}[0pt][0pt]{\rule[0pt]{0.8pt}{\baselineskip}}}}

\newtheorem*{behaupt}{Behauptung}

\newcommand{\fallfac}[2]{{#1}^{\underline{#2}}}

\newcommand{\risefac}[2]{{#1}^{\overline{#2}}}

\newcommand{\dif}{\mathop{}\!\mathrm{d}}

\newcommand{\linf}[1]{\lim_{#1 \to \infty}}
\newcommand{\gdw}{\Leftrightarrow}

\renewcommand{\arraystretch}{1.3}

\title{Optimierung}
\subtitle{Blatt 7 Hausaufgaben}
\author{
	Lennart Braun (6523742, Gruppe 6) \\
    Alexander Timmermann (6524072, Gruppe 5)
}
\date{zum 8. Dezember 2014}

\begin{document}
\maketitle

\begin{enumerate}[label=\bfseries\arabic*.]
    \item % 1.
        \begin{enumerate}
            \item $(D)$
                \begin{equation}
                    \begin{gathered}
                        \text{minimiere }
                        -y_1 +2y_2 -3y_3 \\
                        \text{unter den Nebenbedingungen} \\
                        \begin{array}{rrrcr}
                            -y_1 & +6y_2 & -y_3 & = & 1 \\
                            3y_1 & +y_2 & -y_3 & = & 3 \\
                            y_1 & +4y_2 & +2y_3 & \geq & -3 \\
                            -y_1 & -2y_2 & +y_3 & \geq & 2 \\
                            7y_1 & +y_2 & -2y_3 & \geq & 1 \\
                        \end{array} \\
                        y_1, y_3 \geq 0
                    \end{gathered}
                \end{equation}

            \item $(D)$
                \begin{equation}
                    \begin{gathered}
                        \text{maximiere} \\
                        -3y_1 +9y_2 +5y_3 +8y_4 +4y_5 -y_6 -10y_7 +9y_8 \\
                        \text{unter den Nebenbedingungen} \\
                        \begin{array}{rrrrrrrrcr}
                            -y_1 & +y_2 & +y_3 & +2y_4 & +y_5 & +4y_6 & -4y_7 & +y_8 & \leq & 5 \\
                            5y_1 & +4y_2 & & +4y_4 & -3y_5 & -3y_6 & +3y_7 & +2y_8 & \leq & -1 \\
                            -y_1 & +y_2 & +y_3 & -y_4 & +y_5 & & -5y_7 & +y_8 & \leq & 1 \\
                            -2y_1 & & & +y_4 & & & +y_7 & +7y_8 & \leq & 2 \\
                        \end{array} \\
                        y_1, y_3, y_6, y_7, y_8 \geq 0
                    \end{gathered}
                \end{equation}

        \end{enumerate}

    \item % 2.
        \begin{enumerate}
            \item
                \begin{enumerate}[(i)]
                    \item Das LP-Problem $(P)$
                        \begin{equation}
                            \begin{gathered}
                                \text{maximiere }
                                40x_1 +70x_2 \\
                                \text{unter den Nebenbedingungen} \\
                                \begin{array}{rrcr}
                                    x_1 & +x_2 & \leq & 100 \\
                                    10x_1 & +50x_2 & \leq & 4000 \\
                                \end{array} \\
                                x_1, x_2 \geq 0
                            \end{gathered}
                        \end{equation}

                        Starttableau
                        \begin{equation}
                            \begin{array}{rcrrr}
                                x_3 & = & 100 & -x_1 & -x_2 \\
                                x_4 & = & 4000 & -10x_1 & -50x_2 \\
                                \hline
                                z & = & & 40x_1 & +70x_2 \\
                            \end{array}
                        \end{equation}
                        Eingangsvariable: $x_2$ \\
                        Ausgangsvariable: $x_4$

                        1. Iteration
                        \begin{equation}
                            \begin{array}{rcrrr}
                                x_2 & = & 80 & -\frac{1}{5}x_1 & -\frac{1}{50}x_4 \\
                                x_3 & = & 20 & -\frac{4}{5}x_1 & +\frac{1}{50}x_4 \\
                                \hline
                                z & = & 5600 & +26x_1 & -\frac{7}{5}x_4 \\
                            \end{array}
                        \end{equation}
                        Eingangsvariable: $x_1$ \\
                        Ausgangsvariable: $x_3$

                        2. Iteration
                        \begin{equation}
                            \begin{array}{rcrrr}
                                x_1 & = & 25 & +\frac{1}{40}x_4 & -\frac{5}{4}x_3 \\
                                x_2 & = & 75 & -\frac{1}{40}x_4 & +\frac{1}{4}x_3 \\
                                \hline
                                z & = & 6250 & -0.75x_4 & -32.5x_3 \\
                            \end{array}
                        \end{equation}
                        $x* = \left( 25, 75 \right)$ und $y* = \left( 0.75, 32.5 \right)$
                        sind optimale Lösungen für $(P)$ bzw. das zu $(P)$ duale
                        Problem $(D)$.
                        

                    \item $(D)$
                        \begin{equation}
                            \begin{gathered}
                                \text{minimiere }
                                100y_1 +4000y_2 \\
                                \text{unter den Nebenbedingungen} \\
                                \begin{array}{rrcr}
                                    y_1 & +10y_2 & \geq & 40 \\
                                    y_1 & +50y_2 & \geq & 70 \\
                                \end{array} \\
                                y_1, y_2 \geq 0
                            \end{gathered}
                        \end{equation}

                        Weder ein $x_i^*$ noch ein $y_i^*$ ist 0, daher müssen
                        für alle Ungleichungen in $(P)$ bzw. $(D)$ Gleichheit
                        gelten, wenn $x^*$ bzw. $y^*$ eingesetzt wird.

                        $x^*$ in $(P)$
                        \begin{equation}
                            \begin{split}
                                25 +75 &= 100 \\
                                250 +3750 &= 4000 \\
                            \end{split}
                        \end{equation}

                        $y^*$ in $(D)$
                        \begin{equation}
                            \begin{split}
                                32.5 +7.5 &= 40 \\
                                32.5 +37.5 &= 70 \\
                            \end{split}
                        \end{equation}
                        
                        Die komplementären Schlupfbedingungen sind für $x^*$
                        und $y^*$ erfüllt.
                        Es handelt sich daher um optimale Lösungen für $(P)$
                        bzw. $(D)$.

                \end{enumerate}

            \item
                \begin{equation}
                    \begin{gathered}
                        \text{maximiere }
                        40x_1 +70x_2 \\
                        \text{unter den Nebenbedingungen} \\
                        \begin{array}{rrcr}
                            x_1 & +x_2 & \leq & 100 \\
                            10x_1 & +50x_2 & \leq & 4000 + t \\
                        \end{array} \\
                        x_1, x_2 \geq 0
                    \end{gathered}
                \end{equation}

                Starttableau
                \begin{equation}
                    \begin{array}{rcrrrrc}
                        x_3 & = & 100 & & -x_1 & -x_2 \\
                        x_4 & = & 4000 & +t & -10x_1 & -50x_2 & (\star\star) \\
                        \hline
                        z & = & & & 40x_1 & +70x_2 \\
                    \end{array}
                \end{equation}
                Eingangsvariable: $x_2$ \\
                Ausgangsvariable: $x_4$

                1. Iteration
                \begin{equation}
                    \begin{array}{rcrrrrc}
                        x_2 & = & 80 & +\frac{1}{50}t & -\frac{1}{5}x_1 & -\frac{1}{50}x_4 & (\star\star) \\
                        x_3 & = & 20 & -\frac{1}{50}t & -\frac{4}{5}x_1 & +\frac{1}{50}x_4 & (\star) \\
                        \hline
                        z & = & 5600 & +\frac{7}{5}t & +26x_1 & -\frac{7}{5}x_4 \\
                    \end{array}
                \end{equation}

                Eingangsvariable: $x_1$ \\
                Ausgangsvariable: $x_3$

                2. Iteration
                \begin{equation}
                    \begin{array}{rcrrrrc}
                        x_1 & = & 25 & -\frac{1}{40}t & +\frac{1}{40}x_4 & -\frac{5}{4}x_3 & (\star) \\
                        x_2 & = & 75 & +\frac{1}{40}t & -\frac{1}{40}x_4 & +\frac{1}{4}x_3 & (\star\star) \\
                        \hline
                        z & = & 6250 & +0.75t & -0.75x_4 & -32.5x_3 \\
                    \end{array}
                \end{equation}

                \begin{itemize}
                    \item
                        In den mit $(\star)$ markierten Zeilen wird von der
                        Voraussetzung $t \leq 1000$ Gebrauch gemacht, da
                        $x_1, x_3 \geq 0$ gelten muss.
                    \item
                        In den mit $(\star\star)$ markierten Zeilen wird
                        Gebrauch von der Voraussetzung $t \geq 0$ gemacht.
                        Für große Beträge von $x$, könnte $x_2 \geq 0$ verletzt
                        werden.
                \end{itemize}

                Für die optimale Lösung $x^* = \left( 25 -0.025t, 75+0.025t \right)$
                wird ein Gewinn von $z = 6250 + 0.75t$ erzielt.

        \end{enumerate}

\end{enumerate}


\end{document}
