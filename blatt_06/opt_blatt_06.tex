\documentclass[a4paper]{scrartcl}

% font/encoding packages
\usepackage[utf8]{inputenc}
\usepackage[T1]{fontenc}
\usepackage{lmodern}
\usepackage[ngerman]{babel}

% \usepackage[top=2.5cm,bottom=3cm,left=1.5cm,right=1.5cm]{geometry}

\usepackage{amsmath, amssymb, amsfonts, amsthm, mathtools}
\usepackage{marvosym}
\setcounter{MaxMatrixCols}{13}
\allowdisplaybreaks
\usepackage[output-decimal-marker={,}]{siunitx}
\usepackage{graphicx}
\usepackage{subcaption}
\usepackage{caption}
\usepackage{venndiagram}
\usepackage[shortlabels]{enumitem}
\usepackage{polynom}
\polyset{style=C, div=:,vars=x}
\usepackage[section]{placeins}
\usepackage{tikz}
\usetikzlibrary{tikzmark}
\usetikzlibrary{calc}
\usetikzlibrary{arrows}
\usetikzlibrary{arrows.meta}
\usepackage{pgfplots}
\pgfplotsset{compat=1.5.1}
\usepgfplotslibrary{fillbetween}
\usepackage{gauss}
\usepackage{systeme}
\usepackage{cancel}

\tikzset{
    declare function={Floor(\x)=round(\x-0.5);}
}

% Matrizen
\newdimen\asep 
\asep=0.25\baselineskip 

\providecommand{\vo}{ 
    \text{\raisebox{-\asep}[0pt][0pt]{\rule[0pt]{0.8pt}{3.5\asep}}}} 
\providecommand{\vu}{ 
    \text{\raisebox{-0.5\asep}[0pt][0pt]{\rule[0pt]{0.8pt}{\baselineskip}}}} 
\providecommand{\vm}{ 
    \text{\raisebox{-\asep}[0pt][0pt]{\rule[0pt]{0.8pt}{\baselineskip}}}}

\newtheorem*{behaupt}{Behauptung}

\newcommand{\fallfac}[2]{{#1}^{\underline{#2}}}

\newcommand{\risefac}[2]{{#1}^{\overline{#2}}}

\newcommand{\dif}{\mathop{}\!\mathrm{d}}

\newcommand{\linf}[1]{\lim_{#1 \to \infty}}
\newcommand{\gdw}{\Leftrightarrow}

\renewcommand{\arraystretch}{1.3}

\title{Optimierung}
\subtitle{Blatt 6 Hausaufgaben}
\author{
	Lennart Braun (6523742, Gruppe 6) \\
    Alexander Timmermann (6524072, Gruppe 5)
}
\date{zum 24. November 2014}

\begin{document}
\maketitle

\begin{enumerate}[label=\bfseries\arabic*.]
    \item % 1.
        \begin{enumerate}
            \item
                \begin{enumerate}[(i)]
                    \item $(D)$
                        \begin{equation}
                            \begin{gathered}
                                \text{minimiere } 4y_1 +6y_2 +2y_3 +2y_4 \\
                                \text{unter den Nebenbedingungen} \\
                                \begin{array}{rrrrcr}
                                    3y_1 & +5y_2 & -y_3 & +3y_4 & \geq & 9 \\
                                    3y_1 & +3y_2 & +3y_3 & -4y_4 & \geq & -5 \\
                                    -y_1 & +y_2 & +y_3 & -y_4 & \geq & -4 \\
                                \end{array} \\
                                y_1, y_2, y_3, y_4 \geq 0
                            \end{gathered}
                        \end{equation}

                    \item
                        Eine optimale Lösung für $(D)$ ist
                        \begin{equation}
                            y^* = \left( y_1^*, y_2^*, y_3^*, y_4^* \right)
                            = \left( 0, \frac{21}{29}, 0, \frac{52}{29}  \right)
                        \end{equation}
                        
                    \item
                        Alle $y_i^*$ sind $\geq 0$.
                        \begin{align}
                            5 \cdot \frac{21}{29} + 3 \cdot \frac{52}{29} = \frac{261}{29} &\geq \frac{261}{29} = 9 \\
                            3 \cdot \frac{21}{29} - 4 \cdot \frac{52}{29} = -\frac{145}{29} &\geq -\frac{145}{29} = -5 \\
                            \frac{21}{29} -\frac{52}{29} = -\frac{31}{29} &\geq -\frac{116}{29} = -4
                        \end{align}
                        
                    \item
                        \begin{equation}
                            \sum_{j=1}^3 c_jx_j^* = 9 \cdot \frac{30}{29} - 5 \cdot \frac{8}{29} = \frac{230}{29} = 6 \cdot \frac{21}{29} + 2 \cdot \frac{52}{29} = \sum_{i=1}^4 b_iy_i^*
                        \end{equation}

                    \item
                        \begin{enumerate}
                            \item
                                \begin{itemize}
                                    \item $\sum_{i=1}^4 a_{i1}y_i^* = c_1$ (siehe (iii))

                                    \item $\sum_{i=1}^4 a_{i2}y_i^* = c_2$ (siehe (iii))

                                    \item $x_3^* = 0$

                                \end{itemize}

                            \item
                                \begin{itemize}
                                    \item $y_1^* = 0$

                                    \item $\sum_{j=1}^3 a_{2j}x_j^* = b_2$ \\
                                        \begin{equation}
                                            5 \cdot x_1^* + 3 \cdot x_2^* + x_3^* = \frac{150}{29} + \frac{24}{29} = \frac{174}{29} = 6
                                        \end{equation}
                                        
                                    \item $y_3^* = 0$

                                    \item $\sum_{j=1}^3 a_{4j}x_j^* = b_4$ (siehe (iii))
                                        \begin{equation}
                                            3 \cdot x_1^* - 4 \cdot x_2^* - x_3^* = \frac{90}{29} - \frac{32}{29} = \frac{58}{29} = 2
                                        \end{equation}
                                        
                                \end{itemize}
                        \end{enumerate}
                        

                \end{enumerate}

            \item
                \begin{enumerate}[(i)]
                    \item

                    \item

                    \item

                    \item

                    \item

                \end{enumerate}

        \end{enumerate}

    \item % 2.


\end{enumerate}


\end{document}
