\documentclass[a4paper]{scrartcl}

% font/encoding packages
\usepackage[utf8]{inputenc}
\usepackage[T1]{fontenc}
\usepackage{lmodern}
\usepackage[ngerman]{babel}

\usepackage[top=1.1in, bottom=1.5in]{geometry}

\usepackage{amsmath, amssymb, amsfonts, amsthm, mathtools}
\usepackage{stmaryrd}
\usepackage{marvosym}
\setcounter{MaxMatrixCols}{13}
\allowdisplaybreaks
\usepackage[output-decimal-marker={,}]{siunitx}
\usepackage{graphicx}
\usepackage{subcaption}
\usepackage{caption}
\usepackage{venndiagram}
\usepackage[shortlabels]{enumitem}
\usepackage{polynom}
\polyset{style=C, div=:,vars=x}
\usepackage[section]{placeins}
\usepackage{tikz}
\usetikzlibrary{tikzmark}
\usetikzlibrary{calc}
\usetikzlibrary{arrows}
\usetikzlibrary{arrows.meta}
\usetikzlibrary{decorations}
\usepackage{pgfplots}
\pgfplotsset{compat=1.5.1}
\usepgfplotslibrary{fillbetween}
\usepackage{gauss}
\usepackage{systeme}
\usepackage{cancel}

\tikzset{
    declare function={Floor(\x)=round(\x-0.5);}
}

% Matrizen
\newdimen\asep 
\asep=0.25\baselineskip 

\providecommand{\vo}{ 
    \text{\raisebox{-\asep}[0pt][0pt]{\rule[0pt]{0.8pt}{3.5\asep}}}} 
\providecommand{\vu}{ 
    \text{\raisebox{-0.5\asep}[0pt][0pt]{\rule[0pt]{0.8pt}{\baselineskip}}}} 
\providecommand{\vm}{ 
    \text{\raisebox{-\asep}[0pt][0pt]{\rule[0pt]{0.8pt}{\baselineskip}}}}

\newtheorem*{behaupt}{Behauptung}

\newcommand{\fallfac}[2]{{#1}^{\underline{#2}}}

\newcommand{\risefac}[2]{{#1}^{\overline{#2}}}

\newcommand{\dif}{\mathop{}\!\mathrm{d}}

\newcommand{\linf}[1]{\lim_{#1 \to \infty}}
\newcommand{\gdw}{\Leftrightarrow}

\DeclareSIUnit\calorie{cal}

\renewcommand{\arraystretch}{1.3}

\title{Optimierung}
\subtitle{Blatt 11 Hausaufgaben}
\author{
	Lennart Braun (6523742, Gruppe 6) \\
    Alexander Timmermann (6524072, Gruppe 5)
}
\date{zum 12. Januar 2015}

\begin{document}
\maketitle

\begin{enumerate}[label=\bfseries\arabic*.]
    \item
        \begin{enumerate}
            \item
                \begin{enumerate}[(i)]
                    \item
                        Die Knoten werden in folgender Reihenfolge markiert:
                        $s, a, c, d, e, f, b, t$.
                        Es gibt keine Knoten, die unmarkiert bleiben.

                    \item
                        \begin{align}
                            P &= sacft \\
                            f_2(s,a) &= 27 \\
                            f_2(a, c) &= 10 \\
                            f_2(c, f) &= 13 \\
                            f_2(f, t) &= 15 \\
                            w(f_2) &= 30
                        \end{align}

                \end{enumerate}

            \item
                Knoten, die während des letzten Versuchs, einen augmentierenden
                Pfad zu finden, markiert wurden, gehören zu $S$; die
                unmarkierten gehören zu $T$.

        \end{enumerate}

    \item
        \begin{enumerate}
            \item \hfill \\
                \begin{figure}[h]
                    \centering
                    \begin{tikzpicture}[
                        auto,
                        scale=2,
                    ]
                        \tikzstyle{vertex}=[
                            circle,
                            fill,
                            minimum size=0.005cm,
                            align=center,
                        ]
                        \tikzstyle{edge}=[
                        ]
                        \tikzstyle{match}=[
                            decorate,
                            decoration={snake, segment length=2mm},
                        ]

                        \node[vertex, label=below:$y_1$] (y1) at (0,0) {};
                        \node[vertex, label=below:$y_2$] (y2) at (1,0) {};
                        \node[vertex, label=below:$y_3$] (y3) at (2,0) {};
                        \node[vertex, label=below:$y_4$] (y4) at (3,0) {};
                        \node[vertex, label=below:$y_5$] (y5) at (4,0) {};
                        \node[vertex, label=below:$y_6$] (y6) at (5,0) {};

                        \node[vertex, label=above:$x_1$] (x1) at (0,2) {};
                        \node[vertex, label=above:$x_2$] (x2) at (1,2) {};
                        \node[vertex, label=above:$x_3$] (x3) at (2,2) {};
                        \node[vertex, label=above:$x_4$] (x4) at (3,2) {};
                        \node[vertex, label=above:$x_5$] (x5) at (4,2) {};
                        \node[vertex, label=above:$x_6$] (x6) at (5,2) {};
                        \node[vertex, label=above:$x_7$] (x7) at (6,2) {};

                        \draw[match]      (x1) to (y1);
                        \draw[edge]       (x1) to (y2);
                        \draw[edge]       (x1) to (y3);
                        \draw[edge]       (x1) to (y4);

                        \draw[edge]       (x2) to (y1);
                        \draw[edge]       (x2) to (y4);

                        \draw[edge]       (x3) to (y1);
                        \draw[edge]       (x3) to (y5);

                        \draw[edge]       (x4) to (y2);
                        \draw[edge]       (x4) to (y3);
                        \draw[edge]       (x4) to (y4);
                        \draw[edge]       (x4) to (y5);
                        \draw[edge]       (x4) to (y6);

                        \draw[edge]       (x5) to (y1);
                        \draw[edge]       (x5) to (y5);

                        \draw[edge]       (x6) to (y2);
                        \draw[edge]       (x6) to (y3);
                        \draw[edge]       (x6) to (y4);
                        \draw[edge]       (x6) to (y5);
                        \draw[edge]       (x6) to (y6);

                        \draw[edge]       (x7) to (y3);
                        \draw[edge]       (x7) to (y5);

                    \end{tikzpicture}
                    \caption{$M_1 = \left\{ \{x_1, y_1\} \right\}$}
                    \label{fig:a-1}
                \end{figure}

                \begin{figure}[h]
                    \centering
                    \begin{tikzpicture}[
                        auto,
                        scale=2,
                    ]
                        \tikzstyle{vertex}=[
                            circle,
                            fill,
                            minimum size=0.005cm,
                            align=center,
                        ]
                        \tikzstyle{edge}=[
                        ]
                        \tikzstyle{match}=[
                            decorate,
                            decoration={snake, segment length=2mm},
                        ]

                        \node[vertex, label=below:$y_1$] (y1) at (0,0) {};
                        \node[vertex, label=below:$y_2$] (y2) at (1,0) {};
                        \node[vertex, label=below:$y_3$] (y3) at (2,0) {};
                        \node[vertex, label=below:$y_4$] (y4) at (3,0) {};
                        \node[vertex, label=below:$y_5$] (y5) at (4,0) {};
                        \node[vertex, label=below:$y_6$] (y6) at (5,0) {};

                        \node[vertex, label=above:$x_1$] (x1) at (0,2) {};
                        \node[vertex, label=above:$x_2$] (x2) at (1,2) {};
                        \node[vertex, label=above:$x_3$] (x3) at (2,2) {};
                        \node[vertex, label=above:$x_4$] (x4) at (3,2) {};
                        \node[vertex, label=above:$x_5$] (x5) at (4,2) {};
                        \node[vertex, label=above:$x_6$] (x6) at (5,2) {};
                        \node[vertex, label=above:$x_7$] (x7) at (6,2) {};

                        \draw[match]      (x1) to (y1);
                        \draw[edge]       (x1) to (y2);
                        \draw[edge]       (x1) to (y3);
                        \draw[edge]       (x1) to (y4);

                        \draw[edge]       (x2) to (y1);
                        \draw[match]      (x2) to (y4);

                        \draw[edge]       (x3) to (y1);
                        \draw[edge]       (x3) to (y5);

                        \draw[edge]       (x4) to (y2);
                        \draw[edge]       (x4) to (y3);
                        \draw[edge]       (x4) to (y4);
                        \draw[edge]       (x4) to (y5);
                        \draw[edge]       (x4) to (y6);

                        \draw[edge]       (x5) to (y1);
                        \draw[edge]       (x5) to (y5);

                        \draw[edge]       (x6) to (y2);
                        \draw[edge]       (x6) to (y3);
                        \draw[edge]       (x6) to (y4);
                        \draw[edge]       (x6) to (y5);
                        \draw[edge]       (x6) to (y6);

                        \draw[edge]       (x7) to (y3);
                        \draw[edge]       (x7) to (y5);

                    \end{tikzpicture}
                    \caption{$M_2 = \left\{ \{x_1, y_1\}, \{x_2, y_4\} \right\}$}
                    \label{fig:a-2}
                \end{figure}

                \begin{figure}[h]
                    \centering
                    \begin{tikzpicture}[
                        auto,
                        scale=2,
                    ]
                        \tikzstyle{vertex}=[
                            circle,
                            fill,
                            minimum size=0.005cm,
                            align=center,
                        ]
                        \tikzstyle{edge}=[
                        ]
                        \tikzstyle{match}=[
                            decorate,
                            decoration={snake, segment length=2mm},
                        ]

                        \node[vertex, label=below:$y_1$] (y1) at (0,0) {};
                        \node[vertex, label=below:$y_2$] (y2) at (1,0) {};
                        \node[vertex, label=below:$y_3$] (y3) at (2,0) {};
                        \node[vertex, label=below:$y_4$] (y4) at (3,0) {};
                        \node[vertex, label=below:$y_5$] (y5) at (4,0) {};
                        \node[vertex, label=below:$y_6$] (y6) at (5,0) {};

                        \node[vertex, label=above:$x_1$] (x1) at (0,2) {};
                        \node[vertex, label=above:$x_2$] (x2) at (1,2) {};
                        \node[vertex, label=above:$x_3$] (x3) at (2,2) {};
                        \node[vertex, label=above:$x_4$] (x4) at (3,2) {};
                        \node[vertex, label=above:$x_5$] (x5) at (4,2) {};
                        \node[vertex, label=above:$x_6$] (x6) at (5,2) {};
                        \node[vertex, label=above:$x_7$] (x7) at (6,2) {};

                        \draw[match]      (x1) to (y1);
                        \draw[edge]       (x1) to (y2);
                        \draw[edge]       (x1) to (y3);
                        \draw[edge]       (x1) to (y4);

                        \draw[edge]       (x2) to (y1);
                        \draw[match]      (x2) to (y4);

                        \draw[edge]       (x3) to (y1);
                        \draw[match]      (x3) to (y5);

                        \draw[edge]       (x4) to (y2);
                        \draw[edge]       (x4) to (y3);
                        \draw[edge]       (x4) to (y4);
                        \draw[edge]       (x4) to (y5);
                        \draw[edge]       (x4) to (y6);

                        \draw[edge]       (x5) to (y1);
                        \draw[edge]       (x5) to (y5);

                        \draw[edge]       (x6) to (y2);
                        \draw[edge]       (x6) to (y3);
                        \draw[edge]       (x6) to (y4);
                        \draw[edge]       (x6) to (y5);
                        \draw[edge]       (x6) to (y6);

                        \draw[edge]       (x7) to (y3);
                        \draw[edge]       (x7) to (y5);

                    \end{tikzpicture}
                    \caption{$M_3 = \left\{ \{x_1, y_1\}, \{x_2, y_4\}, 
                             \{x_3, y_5\} \right\}$}
                    \label{fig:a-3}
                \end{figure}

                \begin{figure}[h]
                    \centering
                    \begin{tikzpicture}[
                        auto,
                        scale=2,
                    ]
                        \tikzstyle{vertex}=[
                            circle,
                            fill,
                            minimum size=0.005cm,
                            align=center,
                        ]
                        \tikzstyle{edge}=[
                        ]
                        \tikzstyle{match}=[
                            decorate,
                            decoration={snake, segment length=2mm},
                        ]

                        \node[vertex, label=below:$y_1$] (y1) at (0,0) {};
                        \node[vertex, label=below:$y_2$] (y2) at (1,0) {};
                        \node[vertex, label=below:$y_3$] (y3) at (2,0) {};
                        \node[vertex, label=below:$y_4$] (y4) at (3,0) {};
                        \node[vertex, label=below:$y_5$] (y5) at (4,0) {};
                        \node[vertex, label=below:$y_6$] (y6) at (5,0) {};

                        \node[vertex, label=above:$x_1$] (x1) at (0,2) {};
                        \node[vertex, label=above:$x_2$] (x2) at (1,2) {};
                        \node[vertex, label=above:$x_3$] (x3) at (2,2) {};
                        \node[vertex, label=above:$x_4$] (x4) at (3,2) {};
                        \node[vertex, label=above:$x_5$] (x5) at (4,2) {};
                        \node[vertex, label=above:$x_6$] (x6) at (5,2) {};
                        \node[vertex, label=above:$x_7$] (x7) at (6,2) {};

                        \draw[match]      (x1) to (y1);
                        \draw[edge]       (x1) to (y2);
                        \draw[edge]       (x1) to (y3);
                        \draw[edge]       (x1) to (y4);

                        \draw[edge]       (x2) to (y1);
                        \draw[match]      (x2) to (y4);

                        \draw[edge]       (x3) to (y1);
                        \draw[match]      (x3) to (y5);

                        \draw[match]      (x4) to (y2);
                        \draw[edge]       (x4) to (y3);
                        \draw[edge]       (x4) to (y4);
                        \draw[edge]       (x4) to (y5);
                        \draw[edge]       (x4) to (y6);

                        \draw[edge]       (x5) to (y1);
                        \draw[edge]       (x5) to (y5);

                        \draw[edge]       (x6) to (y2);
                        \draw[edge]       (x6) to (y3);
                        \draw[edge]       (x6) to (y4);
                        \draw[edge]       (x6) to (y5);
                        \draw[edge]       (x6) to (y6);

                        \draw[edge]       (x7) to (y3);
                        \draw[edge]       (x7) to (y5);

                    \end{tikzpicture}
                    \caption{$M_4 = \left\{ \{x_1, y_1\}, \{x_2, y_4\}, 
                        \{x_3, y_5\}, \{x_4, y_2\} \right\}$}
                    \label{fig:a-4}
                \end{figure}

                \begin{figure}[h]
                    \centering
                    \begin{tikzpicture}[
                        auto,
                        scale=2,
                    ]
                        \tikzstyle{vertex}=[
                            circle,
                            fill,
                            minimum size=0.005cm,
                            align=center,
                        ]
                        \tikzstyle{edge}=[
                        ]
                        \tikzstyle{match}=[
                            decorate,
                            decoration={snake, segment length=2mm},
                        ]

                        \node[vertex, label=below:$y_1$] (y1) at (0,0) {};
                        \node[vertex, label=below:$y_2$] (y2) at (1,0) {};
                        \node[vertex, label=below:$y_3$] (y3) at (2,0) {};
                        \node[vertex, label=below:$y_4$] (y4) at (3,0) {};
                        \node[vertex, label=below:$y_5$] (y5) at (4,0) {};
                        \node[vertex, label=below:$y_6$] (y6) at (5,0) {};

                        \node[vertex, label=above:$x_1$] (x1) at (0,2) {};
                        \node[vertex, label=above:$x_2$] (x2) at (1,2) {};
                        \node[vertex, label=above:$x_3$] (x3) at (2,2) {};
                        \node[vertex, label=above:$x_4$] (x4) at (3,2) {};
                        \node[vertex, label=above:$x_5$] (x5) at (4,2) {};
                        \node[vertex, label=above:$x_6$] (x6) at (5,2) {};
                        \node[vertex, label=above:$x_7$] (x7) at (6,2) {};

                        \draw[match]      (x1) to (y1);
                        \draw[edge]       (x1) to (y2);
                        \draw[edge]       (x1) to (y3);
                        \draw[edge]       (x1) to (y4);

                        \draw[edge]       (x2) to (y1);
                        \draw[match]      (x2) to (y4);

                        \draw[edge]       (x3) to (y1);
                        \draw[match]      (x3) to (y5);

                        \draw[match]      (x4) to (y2);
                        \draw[edge]       (x4) to (y3);
                        \draw[edge]       (x4) to (y4);
                        \draw[edge]       (x4) to (y5);
                        \draw[edge]       (x4) to (y6);

                        \draw[edge]       (x5) to (y1);
                        \draw[edge]       (x5) to (y5);

                        \draw[edge]       (x6) to (y2);
                        \draw[match]      (x6) to (y3);
                        \draw[edge]       (x6) to (y4);
                        \draw[edge]       (x6) to (y5);
                        \draw[edge]       (x6) to (y6);

                        \draw[edge]       (x7) to (y3);
                        \draw[edge]       (x7) to (y5);

                    \end{tikzpicture}
                    \caption{$M_5 = \left\{ \{x_1, y_1\}, \{x_2, y_4\}, 
                    \{x_3, y_5\}, \{x_4, y_2\}, \{x_6, y_3\} \right\}$}
                    \label{fig:a-5}
                \end{figure}

                \begin{figure}[h]
                    \centering
                    \begin{tikzpicture}[
                        auto,
                        scale=2,
                    ]
                        \tikzstyle{vertex}=[
                            circle,
                            fill,
                            minimum size=0.005cm,
                            align=center,
                        ]
                        \tikzstyle{cover}=[
                            draw,
                            fill=none,
                        ]

                        \tikzstyle{edge}=[
                        ]
                        \tikzstyle{match}=[
                            decorate,
                            decoration={snake, segment length=2mm},
                        ]

                        \node[vertex, label=below:$y_1$, cover] (y1) at (0,0) {};
                        \node[vertex, label=below:$y_2$, cover] (y2) at (1,0) {};
                        \node[vertex, label=below:$y_3$, cover] (y3) at (2,0) {};
                        \node[vertex, label=below:$y_4$, cover] (y4) at (3,0) {};
                        \node[vertex, label=below:$y_5$, cover] (y5) at (4,0) {};
                        \node[vertex, label=below:$y_6$, cover] (y6) at (5,0) {};

                        \node[vertex, label=above:$x_1$] (x1) at (0,2) {};
                        \node[vertex, label=above:$x_2$] (x2) at (1,2) {};
                        \node[vertex, label=above:$x_3$] (x3) at (2,2) {};
                        \node[vertex, label=above:$x_4$] (x4) at (3,2) {};
                        \node[vertex, label=above:$x_5$] (x5) at (4,2) {};
                        \node[vertex, label=above:$x_6$] (x6) at (5,2) {};
                        \node[vertex, label=above:$x_7$] (x7) at (6,2) {};

                        \draw[match]      (x1) to (y1);
                        \draw[edge]       (x1) to (y2);
                        \draw[edge]       (x1) to (y3);
                        \draw[edge]       (x1) to (y4);

                        \draw[edge]       (x2) to (y1);
                        \draw[match]      (x2) to (y4);

                        \draw[edge]       (x3) to (y1);
                        \draw[match]      (x3) to (y5);

                        \draw[match]      (x4) to (y2);
                        \draw[edge]       (x4) to (y3);
                        \draw[edge]       (x4) to (y4);
                        \draw[edge]       (x4) to (y5);
                        \draw[edge]       (x4) to (y6);

                        \draw[edge]       (x5) to (y1);
                        \draw[edge]       (x5) to (y5);

                        \draw[edge]       (x6) to (y2);
                        \draw[edge]       (x6) to (y3);
                        \draw[edge]       (x6) to (y4);
                        \draw[edge]       (x6) to (y5);
                        \draw[match]      (x6) to (y6);

                        \draw[match]      (x7) to (y3);
                        \draw[edge]       (x7) to (y5);

                    \end{tikzpicture}
                    \caption{$M_6 = \left\{ \{x_1, y_1\}, \{x_2, y_4\}, 
                    \{x_3, y_5\}, \{x_4, y_2\}, \{x_7, y_3\}, \{x_6, y_6\}
                    \right\}$ \\ (Überdeckende Knoten weiß gefüllt.)}
                    \label{fig:a-6}
                \end{figure}

                \FloatBarrier

                In der 7. Iteration gelingt es nicht, einen augmentierenden Pfad
                zu finden.
                Dabei werden die folgenden Knoten mit alternierenden Pfaden
                erreicht und markiert:
                $x_5, y_1, y_5, x_1, x_3, y_2, y_3, y_4, x_4, x_7, x_2, y_6, x_6$

                \paragraph{Ergebnis:}
                Der Algorithmus liefert das Matching $M$ in Abb. \ref{fig:a-6}
                mit $6$ Kanten zusammen mit der minimalen Knotenüberdeckung
                \begin{equation}
                    U = \left\{ y_1, y_2, y_3, y_4, y_5, y_6 \right\} \text{ .}
                \end{equation}

            \FloatBarrier

            \item \hfill \\
                \begin{figure}[h]
                    \centering
                    \begin{tikzpicture}[
                        auto,
                        scale=2,
                    ]
                        \tikzstyle{vertex}=[
                            circle,
                            fill,
                            minimum size=0.005cm,
                            align=center,
                        ]
                        \tikzstyle{edge}=[
                        ]
                        \tikzstyle{match}=[
                            decorate,
                            decoration={snake, segment length=2mm},
                        ]

                        \node[vertex, label=below:$y_1$] (y1) at (0,0) {};
                        \node[vertex, label=below:$y_2$] (y2) at (1,0) {};
                        \node[vertex, label=below:$y_3$] (y3) at (2,0) {};
                        \node[vertex, label=below:$y_4$] (y4) at (3,0) {};
                        \node[vertex, label=below:$y_5$] (y5) at (4,0) {};
                        \node[vertex, label=below:$y_6$] (y6) at (5,0) {};

                        \node[vertex, label=above:$x_1$] (x1) at (0,2) {};
                        \node[vertex, label=above:$x_2$] (x2) at (1,2) {};
                        \node[vertex, label=above:$x_3$] (x3) at (2,2) {};
                        \node[vertex, label=above:$x_4$] (x4) at (3,2) {};
                        \node[vertex, label=above:$x_5$] (x5) at (4,2) {};
                        \node[vertex, label=above:$x_6$] (x6) at (5,2) {};
                        \node[vertex, label=above:$x_7$] (x7) at (6,2) {};

                        \draw[match]      (x1) to (y1);
                        \draw[edge]       (x1) to (y3);
                        \draw[edge]       (x1) to (y4);
                        \draw[edge]       (x1) to (y5);

                        \draw[edge]       (x2) to (y1);
                        \draw[edge]       (x2) to (y2);
                        \draw[edge]       (x2) to (y3);
                        \draw[edge]       (x2) to (y5);

                        \draw[edge]       (x3) to (y1);
                        \draw[edge]       (x3) to (y2);
                        \draw[edge]       (x3) to (y4);

                        \draw[edge]       (x4) to (y2);
                        \draw[edge]       (x4) to (y4);

                        \draw[edge]       (x5) to (y2);
                        \draw[edge]       (x5) to (y6);

                        \draw[edge]       (x6) to (y1);
                        \draw[edge]       (x6) to (y4);
                        \draw[edge]       (x6) to (y6);

                        \draw[edge]       (x7) to (y1);
                        \draw[edge]       (x7) to (y4);
                        \draw[edge]       (x7) to (y6);

                    \end{tikzpicture}
                    \caption{$M_1 = \left\{ \{x_1, y_1\} \right\}$}
                    \label{fig:b-1}
                \end{figure}

                \begin{figure}[h]
                    \centering
                    \begin{tikzpicture}[
                        auto,
                        scale=2,
                    ]
                        \tikzstyle{vertex}=[
                            circle,
                            fill,
                            minimum size=0.005cm,
                            align=center,
                        ]
                        \tikzstyle{edge}=[
                        ]
                        \tikzstyle{match}=[
                            decorate,
                            decoration={snake, segment length=2mm},
                        ]

                        \node[vertex, label=below:$y_1$] (y1) at (0,0) {};
                        \node[vertex, label=below:$y_2$] (y2) at (1,0) {};
                        \node[vertex, label=below:$y_3$] (y3) at (2,0) {};
                        \node[vertex, label=below:$y_4$] (y4) at (3,0) {};
                        \node[vertex, label=below:$y_5$] (y5) at (4,0) {};
                        \node[vertex, label=below:$y_6$] (y6) at (5,0) {};

                        \node[vertex, label=above:$x_1$] (x1) at (0,2) {};
                        \node[vertex, label=above:$x_2$] (x2) at (1,2) {};
                        \node[vertex, label=above:$x_3$] (x3) at (2,2) {};
                        \node[vertex, label=above:$x_4$] (x4) at (3,2) {};
                        \node[vertex, label=above:$x_5$] (x5) at (4,2) {};
                        \node[vertex, label=above:$x_6$] (x6) at (5,2) {};
                        \node[vertex, label=above:$x_7$] (x7) at (6,2) {};

                        \draw[match]      (x1) to (y1);
                        \draw[edge]       (x1) to (y3);
                        \draw[edge]       (x1) to (y4);
                        \draw[edge]       (x1) to (y5);

                        \draw[edge]       (x2) to (y1);
                        \draw[match]      (x2) to (y2);
                        \draw[edge]       (x2) to (y3);
                        \draw[edge]       (x2) to (y5);

                        \draw[edge]       (x3) to (y1);
                        \draw[edge]       (x3) to (y2);
                        \draw[edge]       (x3) to (y4);

                        \draw[edge]       (x4) to (y2);
                        \draw[edge]       (x4) to (y4);

                        \draw[edge]       (x5) to (y2);
                        \draw[edge]       (x5) to (y6);

                        \draw[edge]       (x6) to (y1);
                        \draw[edge]       (x6) to (y4);
                        \draw[edge]       (x6) to (y6);

                        \draw[edge]       (x7) to (y1);
                        \draw[edge]       (x7) to (y4);
                        \draw[edge]       (x7) to (y6);

                    \end{tikzpicture}
                    \caption{$M_2 = \left\{ \{x_1, y_1\}, \{x_2, y_2\}
                             \right\}$}
                    \label{fig:b-2}
                \end{figure}

                \begin{figure}[h]
                    \centering
                    \begin{tikzpicture}[
                        auto,
                        scale=2,
                    ]
                        \tikzstyle{vertex}=[
                            circle,
                            fill,
                            minimum size=0.005cm,
                            align=center,
                        ]
                        \tikzstyle{edge}=[
                        ]
                        \tikzstyle{match}=[
                            decorate,
                            decoration={snake, segment length=2mm},
                        ]

                        \node[vertex, label=below:$y_1$] (y1) at (0,0) {};
                        \node[vertex, label=below:$y_2$] (y2) at (1,0) {};
                        \node[vertex, label=below:$y_3$] (y3) at (2,0) {};
                        \node[vertex, label=below:$y_4$] (y4) at (3,0) {};
                        \node[vertex, label=below:$y_5$] (y5) at (4,0) {};
                        \node[vertex, label=below:$y_6$] (y6) at (5,0) {};

                        \node[vertex, label=above:$x_1$] (x1) at (0,2) {};
                        \node[vertex, label=above:$x_2$] (x2) at (1,2) {};
                        \node[vertex, label=above:$x_3$] (x3) at (2,2) {};
                        \node[vertex, label=above:$x_4$] (x4) at (3,2) {};
                        \node[vertex, label=above:$x_5$] (x5) at (4,2) {};
                        \node[vertex, label=above:$x_6$] (x6) at (5,2) {};
                        \node[vertex, label=above:$x_7$] (x7) at (6,2) {};

                        \draw[match]      (x1) to (y1);
                        \draw[edge]       (x1) to (y3);
                        \draw[edge]       (x1) to (y4);
                        \draw[edge]       (x1) to (y5);

                        \draw[edge]       (x2) to (y1);
                        \draw[match]      (x2) to (y2);
                        \draw[edge]       (x2) to (y3);
                        \draw[edge]       (x2) to (y5);

                        \draw[edge]       (x3) to (y1);
                        \draw[edge]       (x3) to (y2);
                        \draw[match]      (x3) to (y4);

                        \draw[edge]       (x4) to (y2);
                        \draw[edge]       (x4) to (y4);

                        \draw[edge]       (x5) to (y2);
                        \draw[edge]       (x5) to (y6);

                        \draw[edge]       (x6) to (y1);
                        \draw[edge]       (x6) to (y4);
                        \draw[edge]       (x6) to (y6);

                        \draw[edge]       (x7) to (y1);
                        \draw[edge]       (x7) to (y4);
                        \draw[edge]       (x7) to (y6);

                    \end{tikzpicture}
                    \caption{$M_3 = \left\{ \{x_1, y_1\}, \{x_2, y_2\},
                    \{x_3, x_4\} \right\}$}
                    \label{fig:b-3}
                \end{figure}

                \begin{figure}[h]
                    \centering
                    \begin{tikzpicture}[
                        auto,
                        scale=2,
                    ]
                        \tikzstyle{vertex}=[
                            circle,
                            fill,
                            minimum size=0.005cm,
                            align=center,
                        ]
                        \tikzstyle{edge}=[
                        ]
                        \tikzstyle{match}=[
                            decorate,
                            decoration={snake, segment length=2mm},
                        ]

                        \node[vertex, label=below:$y_1$] (y1) at (0,0) {};
                        \node[vertex, label=below:$y_2$] (y2) at (1,0) {};
                        \node[vertex, label=below:$y_3$] (y3) at (2,0) {};
                        \node[vertex, label=below:$y_4$] (y4) at (3,0) {};
                        \node[vertex, label=below:$y_5$] (y5) at (4,0) {};
                        \node[vertex, label=below:$y_6$] (y6) at (5,0) {};

                        \node[vertex, label=above:$x_1$] (x1) at (0,2) {};
                        \node[vertex, label=above:$x_2$] (x2) at (1,2) {};
                        \node[vertex, label=above:$x_3$] (x3) at (2,2) {};
                        \node[vertex, label=above:$x_4$] (x4) at (3,2) {};
                        \node[vertex, label=above:$x_5$] (x5) at (4,2) {};
                        \node[vertex, label=above:$x_6$] (x6) at (5,2) {};
                        \node[vertex, label=above:$x_7$] (x7) at (6,2) {};

                        \draw[match]      (x1) to (y1);
                        \draw[edge]       (x1) to (y3);
                        \draw[edge]       (x1) to (y4);
                        \draw[edge]       (x1) to (y5);

                        \draw[edge]       (x2) to (y1);
                        \draw[match]      (x2) to (y2);
                        \draw[edge]       (x2) to (y3);
                        \draw[edge]       (x2) to (y5);

                        \draw[edge]       (x3) to (y1);
                        \draw[edge]       (x3) to (y2);
                        \draw[match]      (x3) to (y4);

                        \draw[edge]       (x4) to (y2);
                        \draw[edge]       (x4) to (y4);

                        \draw[edge]       (x5) to (y2);
                        \draw[match]      (x5) to (y6);

                        \draw[edge]       (x6) to (y1);
                        \draw[edge]       (x6) to (y4);
                        \draw[edge]       (x6) to (y6);

                        \draw[edge]       (x7) to (y1);
                        \draw[edge]       (x7) to (y4);
                        \draw[edge]       (x7) to (y6);

                    \end{tikzpicture}
                    \caption{$M_4 = \left\{ \{x_1, y_1\}, \{x_2, y_2\},
                    \{x_3, x_4\}, \{x_5, y_6\} \right\}$}
                    \label{fig:b-4}
                \end{figure}

                \begin{figure}[h]
                    \centering
                    \begin{tikzpicture}[
                        auto,
                        scale=2,
                    ]
                        \tikzstyle{vertex}=[
                            circle,
                            fill,
                            minimum size=0.005cm,
                            align=center,
                        ]
                        \tikzstyle{edge}=[
                        ]
                        \tikzstyle{match}=[
                            decorate,
                            decoration={snake, segment length=2mm},
                        ]

                        \node[vertex, label=below:$y_1$] (y1) at (0,0) {};
                        \node[vertex, label=below:$y_2$] (y2) at (1,0) {};
                        \node[vertex, label=below:$y_3$] (y3) at (2,0) {};
                        \node[vertex, label=below:$y_4$] (y4) at (3,0) {};
                        \node[vertex, label=below:$y_5$] (y5) at (4,0) {};
                        \node[vertex, label=below:$y_6$] (y6) at (5,0) {};

                        \node[vertex, label=above:$x_1$] (x1) at (0,2) {};
                        \node[vertex, label=above:$x_2$] (x2) at (1,2) {};
                        \node[vertex, label=above:$x_3$] (x3) at (2,2) {};
                        \node[vertex, label=above:$x_4$] (x4) at (3,2) {};
                        \node[vertex, label=above:$x_5$] (x5) at (4,2) {};
                        \node[vertex, label=above:$x_6$] (x6) at (5,2) {};
                        \node[vertex, label=above:$x_7$] (x7) at (6,2) {};

                        \draw[match]      (x1) to (y1);
                        \draw[edge]       (x1) to (y3);
                        \draw[edge]       (x1) to (y4);
                        \draw[edge]       (x1) to (y5);

                        \draw[edge]       (x2) to (y1);
                        \draw[edge]       (x2) to (y2);
                        \draw[match]      (x2) to (y3);
                        \draw[edge]       (x2) to (y5);

                        \draw[edge]       (x3) to (y1);
                        \draw[edge]       (x3) to (y2);
                        \draw[match]      (x3) to (y4);

                        \draw[match]      (x4) to (y2);
                        \draw[edge]       (x4) to (y4);

                        \draw[edge]       (x5) to (y2);
                        \draw[match]      (x5) to (y6);

                        \draw[edge]       (x6) to (y1);
                        \draw[edge]       (x6) to (y4);
                        \draw[edge]       (x6) to (y6);

                        \draw[edge]       (x7) to (y1);
                        \draw[edge]       (x7) to (y4);
                        \draw[edge]       (x7) to (y6);

                    \end{tikzpicture}
                    \caption{$M_5 = \left\{ \{x_1, y_1\}, \{x_3, x_4\},
                    \{x_5, y_6\}, \{x_4, y_2\}, \{x_2, y_3\} \right\}$}
                    \label{fig:b-5}
                \end{figure}

                \begin{figure}[h]
                    \centering
                    \begin{tikzpicture}[
                        auto,
                        scale=2,
                    ]
                        \tikzstyle{vertex}=[
                            circle,
                            fill,
                            minimum size=0.005cm,
                            align=center,
                        ]
                        \tikzstyle{cover}=[
                            draw,
                            fill=none,
                        ]
                        
                        \tikzstyle{edge}=[
                        ]
                        \tikzstyle{match}=[
                            decorate,
                            decoration={snake, segment length=2mm},
                        ]

                        \node[vertex, label=below:$y_1$, cover] (y1) at (0,0) {};
                        \node[vertex, label=below:$y_2$, cover] (y2) at (1,0) {};
                        \node[vertex, label=below:$y_3$] (y3) at (2,0) {};
                        \node[vertex, label=below:$y_4$, cover] (y4) at (3,0) {};
                        \node[vertex, label=below:$y_5$] (y5) at (4,0) {};
                        \node[vertex, label=below:$y_6$, cover] (y6) at (5,0) {};

                        \node[vertex, label=above:$x_1$, cover] (x1) at (0,2) {};
                        \node[vertex, label=above:$x_2$, cover] (x2) at (1,2) {};
                        \node[vertex, label=above:$x_3$] (x3) at (2,2) {};
                        \node[vertex, label=above:$x_4$] (x4) at (3,2) {};
                        \node[vertex, label=above:$x_5$] (x5) at (4,2) {};
                        \node[vertex, label=above:$x_6$] (x6) at (5,2) {};
                        \node[vertex, label=above:$x_7$] (x7) at (6,2) {};

                        \draw[edge]       (x1) to (y1);
                        \draw[edge]       (x1) to (y3);
                        \draw[edge]       (x1) to (y4);
                        \draw[match]      (x1) to (y5);

                        \draw[edge]       (x2) to (y1);
                        \draw[edge]       (x2) to (y2);
                        \draw[match]      (x2) to (y3);
                        \draw[edge]       (x2) to (y5);

                        \draw[edge]       (x3) to (y1);
                        \draw[edge]       (x3) to (y2);
                        \draw[match]      (x3) to (y4);

                        \draw[match]      (x4) to (y2);
                        \draw[edge]       (x4) to (y4);

                        \draw[edge]       (x5) to (y2);
                        \draw[match]      (x5) to (y6);

                        \draw[match]      (x6) to (y1);
                        \draw[edge]       (x6) to (y4);
                        \draw[edge]       (x6) to (y6);

                        \draw[edge]       (x7) to (y1);
                        \draw[edge]       (x7) to (y4);
                        \draw[edge]       (x7) to (y6);

                    \end{tikzpicture}
                    \caption{$M_6 = \left\{ \{x_3, x_4\}, \{x_5, y_6\},
                    \{x_4, y_2\}, \{x_2, y_3\}, \{x_1, y_5\}, \{x_6, y_1\}
                    \right\}$ \\ (Überdeckende Knoten weiß gefüllt.)}
                    \label{fig:b-6}
                \end{figure}

                \FloatBarrier

                In der 7. Iteration gelingt es nicht, einen augmentierenden Pfad
                zu finden.
                Dabei werden die folgenden Knoten mit alternierenden Pfaden
                erreicht und markiert:
                $x_7, y_1, y_4, y_6, x_6, x_3, x_5, y_2, y_4$

                \paragraph{Ergebnis:}
                Der Algorithmus liefert das Matching $M$ in Abb. \ref{fig:b-6}
                mit $6$ Kanten zusammen mit der minimalen Knotenüberdeckung
                \begin{equation}
                    U = \left\{ x_1, x_2, y_1, y_2, y_4, y_6 \right\} \text{ .}
                \end{equation}

        \end{enumerate}

\end{enumerate}


\end{document}
