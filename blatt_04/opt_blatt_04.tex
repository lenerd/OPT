\documentclass[a4paper]{scrartcl}

% font/encoding packages
\usepackage[utf8]{inputenc}
\usepackage[T1]{fontenc}
\usepackage{lmodern}
\usepackage[ngerman]{babel}

\usepackage{amsmath, amssymb, amsfonts, amsthm, mathtools}
\setcounter{MaxMatrixCols}{13}
\allowdisplaybreaks
\usepackage[output-decimal-marker={,}]{siunitx}
\usepackage{graphicx}
\usepackage{subcaption}
\usepackage{caption}
\usepackage{venndiagram}
\usepackage{enumerate}
\usepackage{polynom}
\polyset{style=C, div=:,vars=x}
\usepackage[section]{placeins}
\usepackage{tikz}
\usetikzlibrary{tikzmark}
\usetikzlibrary{calc}
\usetikzlibrary{arrows}
\usetikzlibrary{arrows.meta}
\usepackage{pgfplots}
\pgfplotsset{compat=1.5.1}
\usepgfplotslibrary{fillbetween}
\usepackage{gauss}
\usepackage{systeme}
\usepackage{cancel}

\tikzset{
    declare function={Floor(\x)=round(\x-0.5);}
}

% Matrizen
\newdimen\asep 
\asep=0.25\baselineskip 

\providecommand{\vo}{ 
    \text{\raisebox{-\asep}[0pt][0pt]{\rule[0pt]{0.8pt}{3.5\asep}}}} 
\providecommand{\vu}{ 
    \text{\raisebox{-0.5\asep}[0pt][0pt]{\rule[0pt]{0.8pt}{\baselineskip}}}} 
\providecommand{\vm}{ 
    \text{\raisebox{-\asep}[0pt][0pt]{\rule[0pt]{0.8pt}{\baselineskip}}}}

\newtheorem*{behaupt}{Behauptung}

\newcommand{\fallfac}[2]{{#1}^{\underline{#2}}}

\newcommand{\risefac}[2]{{#1}^{\overline{#2}}}

\newcommand{\dif}{\mathop{}\!\mathrm{d}}

\newcommand{\linf}[1]{\lim_{#1 \to \infty}}
\newcommand{\gdw}{\Leftrightarrow}

\renewcommand{\arraystretch}{1.3}

\title{Optimierung}
\subtitle{Blatt 4 Hausaufgaben}
\author{
	Lennart Braun (6523742, Gruppe 6) \\
    Alexander Timmermann (6524072, Gruppe 5)
}
\date{zum 10. November 2014}

\begin{document}
\maketitle

\begin{enumerate}
    \item % 1.
        \begin{enumerate}
            \item
                \begin{enumerate}[(i)]
                    \item
                        Starttableau
                        \begin{equation}
                            \begin{array}{rcrrr}
                                x_3 & = & 10 & -x_1 & -x_2 \\
                                x_4 & = & 8 & -x_1 &  \\
                                x_5 & = & 3 &  & -x_2 \\
                                \hline
                                z   & = &   & 2x_1 & +3x_2 \\
                            \end{array}
                        \end{equation}

                        \begin{enumerate}
                            \item Regel vom größten Koeffizienten

                                Eingangsvariable: $x_2$ \\
                                Ausgangsvariable: $x_5$

                                1. Iteration
                                \begin{equation}
                                    \begin{array}{rcrrr}
                                        x_2 & = & 3 &  & -x_5 \\
                                        x_3 & = & 7 & -x_1 & +x_5 \\
                                        x_4 & = & 8 & -x_1 &  \\
                                        \hline
                                        z   & = & 9 & +2x_1 & -3x_5 \\
                                    \end{array}
                                \end{equation}

                                Eingangsvariable: $x_1$ \\
                                Ausgangsvariable: $x_3$

                                2. Iteration
                                \begin{equation}
                                    \begin{array}{rcrrr}
                                        x_1 & = & 7 & +x_5 & -x_3 \\
                                        x_2 & = & 3 & -x_5 &  \\
                                        x_4 & = & 1 & -x_5 & +x_3 \\
                                        \hline
                                        z   & = & 23 & -x_5 & -2x_3 \\
                                    \end{array}
                                \end{equation}

                                Optimale Lösung
                                \begin{align*}
                                    x_1 &= 7 \\
                                    x_2 &= 3 \\
                                    x_3 &= 0 \\
                                    x_4 &= 1 \\
                                    x_5 &= 0 \\
                                    z   &= 23
                                \end{align*}

                            \item Regel vom größten Zuwachs

                                Eingangsvariable: $x_1$ \\
                                Ausgangsvariable: $x_4$

                                1. Iteration
                                \begin{equation}
                                    \begin{array}{rcrrr}
                                        x_1 & = & 8 &  & -x_4 \\
                                        x_3 & = & 2 & -x_2 & +x_4 \\
                                        x_5 & = & 3 & -x_2 &  \\
                                        \hline
                                        z   & = & 16 & +3x_2 & -2x_4 \\
                                    \end{array}
                                \end{equation}

                                Eingangsvariable: $x_2$ \\
                                Ausgangsvariable: $x_3$

                                2. Iteration
                                \begin{equation}
                                    \begin{array}{rcrrr}
                                        x_2 & = & 2 & +x_4 & -x_3 \\
                                        x_1 & = & 8 & -x_4 &  \\
                                        x_5 & = & 1 & -x_4 & -x_3 \\
                                        \hline
                                        z   & = & 22 & +x_4 & -3x_3 \\
                                    \end{array}
                                \end{equation}

                                Eingangsvariable: $x_4$ \\
                                Ausgangsvariable: $x_5$

                                3. Iteration
                                \begin{equation}
                                    \begin{array}{rcrrr}
                                        x_4 & = & 1 & -x_3 & -x_5 \\
                                        x_2 & = & 3 & -2x_3 & -x_5 \\
                                        x_1 & = & 7 & +x_3 & +x_5 \\
                                        \hline
                                        z   & = & 23 & -3x_3 & -x_5 \\
                                    \end{array}
                                \end{equation}

                                Optimale Lösung
                                \begin{align*}
                                    x_1 &= 7 \\
                                    x_2 &= 3 \\
                                    x_3 &= 0 \\
                                    x_4 &= 1 \\
                                    x_5 &= 0 \\
                                    z   &= 23
                                \end{align*}

                        \end{enumerate}

                    \item
                        \begin{center}
                            \begin{tikzpicture}
                                \begin{axis}[
                                    axis lines=middle,
                                    axis equal,
                                    xlabel=$x_1$,
                                    ylabel=$x_2$,
                                    width=1.0\textwidth,
                                    xmin = 0, xmax = 10,
                                    ymin = 0, ymax = 10,
                                    x axis line style={name path=xaxis}
                                ]
                                    \addplot
                                    [name path=a, samples=200, domain=0:10]
                                    {-x+10};
                                    \draw ({axis cs:8,0}|-{rel axis cs:0,0})
                                        -- ({axis cs:8,0}|-{rel axis cs:0,1});
                                    \addplot
                                    [name path=b, samples=200, domain=0:10]
                                    {3};
                                    \addplot
                                    [dashed, name path=z, samples=200, domain=0:10]
                                    {-2/3*x + 23/3};
                                    \node at (axis cs:0.3,0.3) {$P$};
                                    \node at (axis cs:7.7,0.3) {$Q$};
                                    \node at (axis cs:7.7,1.8) {$R$};
                                    \node at (axis cs:6.8,2.7) {$S$};
                                    \node at (axis cs:0.3,2.7) {$T$};
                                    \node at (axis cs:3,5) {$z = 23$};
                                    \addplot[gray!50] fill between[of=b and xaxis, soft clip={domain=0:7+0.01}];
                                    \addplot[gray!50] fill between[of=a and xaxis, soft clip={domain=7-0.01:8}];
                                \end{axis}
                            \end{tikzpicture}
                        \end{center}

                        Durchlaufen der Ecken mit der Regel vom größten
                        \begin{itemize}
                            \item Koeffizienten: $(P, T, S)$
                            \item Zuwachs: $(P, Q, R, S)$
                        \end{itemize}
                        
                \end{enumerate}

            \item
                \begin{enumerate}[(i)]
                    \item

                    \item
                        \begin{center}
                            \begin{tikzpicture}
                                \begin{axis}[
                                    axis lines=middle,
                                    %axis equal,
                                    xlabel=$x_1$,
                                    ylabel=$x_2$,
                                    width=1.0\textwidth,
                                    xmin = 0, xmax = 4,
                                    ymin = 0, ymax = 12,
                                    x axis line style={name path=xaxis}
                                ]
                                    \addplot
                                    [name path=a, samples=200, domain=0:10]
                                    {-5*x+10};
                                    \addplot
                                    [dashed, name path=z, samples=200, domain=0:10]
                                    {-3*x+10};
                                    \node at (axis cs:0.1,0.3) {$P$};
                                    \node at (axis cs:1.8,0.3) {$Q$};
                                    \node at (axis cs:0.1,8.7) {$R$};
                                    \node at (axis cs:2,5) {$z = 10$};
                                    \addplot[gray!50] fill between[of=a and xaxis, soft clip={domain=0:2}];
                                \end{axis}
                            \end{tikzpicture}
                        \end{center}

                        Durchlaufen der Ecken mit der Regel vom größten
                        \begin{itemize}
                            \item Koeffizienten: TODO
                            \item Zuwachs: TODO
                        \end{itemize}

                \end{enumerate}
        \end{enumerate}

    \item % 2.



\end{enumerate}


\end{document}
