\documentclass[a4paper]{scrartcl}

% font/encoding packages
\usepackage[utf8]{inputenc}
\usepackage[T1]{fontenc}
\usepackage{lmodern}
\usepackage[ngerman]{babel}

\usepackage{amsmath, amssymb, amsfonts, amsthm, mathtools}
\setcounter{MaxMatrixCols}{13}
\allowdisplaybreaks
\usepackage[output-decimal-marker={,}]{siunitx}
\usepackage{graphicx}
\usepackage{subcaption}
\usepackage{caption}
\usepackage{venndiagram}
\usepackage{enumerate}
\usepackage{polynom}
\polyset{style=C, div=:,vars=x}
\usepackage[section]{placeins}
\usepackage{tikz}
\usetikzlibrary{tikzmark}
\usetikzlibrary{calc}
\usetikzlibrary{arrows}
\usetikzlibrary{arrows.meta}
\usepackage{pgfplots}
\pgfplotsset{compat=1.5.1}
\usepgfplotslibrary{fillbetween}
\usepackage{gauss}
\usepackage{systeme}
\usepackage{cancel}

\tikzset{
    declare function={Floor(\x)=round(\x-0.5);}
}

% Matrizen
\newdimen\asep 
\asep=0.25\baselineskip 

\providecommand{\vo}{ 
    \text{\raisebox{-\asep}[0pt][0pt]{\rule[0pt]{0.8pt}{3.5\asep}}}} 
\providecommand{\vu}{ 
    \text{\raisebox{-0.5\asep}[0pt][0pt]{\rule[0pt]{0.8pt}{\baselineskip}}}} 
\providecommand{\vm}{ 
    \text{\raisebox{-\asep}[0pt][0pt]{\rule[0pt]{0.8pt}{\baselineskip}}}}

\newtheorem*{behaupt}{Behauptung}

\newcommand{\fallfac}[2]{{#1}^{\underline{#2}}}

\newcommand{\risefac}[2]{{#1}^{\overline{#2}}}

\newcommand{\dif}{\mathop{}\!\mathrm{d}}

\newcommand{\linf}[1]{\lim_{#1 \to \infty}}
\newcommand{\gdw}{\Leftrightarrow}

\renewcommand{\arraystretch}{1.3}

\title{Optimierung}
\subtitle{Blatt 4 Hausaufgaben}
\author{
	Lennart Braun (6523742, Gruppe 6) \\
    Alexander Timmermann (6524072, Gruppe 5)
}
\date{zum 10. November 2014}

\begin{document}
\maketitle

\begin{enumerate}
    \item % 1.
        \begin{enumerate}
            \item
                Starttableau
                \begin{equation}
                    \begin{array}{rcrrr}
                        x_3 & = & 10 & -x_1 & -x_2 \\
                        x_4 & = & 8 & -x_1 &  \\
                        x_5 & = & 3 &  & -x_2 \\
                        \hline
                        z   & = &   & 2x_1 & +3x_2 \\
                    \end{array}
                \end{equation}

                \begin{enumerate}[(i)]
                    \item Regel vom größten Koeffizienten

                        Eingangsvariable: $x_2$ \\
                        Ausgangsvariable: $x_5$

                        1. Iteration
                        \begin{equation}
                            \begin{array}{rcrrr}
                                x_2 & = & 3 &  & -x_5 \\
                                x_3 & = & 7 & -x_1 & +x_5 \\
                                x_4 & = & 8 & -x_1 &  \\
                                \hline
                                z   & = & 9 & +2x_1 & -3x_5 \\
                            \end{array}
                        \end{equation}

                        Eingangsvariable: $x_1$ \\
                        Ausgangsvariable: $x_3$

                        2. Iteration
                        \begin{equation}
                            \begin{array}{rcrrr}
                                x_1 & = & 7 & +x_5 & -x_3 \\
                                x_2 & = & 3 & -x_5 &  \\
                                x_4 & = & 1 & -x_5 & +x_3 \\
                                \hline
                                z   & = & 23 & -x_5 & -2x_3 \\
                            \end{array}
                        \end{equation}

                        Optimale Lösung
                        \begin{align*}
                            x_1 &= 7 \\
                            x_2 &= 3 \\
                            x_3 &= 0 \\
                            x_4 &= 1 \\
                            x_5 &= 0 \\
                            Z   &= 23
                        \end{align*}

                    \item Regel vom größten Zuwachs

                \end{enumerate}

            \item
                \begin{enumerate}[(i)]
                    \item

                    \item

                \end{enumerate}
        \end{enumerate}

    \item % 2.



\end{enumerate}


\end{document}
