\documentclass[a4paper]{scrartcl}

% font/encoding packages
\usepackage[utf8]{inputenc}
\usepackage[T1]{fontenc}
\usepackage{lmodern}
\usepackage[ngerman]{babel}

\usepackage[top=1.1in, bottom=1.5in]{geometry}

\usepackage{amsmath, amssymb, amsfonts, amsthm, mathtools}
\usepackage{stmaryrd}
\usepackage{marvosym}
\setcounter{MaxMatrixCols}{13}
\allowdisplaybreaks
\usepackage[output-decimal-marker={,}]{siunitx}
\usepackage{graphicx}
\usepackage{subcaption}
\usepackage{caption}
\usepackage{venndiagram}
\usepackage[shortlabels]{enumitem}
\usepackage{polynom}
\polyset{style=C, div=:,vars=x}
\usepackage[section]{placeins}
\usepackage{tikz}
\usetikzlibrary{tikzmark}
\usetikzlibrary{calc}
\usetikzlibrary{arrows}
\usetikzlibrary{arrows.meta}
\usepackage{pgfplots}
\pgfplotsset{compat=1.5.1}
\usepgfplotslibrary{fillbetween}
\usepackage{gauss}
\usepackage{systeme}
\usepackage{cancel}

\tikzset{
    declare function={Floor(\x)=round(\x-0.5);}
}

% Matrizen
\newdimen\asep 
\asep=0.25\baselineskip 

\providecommand{\vo}{ 
    \text{\raisebox{-\asep}[0pt][0pt]{\rule[0pt]{0.8pt}{3.5\asep}}}} 
\providecommand{\vu}{ 
    \text{\raisebox{-0.5\asep}[0pt][0pt]{\rule[0pt]{0.8pt}{\baselineskip}}}} 
\providecommand{\vm}{ 
    \text{\raisebox{-\asep}[0pt][0pt]{\rule[0pt]{0.8pt}{\baselineskip}}}}

\newtheorem*{behaupt}{Behauptung}

\newcommand{\fallfac}[2]{{#1}^{\underline{#2}}}

\newcommand{\risefac}[2]{{#1}^{\overline{#2}}}

\newcommand{\dif}{\mathop{}\!\mathrm{d}}

\newcommand{\linf}[1]{\lim_{#1 \to \infty}}
\newcommand{\gdw}{\Leftrightarrow}

\DeclareSIUnit\calorie{cal}

\renewcommand{\arraystretch}{1.3}

\title{Optimierung}
\subtitle{Blatt 10 Hausaufgaben}
\author{
	Lennart Braun (6523742, Gruppe 6) \\
    Alexander Timmermann (6524072, Gruppe 5)
}
\date{zum 5. Januar 2015}

\begin{document}
\maketitle

\begin{enumerate}[label=\bfseries\arabic*.]
    \item
        Algorithmus von Edmonds und Karp

        \begin{figure}[h]
            \centering
            \begin{tikzpicture}[
                auto,
                scale=4,
            ]
                \tikzstyle{vertex}=[
                    circle,
                    fill,
                    minimum size=0.1cm,
                    align=center,
                ]
                \tikzstyle{edge}=[
                    ->,
                    >=stealth',
                ]
                \tikzstyle{augm}=[
                    ultra thick,
                ]

                \node[vertex] (s) at (0,1) {};
                \node[vertex] (a) at (1,2) {};
                \node[vertex] (b) at (1,1) {};
                \node[vertex] (c) at (1,0) {};
                \node[vertex] (d) at (2,0) {};
                \node[vertex] (e) at (2,1) {};
                \node[vertex] (f) at (2,1.5) {};
                \node[vertex] (t) at (3,1) {};

                \node[align=center, anchor=east]  at (s.west)  {$s$\\$(-,\infty)$};
                \node[align=center, anchor=south] at (a.north) {$(s,+,5)$\\$a$};
                \node[align=center, anchor=north] at (b.south) {$b$\\$(s,+,5)$};
                \node[align=center, anchor=north] at (c.south) {$c$\\$(s,+,6)$};
                \node[align=center, anchor=north] at (d.south) {$d$\\$(b,+,5)$};
                \node[align=center, anchor=north] at (e.south) {$e$\\$(b,+,3)$};
                \node[align=center, anchor=south] at (f.north) {$(a,+,3)$\\$f$};
                \node[align=center, anchor=west]  at (t.east)  {$t$\\$(f,+,3)$};

                \draw[edge, augm] (s) to node {$0(5)$} (a);
                \draw[edge]       (s) to node {$0(5)$} (b);
                \draw[edge]       (s) to node {$0(6)$} (c);
                \draw[edge]       (a) to node {$0(2)$} (b);
                \draw[edge, augm] (a) to node {$0(3)$} (f);
                \draw[edge]       (b) to node {$0(6)$} (d);
                \draw[edge]       (b) to node {$0(3)$} (e);
                \draw[edge]       (b) to node {$0(2)$} (f);
                \draw[edge]       (c) to node {$0(4)$} (d);
                \draw[edge]       (d) to node {$0(5)$} (t);
                \draw[edge]       (e) to node {$0(4)$} (t);
                \draw[edge, augm] (f) to node {$0(4)$} (t);

            \end{tikzpicture}
            \caption{$w(f_0) = 0$}
            \label{fig:ek0}
        \end{figure}

        \begin{figure}[h]
            \centering
            \begin{tikzpicture}[
                auto,
                scale=4,
            ]
                \tikzstyle{vertex}=[
                    circle,
                    fill,
                    minimum size=0.1cm,
                    align=center,
                ]
                \tikzstyle{edge}=[
                    ->,
                    >=stealth',
                ]
                \tikzstyle{augm}=[
                    ultra thick,
                ]

                \node[vertex] (s) at (0,1) {};
                \node[vertex] (a) at (1,2) {};
                \node[vertex] (b) at (1,1) {};
                \node[vertex] (c) at (1,0) {};
                \node[vertex] (d) at (2,0) {};
                \node[vertex] (e) at (2,1) {};
                \node[vertex] (f) at (2,1.5) {};
                \node[vertex] (t) at (3,1) {};

                \node[align=center, anchor=east]  at (s.west)  {$s$\\$(-,\infty)$};
                \node[align=center, anchor=south] at (a.north) {$(s,+,2)$\\$a$};
                \node[align=center, anchor=north] at (b.south) {$b$\\$(s,+,5)$};
                \node[align=center, anchor=north] at (c.south) {$c$\\$(s,+,6)$};
                \node[align=center, anchor=north] at (d.south) {$d$\\$(b,+,5)$};
                \node[align=center, anchor=north] at (e.south) {$e$\\$(b,+,3)$};
                \node[align=center, anchor=south] at (f.north) {$(b,+,2)$\\$f$};
                \node[align=center, anchor=west]  at (t.east)  {$t$\\$(d,+,5)$};

                \draw[edge]       (s) to node {$3(5)$} (a);
                \draw[edge, augm] (s) to node {$0(5)$} (b);
                \draw[edge]       (s) to node {$0(6)$} (c);
                \draw[edge]       (a) to node {$0(2)$} (b);
                \draw[edge]       (a) to node {$3(3)$} (f);
                \draw[edge, augm] (b) to node {$0(6)$} (d);
                \draw[edge]       (b) to node {$0(3)$} (e);
                \draw[edge]       (b) to node {$0(2)$} (f);
                \draw[edge]       (c) to node {$0(4)$} (d);
                \draw[edge, augm] (d) to node {$0(5)$} (t);
                \draw[edge]       (e) to node {$0(4)$} (t);
                \draw[edge]       (f) to node {$3(4)$} (t);

            \end{tikzpicture}
            \caption{$w(f_1) = 3$}
            \label{fig:ek1}
        \end{figure}

        \begin{figure}[h]
            \centering
            \begin{tikzpicture}[
                auto,
                scale=4,
            ]
                \tikzstyle{vertex}=[
                    circle,
                    fill,
                    minimum size=0.1cm,
                    align=center,
                ]
                \tikzstyle{edge}=[
                    ->,
                    >=stealth',
                ]
                \tikzstyle{augm}=[
                    ultra thick,
                ]

                \node[vertex] (s) at (0,1) {};
                \node[vertex] (a) at (1,2) {};
                \node[vertex] (b) at (1,1) {};
                \node[vertex] (c) at (1,0) {};
                \node[vertex] (d) at (2,0) {};
                \node[vertex] (e) at (2,1) {};
                \node[vertex] (f) at (2,1.5) {};
                \node[vertex] (t) at (3,1) {};

                \node[align=center, anchor=east]  at (s.west)  {$s$\\$(-,\infty)$};
                \node[align=center, anchor=south] at (a.north) {$(s,+,2)$\\$a$};
                \node[align=center, anchor=north] at (b.south) {$b$\\$(a,+,2)$};
                \node[align=center, anchor=north] at (c.south) {$c$\\$(s,+,6)$};
                \node[align=center, anchor=north] at (d.south) {$d$\\$(c,+,4)$};
                \node[align=center, anchor=north] at (e.south) {$e$\\$(b,+,2)$};
                \node[align=center, anchor=south] at (f.north) {$(b,+,2)$\\$f$};
                \node[align=center, anchor=west]  at (t.east)  {$t$\\$(e,+,2)$};

                \draw[edge, augm] (s) to node {$3(5)$} (a);
                \draw[edge]       (s) to node {$5(5)$} (b);
                \draw[edge]       (s) to node {$0(6)$} (c);
                \draw[edge, augm] (a) to node {$0(2)$} (b);
                \draw[edge]       (a) to node {$3(3)$} (f);
                \draw[edge]       (b) to node {$5(6)$} (d);
                \draw[edge, augm] (b) to node {$0(3)$} (e);
                \draw[edge]       (b) to node {$0(2)$} (f);
                \draw[edge]       (c) to node {$0(4)$} (d);
                \draw[edge]       (d) to node {$5(5)$} (t);
                \draw[edge, augm] (e) to node {$0(4)$} (t);
                \draw[edge]       (f) to node {$3(4)$} (t);

            \end{tikzpicture}
            \caption{$w(f_2) = 8$}
            \label{fig:ek2}
        \end{figure}

        \begin{figure}[h]
            \centering
            \begin{tikzpicture}[
                auto,
                scale=4,
            ]
                \tikzstyle{vertex}=[
                    circle,
                    fill,
                    minimum size=0.1cm,
                    align=center,
                ]
                \tikzstyle{edge}=[
                    ->,
                    >=stealth',
                ]
                \tikzstyle{augm}=[
                    ultra thick,
                ]

                \node[vertex] (s) at (0,1) {};
                \node[vertex] (a) at (1,2) {};
                \node[vertex] (b) at (1,1) {};
                \node[vertex] (c) at (1,0) {};
                \node[vertex] (d) at (2,0) {};
                \node[vertex] (e) at (2,1) {};
                \node[vertex] (f) at (2,1.5) {};
                \node[vertex] (t) at (3,1) {};

                \node[align=center, anchor=east]  at (s.west)  {$s$\\$(-,\infty)$};
                \node[align=center, anchor=south] at (a.north) {$(b,-,2)$\\$a$};
                \node[align=center, anchor=north] at (b.south) {$b$\\$(d,-,4)$};
                \node[align=center, anchor=north] at (c.south) {$c$\\$(s,+,6)$};
                \node[align=center, anchor=north] at (d.south) {$d$\\$(c,+,4)$};
                \node[align=center, anchor=north] at (e.south) {$e$\\$(b,+,1)$};
                \node[align=center, anchor=south] at (f.north) {$(b,+,2)$\\$f$};
                \node[align=center, anchor=west]  at (t.east)  {$t$\\$(e,+,1)$};

                \draw[edge]       (s) to node {$5(5)$} (a);
                \draw[edge]       (s) to node {$5(5)$} (b);
                \draw[edge, augm] (s) to node {$0(6)$} (c);
                \draw[edge]       (a) to node {$2(2)$} (b);
                \draw[edge]       (a) to node {$3(3)$} (f);
                \draw[edge, augm] (b) to node {$5(6)$} (d);
                \draw[edge, augm] (b) to node {$2(3)$} (e);
                \draw[edge]       (b) to node {$0(2)$} (f);
                \draw[edge, augm] (c) to node {$0(4)$} (d);
                \draw[edge]       (d) to node {$5(5)$} (t);
                \draw[edge, augm] (e) to node {$2(4)$} (t);
                \draw[edge]       (f) to node {$3(4)$} (t);

            \end{tikzpicture}
            \caption{$w(f_3) = 10$}
            \label{fig:ek3}
        \end{figure}

        \begin{figure}[h]
            \centering
            \begin{tikzpicture}[
                auto,
                scale=4,
            ]
                \tikzstyle{vertex}=[
                    circle,
                    fill,
                    minimum size=0.1cm,
                    align=center,
                ]
                \tikzstyle{edge}=[
                    ->,
                    >=stealth',
                ]
                \tikzstyle{augm}=[
                    ultra thick,
                ]

                \node[vertex] (s) at (0,1) {};
                \node[vertex] (a) at (1,2) {};
                \node[vertex] (b) at (1,1) {};
                \node[vertex] (c) at (1,0) {};
                \node[vertex] (d) at (2,0) {};
                \node[vertex] (e) at (2,1) {};
                \node[vertex] (f) at (2,1.5) {};
                \node[vertex] (t) at (3,1) {};

                \node[align=center, anchor=east]  at (s.west)  {$s$\\$(-,\infty)$};
                \node[align=center, anchor=south] at (a.north) {$(b,-,2)$\\$a$};
                \node[align=center, anchor=north] at (b.south) {$b$\\$(d,-,3)$};
                \node[align=center, anchor=north] at (c.south) {$c$\\$(s,+,5)$};
                \node[align=center, anchor=north] at (d.south) {$d$\\$(c,+,3)$};
                \node[align=center, anchor=north] at (e.south) {$e$};
                \node[align=center, anchor=south] at (f.north) {$(b,+,2)$\\$f$};
                \node[align=center, anchor=west]  at (t.east)  {$t$\\$(f,+,1)$};

                \draw[edge]       (s) to node {$5(5)$} (a);
                \draw[edge]       (s) to node {$5(5)$} (b);
                \draw[edge, augm] (s) to node {$1(6)$} (c);
                \draw[edge]       (a) to node {$2(2)$} (b);
                \draw[edge]       (a) to node {$3(3)$} (f);
                \draw[edge, augm] (b) to node {$4(6)$} (d);
                \draw[edge]       (b) to node {$3(3)$} (e);
                \draw[edge, augm] (b) to node {$0(2)$} (f);
                \draw[edge, augm] (c) to node {$1(4)$} (d);
                \draw[edge]       (d) to node {$5(5)$} (t);
                \draw[edge]       (e) to node {$3(4)$} (t);
                \draw[edge, augm] (f) to node {$3(4)$} (t);

            \end{tikzpicture}
            \caption{$w(f_4) = 11$}
            \label{fig:ek4}
        \end{figure}

        \begin{figure}[h]
            \centering
            \begin{tikzpicture}[
                auto,
                scale=4,
            ]
                \tikzstyle{vertex}=[
                    circle,
                    fill,
                    minimum size=0.1cm,
                    align=center,
                ]
                \tikzstyle{edge}=[
                    ->,
                    >=stealth',
                ]
                \tikzstyle{augm}=[
                    ultra thick,
                ]

                \node[vertex] (s) at (0,1) {};
                \node[vertex] (a) at (1,2) {};
                \node[vertex] (b) at (1,1) {};
                \node[vertex] (c) at (1,0) {};
                \node[vertex] (d) at (2,0) {};
                \node[vertex] (e) at (2,1) {};
                \node[vertex] (f) at (2,1.5) {};
                \node[vertex] (t) at (3,1) {};

                \node[align=center, anchor=east]  at (s.west)  {$s$\\$(-,\infty)$};
                \node[align=center, anchor=south] at (a.north) {$(b,-,2)$\\$a$};
                \node[align=center, anchor=north] at (b.south) {$b$\\$(d,-,2)$};
                \node[align=center, anchor=north] at (c.south) {$c$\\$(s,+,4)$};
                \node[align=center, anchor=north] at (d.south) {$d$\\$(c,+,2)$};
                \node[align=center, anchor=north] at (e.south) {$e$};
                \node[align=center, anchor=south] at (f.north) {$(b,+,1)$\\$f$};
                \node[align=center, anchor=west]  at (t.east)  {$t$};

                \draw[edge]       (s) to node {$5(5)$} (a);
                \draw[edge]       (s) to node {$5(5)$} (b);
                \draw[edge]       (s) to node {$2(6)$} (c);
                \draw[edge]       (a) to node {$2(2)$} (b);
                \draw[edge]       (a) to node {$3(3)$} (f);
                \draw[edge]       (b) to node {$3(6)$} (d);
                \draw[edge]       (b) to node {$3(3)$} (e);
                \draw[edge]       (b) to node {$1(2)$} (f);
                \draw[edge]       (c) to node {$2(4)$} (d);
                \draw[edge]       (d) to node {$5(5)$} (t);
                \draw[edge]       (e) to node {$3(4)$} (t);
                \draw[edge]       (f) to node {$4(4)$} (t);

                \draw [ultra thick] (2.5,1.50) to [out=195, in=90] (1.75,1)
                                               to [out=270, in=165] (2.5,0.25);

            \end{tikzpicture}
            \caption{$w(f_5) = 12 = c(S_f, T_f)$}
            \label{fig:ek5}
        \end{figure}

        \FloatBarrier

    \item
        \begin{enumerate}
            \item
                \begin{align}
                    \begin{split}
                        f \colon& \mathbb{R}^2 \to \mathbb{R}^2 \\
                        & x \mapsto Ax
                    \end{split} \\
                    A &=
                    \begin{pmatrix}
                        5 & -1 \\
                        7 &  3 \\
                    \end{pmatrix}
                \end{align}
                \begin{align}
                    \begin{split}
                        f(e_1) &= Ae_1 =
                        \begin{pmatrix}
                            5 & -1 \\
                            7 &  3 \\
                        \end{pmatrix}
                        \begin{pmatrix}
                            1 \\ 0
                        \end{pmatrix}
                        =
                        \begin{pmatrix}
                            5 \\ 7
                        \end{pmatrix}
                        = v_1
                    \end{split} \\
                    \begin{split}
                        f(e_2) &= Ae_2 =
                        \begin{pmatrix}
                            5 & -1 \\
                            7 &  3 \\
                        \end{pmatrix}
                        \begin{pmatrix}
                            0 \\ 1
                        \end{pmatrix}
                        =
                        \begin{pmatrix}
                            -1 \\ 3
                        \end{pmatrix}
                        = v_2
                    \end{split}
                \end{align}
                
                

            \item

            \item

        \end{enumerate}

    \item
        \begin{enumerate}
            \item
                Die Eingabe des Problems in der Form
                \begin{verbatim}
Minimize p = 67W + 120K + 100H + 90F + 97B + 124N + 98S + 62M \
subject to
8W + 25K + 30H + 22F + 3B + 8N + 6S >= 75
1W + 35K + 8H + 1F + 33N + 13S + 98M >= 90
54W + 42B + 4N + 63S >= 300
S <= 80 \end{verbatim}
                ergab das folgende Ergebnis:
                \begin{verbatim}
Optimal Solution: p = 522.343; w = 5.55556, k = 0, h = 1.01852, \
f = 0, b = 0, n = 0, s = 0, m = 0.778534 \end{verbatim}
                Es ist also optimal, pro Person für $522.343$ Öre
                \SI{555,556}{\gram} Weißbrot, \SI{101,652}{\gram} Hähnchen
                und \SI{77,8534}{\gram} Margarine einzukaufen.

            \item
                Die Eingabe des Problems in der Form
                \begin{verbatim}
Minimize p = 21T + 16K + 371S + 346M + 884O subject to
0.85T + 1.62K + 12.78S + 8.39M >= 15
0.33T + 0.2K + 1.58S + 1.39M + 100O >= 2
0.33T + 0.2K + 1.58S + 1.39M + 100O <= 6
4.64T + 2.37K + 74.69S + 80.7M >= 4
9T + 8K + 7S + 508.2M <= 100
- T + K + S - M - O <= 0 \end{verbatim}
                ergab das folgende Ergebnis:
                \begin{verbatim}
Optimal Solution: p = 232.515; t = 5.8848, k = 5.84318, \
s = 0.0416255, m = 0, o = 0 \end{verbatim}
                Es ist also optimal, einen Salat aus \SI{588,48}{\gram} Tomaten,
                \SI{584,318}{\gram} Kopfsalat und \SI{4,16255}{\gram} Spinat
                zusammenzustellen.
                Der Salat enthält \SI{232,515}{\kilo\calorie}.

            \item
                Die Eingabe des Problems in der Form
                \begin{verbatim}
Minimize p = 3x1 + 24x2 + 13x3 + 9x4 + 20x5 + 19x6 subject to
110x1 + 205x2 + 160x3 + 160x4 + 420x5 + 260x6 >= 2000
4x1 + 32x2 + 13x3 + 8x4 + 4x5 + 14x6 >= 55
2x1 + 12x2 + 54x3 + 285x4 + 22x5 + 80x6 >= 800
x1 <= 4
x2 <= 3
x3 <= 2
x4 <= 8
x5 <= 2
x6 <= 2 \end{verbatim}
                ergab das folgende Ergebnis:
                \begin{verbatim}
Optimal Solution: p = 92.5; x1 = 4, x2 = 0, x3 = 0, x4 = 4.5, \
x5 = 2, x6 = 0 \end{verbatim}
                Es ist also optimal, wenn Paul $4$ Portionen Haferflocken,
                $4.5$ Portionen Milch, 2 Portionen Kirschkuchen und 2 Portionen
                Bohnen zu sich nimmt.
                Paul bezahlt $92.5$ Cent pro Tag.

        \end{enumerate}

\end{enumerate}


\end{document}
