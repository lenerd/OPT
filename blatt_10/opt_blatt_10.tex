\documentclass[a4paper]{scrartcl}

% font/encoding packages
\usepackage[utf8]{inputenc}
\usepackage[T1]{fontenc}
\usepackage{lmodern}
\usepackage[ngerman]{babel}

% \usepackage[top=2.5cm,bottom=3cm,left=1.5cm,right=1.5cm]{geometry}

\usepackage{amsmath, amssymb, amsfonts, amsthm, mathtools}
\usepackage{stmaryrd}
\usepackage{marvosym}
\setcounter{MaxMatrixCols}{13}
\allowdisplaybreaks
\usepackage[output-decimal-marker={,}]{siunitx}
\usepackage{graphicx}
\usepackage{subcaption}
\usepackage{caption}
\usepackage{venndiagram}
\usepackage[shortlabels]{enumitem}
\usepackage{polynom}
\polyset{style=C, div=:,vars=x}
\usepackage[section]{placeins}
\usepackage{tikz}
\usetikzlibrary{tikzmark}
\usetikzlibrary{calc}
\usetikzlibrary{arrows}
\usetikzlibrary{arrows.meta}
\usepackage{pgfplots}
\pgfplotsset{compat=1.5.1}
\usepgfplotslibrary{fillbetween}
\usepackage{gauss}
\usepackage{systeme}
\usepackage{cancel}

\tikzset{
    declare function={Floor(\x)=round(\x-0.5);}
}

% Matrizen
\newdimen\asep 
\asep=0.25\baselineskip 

\providecommand{\vo}{ 
    \text{\raisebox{-\asep}[0pt][0pt]{\rule[0pt]{0.8pt}{3.5\asep}}}} 
\providecommand{\vu}{ 
    \text{\raisebox{-0.5\asep}[0pt][0pt]{\rule[0pt]{0.8pt}{\baselineskip}}}} 
\providecommand{\vm}{ 
    \text{\raisebox{-\asep}[0pt][0pt]{\rule[0pt]{0.8pt}{\baselineskip}}}}

\newtheorem*{behaupt}{Behauptung}

\newcommand{\fallfac}[2]{{#1}^{\underline{#2}}}

\newcommand{\risefac}[2]{{#1}^{\overline{#2}}}

\newcommand{\dif}{\mathop{}\!\mathrm{d}}

\newcommand{\linf}[1]{\lim_{#1 \to \infty}}
\newcommand{\gdw}{\Leftrightarrow}

\DeclareSIUnit\calorie{cal}

\renewcommand{\arraystretch}{1.3}

\title{Optimierung}
\subtitle{Blatt 10 Hausaufgaben}
\author{
	Lennart Braun (6523742, Gruppe 6) \\
    Alexander Timmermann (6524072, Gruppe 5)
}
\date{zum 5. Januar 2015}

\begin{document}
\maketitle

\begin{enumerate}[label=\bfseries\arabic*.]
    \item

    \item
        \begin{enumerate}
            \item

            \item

            \item

        \end{enumerate}

    \item
        \begin{enumerate}
            \item
                Die Eingabe des Problems in der Form
                \begin{verbatim}
Minimize p = 67W + 120K + 100H + 90F + 97B + 124N + 98S + 62M \
subject to
8W + 25K + 30H + 22F + 3B + 8N + 6S >= 75
1W + 35K + 8H + 1F + 33N + 13S + 98M >= 90
54W + 42B + 4N + 63S >= 300
S <= 80 \end{verbatim}
                ergab das folgende Ergebnis:
                \begin{verbatim}
Optimal Solution: p = 522.343; w = 5.55556, k = 0, h = 1.01852, \
f = 0, b = 0, n = 0, s = 0, m = 0.778534 \end{verbatim}
                Es ist also optimal, pro Person für $522.343$ Öre
                \SI{555,556}{\gram} Weißbrot, \SI{101,652}{\gram} Hähnchen
                und \SI{77,8534}{\gram} Margarine einzukaufen.

            \item
                Die Eingabe des Problems in der Form
                \begin{verbatim}
Minimize p = 21T + 16K + 371S + 346M + 884O subject to
0.85T + 1.62K + 12.78S + 8.39M >= 15
0.33T + 0.2K + 1.58S + 1.39M + 100O >= 2
0.33T + 0.2K + 1.58S + 1.39M + 100O <= 6
4.64T + 2.37K + 74.69S + 80.7M >= 4
9T + 8K + 7S + 508.2M <= 100
- T + K + S - M - O <= 0 \end{verbatim}
                ergab das folgende Ergebnis:
                \begin{verbatim}
Optimal Solution: p = 232.515; t = 5.8848, k = 5.84318, \
s = 0.0416255, m = 0, o = 0 \end{verbatim}
                Es ist also optimal, einen Salat aus \SI{588,48}{\gram} Tomaten,
                \SI{584,318}{\gram} Kopfsalat und \SI{4,16255}{\gram} Spinat
                zusammenzustellen.
                Der Salat enthält \SI{232,515}{\kilo\calorie}.

            \item

        \end{enumerate}

\end{enumerate}


\end{document}
