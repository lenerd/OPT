\documentclass[a4paper]{scrartcl}

% font/encoding packages
\usepackage[utf8]{inputenc}
\usepackage[T1]{fontenc}
\usepackage{lmodern}
\usepackage[ngerman]{babel}

\usepackage[top=1.1in, bottom=1.5in]{geometry}

\usepackage{amsmath, amssymb, amsfonts, amsthm, mathtools}
\usepackage{stmaryrd}
\usepackage{marvosym}
\setcounter{MaxMatrixCols}{13}
\allowdisplaybreaks
\usepackage[output-decimal-marker={,}]{siunitx}
\usepackage{graphicx}
\usepackage{subcaption}
\usepackage{caption}
\usepackage{venndiagram}
\usepackage[shortlabels]{enumitem}
\usepackage{polynom}
\polyset{style=C, div=:,vars=x}
\usepackage[section]{placeins}
\usepackage{tikz}
\usetikzlibrary{tikzmark}
\usetikzlibrary{calc}
\usetikzlibrary{arrows}
\usetikzlibrary{arrows.meta}
\usetikzlibrary{decorations}
\usepackage{pgfplots}
\pgfplotsset{compat=1.5.1}
\usepgfplotslibrary{fillbetween}
\usepackage{gauss}
\usepackage{systeme}
\usepackage{cancel}

\tikzset{
    declare function={Floor(\x)=round(\x-0.5);}
}

% Matrizen
\newdimen\asep 
\asep=0.25\baselineskip 

\providecommand{\vo}{ 
    \text{\raisebox{-\asep}[0pt][0pt]{\rule[0pt]{0.8pt}{3.5\asep}}}} 
\providecommand{\vu}{ 
    \text{\raisebox{-0.5\asep}[0pt][0pt]{\rule[0pt]{0.8pt}{\baselineskip}}}} 
\providecommand{\vm}{ 
    \text{\raisebox{-\asep}[0pt][0pt]{\rule[0pt]{0.8pt}{\baselineskip}}}}

\newtheorem*{behaupt}{Behauptung}

\newcommand{\fallfac}[2]{{#1}^{\underline{#2}}}

\newcommand{\risefac}[2]{{#1}^{\overline{#2}}}

\newcommand{\dif}{\mathop{}\!\mathrm{d}}

\newcommand{\linf}[1]{\lim_{#1 \to \infty}}
\newcommand{\gdw}{\Leftrightarrow}

\DeclareSIUnit\calorie{cal}

\renewcommand{\arraystretch}{1.3}

\title{Optimierung}
\subtitle{Blatt 13 Hausaufgaben}
\author{
	Lennart Braun (6523742, Gruppe 6) \\
    Alexander Timmermann (6524072, Gruppe 5)
}
\date{zum 26. Januar 2015}

\begin{document}
\maketitle

\begin{enumerate}[label=\bfseries\arabic*.]
    \item
        Es werden die Gegenstände 3, 4 und 5 mitgenommen.
        Zusammen haben sie ein Gewicht von 19 und einen Wert von 13.
        \begin{table}[h]
            \centering
            \begin{tabular}{r|r|r|r|r|r|r|r|r|r|r|r|r|r|r|r|r|r|r|r|r|}
                \hline
                7 & 0 & 0 & 0 & 2 & 2 & 4 & 4 & 5 & 6 & 6 &  7 &  8 &  9 & 10 & 10 & 11 & 12 & 12 & 13 & 13 \\ \hline
                6 & 0 & 0 & 0 & 0 & 2 & 4 & 4 & 5 & 6 & 6 &  6 &  7 &  9 & 10 & 10 & 11 & 11 & 12 & 12 & 13 \\ \hline
                5 & 0 & 0 & 0 & 0 & 2 & 2 & 3 & 5 & 6 & 6 &  6 &  7 &  8 &  8 &  9 & 11 & 11 & 11 & 11 & \underline{13} \\ \hline
                4 & 0 & 0 & 0 & 0 & 2 & 2 & 3 & 5 & 5 & 5 &  5 &  \underline{7} &  7 &  8 &  8 &  8 &  8 & 10 & 11 & 11 \\ \hline
                3 & 0 & 0 & 0 & 0 & \underline{2} & 2 & 3 & 3 & 3 & 3 &  5 &  6 &  6 &  6 &  6 &  8 &  8 &  9 &  9 &  9 \\ \hline
                2 & 0 & 0 & 0 & 0 & 0 & 0 & 3 & 3 & 3 & 3 &  3 &  6 &  6 &  6 &  6 &  6 &  6 &  9 &  9 &  9 \\ \hline
                1 & 0 & 0 & 0 & 0 & 0 & 0 & 3 & 3 & 3 & 3 &  3 &  3 &  3 &  3 &  3 &  3 &  3 &  3 &  3 &  3 \\ \hline
                0 & 0 & 0 & 0 & 0 & 0 & 0 & 0 & 0 & 0 & 0 &  0 &  0 &  0 &  0 &  0 &  0 &  0 &  0 &  0 &  0 \\ \hline
                  & 0 & 1 & 2 & 3 & 4 & 5 & 6 & 7 & 8 & 9 & 10 & 11 & 12 & 13 & 14 & 15 & 16 & 17 & 18 & 19 \\
            \end{tabular}
            \caption{Tabelle zur Lösung des Rucksackproblems}
            \label{tab:rucksack}
        \end{table}

    \item
        \begin{enumerate}
            \item
                In der sechsten Iteration ändern sich die Werte nicht mehr,
                es existieren also keine negativen Kreise in dem Graphen.
                \begin{table}[h]
                    \centering
                    \begin{tabular}{r|c|c|c|c|c|c|}
                          & $s$ & $a$ & $c$ & $c$ & $d$ & $e$ \\ \hline
                        0 & $0,-$ & $\infty,-$ & $\infty,-$ & $\infty,-$ & $\infty,-$ & $\infty,-$ \\
                        1 & $0,-$ & $7,s$ & $2,s$ & $\infty,-$ & $\infty,-$ & $\infty,-$ \\
                        2 & $0,-$ & $6,b$ & $2,s$ & $0,b$ & $3,a$ & $\infty,-$ \\
                        3 & $0,-$ & $6,b$ & $2,s$ & $0,b$ & $1,c$ & $4,d$ \\
                        4 & $0,-$ & $6,b$ & $2,s$ & $0,b$ & $1,c$ & $2,d$ \\
                        5 & $0,-$ & $6,b$ & $2,s$ & $0,b$ & $1,c$ & $2,d$ \\
                        6 & $0,-$ & $6,b$ & $2,s$ & $0,b$ & $1,c$ & $2,d$ \\
                    \end{tabular}
                    \caption{Tabelle zum Algorithmus von Bellman und Ford}
                    \label{tab:bf1}
                \end{table}

            \item
                In der sechsten Iteration ändern sich die Werte noch,
                es existiert also mindestens ein negativer Kreis in dem Graphen.
                \begin{table}[h]
                    \centering
                    \begin{tabular}{r|c|c|c|c|c|c|}
                          & $s$ & $a$ & $c$ & $c$ & $d$ & $e$ \\ \hline
                        0 & $0,-$ & $\infty,-$ & $\infty,-$ & $\infty,-$ & $\infty,-$ & $\infty,-$ \\
                        1 & $0,-$ & $7,s$ & $4,s$ & $\infty,-$ & $\infty,-$ & $\infty,-$ \\
                        2 & $0,-$ & $6,b$ & $4,s$ & $2,b$ & $3,a$ & $\infty,-$ \\
                        3 & $0,-$ & $5,c$ & $4,s$ & $2,b$ & $2,a$ & $4,d$ \\
                        4 & $0,-$ & $5,c$ & $3,d$ & $2,b$ & $1,a$ & $3,d$ \\
                        5 & $0,-$ & $5,c$ & $2,d$ & $1,b$ & $1,a$ & $2,d$ \\
                        5 & $0,-$ & $4,b$ & $2,d$ & $0,b$ & $1,a$ & $2,d$ \\
                    \end{tabular}
                    \caption{Tabelle zum Algorithmus von Bellman und Ford}
                    \label{tab:bf2}
                \end{table}

        \end{enumerate}

\end{enumerate}

\end{document}
