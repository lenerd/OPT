\documentclass[a4paper]{scrartcl}

% font/encoding packages
\usepackage[utf8]{inputenc}
\usepackage[T1]{fontenc}
\usepackage{lmodern}
\usepackage[ngerman]{babel}

% \usepackage[top=2.5cm,bottom=3cm,left=1.5cm,right=1.5cm]{geometry}

\usepackage{amsmath, amssymb, amsfonts, amsthm, mathtools}
\usepackage{stmaryrd}
\usepackage{marvosym}
\setcounter{MaxMatrixCols}{13}
\allowdisplaybreaks
\usepackage[output-decimal-marker={,}]{siunitx}
\usepackage{graphicx}
\usepackage{subcaption}
\usepackage{caption}
\usepackage{venndiagram}
\usepackage[shortlabels]{enumitem}
\usepackage{polynom}
\polyset{style=C, div=:,vars=x}
\usepackage[section]{placeins}
\usepackage{tikz}
\usetikzlibrary{tikzmark}
\usetikzlibrary{calc}
\usetikzlibrary{arrows}
\usetikzlibrary{arrows.meta}
\usepackage{pgfplots}
\pgfplotsset{compat=1.5.1}
\usepgfplotslibrary{fillbetween}
\usepackage{gauss}
\usepackage{systeme}
\usepackage{cancel}

\tikzset{
    declare function={Floor(\x)=round(\x-0.5);}
}

% Matrizen
\newdimen\asep 
\asep=0.25\baselineskip 

\providecommand{\vo}{ 
    \text{\raisebox{-\asep}[0pt][0pt]{\rule[0pt]{0.8pt}{3.5\asep}}}} 
\providecommand{\vu}{ 
    \text{\raisebox{-0.5\asep}[0pt][0pt]{\rule[0pt]{0.8pt}{\baselineskip}}}} 
\providecommand{\vm}{ 
    \text{\raisebox{-\asep}[0pt][0pt]{\rule[0pt]{0.8pt}{\baselineskip}}}}

\newtheorem*{behaupt}{Behauptung}

\newcommand{\fallfac}[2]{{#1}^{\underline{#2}}}

\newcommand{\risefac}[2]{{#1}^{\overline{#2}}}

\newcommand{\dif}{\mathop{}\!\mathrm{d}}

\newcommand{\linf}[1]{\lim_{#1 \to \infty}}
\newcommand{\gdw}{\Leftrightarrow}

\renewcommand{\arraystretch}{1.3}

\title{Optimierung}
\subtitle{Blatt 9 Hausaufgaben}
\author{
	Lennart Braun (6523742, Gruppe 6) \\
    Alexander Timmermann (6524072, Gruppe 5)
}
\date{zum 15. Dezember 2014}

\begin{document}
\maketitle

\begin{enumerate}[label=\bfseries\arabic*.]
    \item % 1.
        \begin{enumerate}
            \item
                Gegeben ist folgendes LP-Problem.
                \begin{equation}
                    \begin{gathered}
                        \text{maximiere }
                        2x_1 +x_2 \\
                        \text{unter den Nebenbedingungen} \\
                        \begin{array}{rrcr}
                            -x_1 & +2x_2 & \leq & 3 \\
                            & x_2 & \leq & 2 \\
                            x_1 & & \leq & 5 \\
                        \end{array} \\
                        x_1, x_2 \geq 0
                    \end{gathered}
                \end{equation}

            \item
                Gegeben ist folgendes LP-Problem.
                \begin{equation}
                    \begin{gathered}
                        \text{maximiere }
                        -4x_1 +x_2 +6x_3 \\
                        \text{unter den Nebenbedingungen} \\
                        \begin{array}{rrrcr}
                            -2x_1 & -x_2 & +3x_3 & \leq & 2 \\
                            x_1 & +2x_2 & & \leq & 5 \\
                            -x_1 & & +x_3 & \leq & 2 \\
                        \end{array} \\
                        x_1, x_2, x_3 \geq 0
                    \end{gathered}
                \end{equation}

        \end{enumerate}

    \item % 2.
        Gegeben ist folgendes LP-Problem.
        \begin{equation}
            \begin{gathered}
                \text{maximiere }
                3x_1 -11x_2 +x_3 -9x_4 \\
                \text{unter den Nebenbedingungen} \\
                \begin{array}{rrrrcr}
                    x_1 & -7x_2 & -x_3 & -3x_4 & \leq & 1 \\
                    x_1 & +3x_2 & +x_3 & +x_4 & \leq & 3 \\
                \end{array} \\
                x_1, x_2, x_3, x_4 \geq 0
            \end{gathered}
        \end{equation}

\end{enumerate}


\end{document}
