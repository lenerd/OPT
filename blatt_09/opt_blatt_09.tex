\documentclass[a4paper]{scrartcl}

% font/encoding packages
\usepackage[utf8]{inputenc}
\usepackage[T1]{fontenc}
\usepackage{lmodern}
\usepackage[ngerman]{babel}

% \usepackage[top=2.5cm,bottom=3cm,left=1.5cm,right=1.5cm]{geometry}

\usepackage{amsmath, amssymb, amsfonts, amsthm, mathtools}
\usepackage{stmaryrd}
\usepackage{marvosym}
\setcounter{MaxMatrixCols}{13}
\allowdisplaybreaks
\usepackage[output-decimal-marker={,}]{siunitx}
\usepackage{graphicx}
\usepackage{subcaption}
\usepackage{caption}
\usepackage{venndiagram}
\usepackage[shortlabels]{enumitem}
\usepackage{polynom}
\polyset{style=C, div=:,vars=x}
\usepackage[section]{placeins}
\usepackage{tikz}
\usetikzlibrary{tikzmark}
\usetikzlibrary{calc}
\usetikzlibrary{arrows}
\usetikzlibrary{arrows.meta}
\usepackage{pgfplots}
\pgfplotsset{compat=1.5.1}
\usepgfplotslibrary{fillbetween}
\usepackage{gauss}
\usepackage{systeme}
\usepackage{cancel}

\tikzset{
    declare function={Floor(\x)=round(\x-0.5);}
}

% Matrizen
\newdimen\asep 
\asep=0.25\baselineskip 

\providecommand{\vo}{ 
    \text{\raisebox{-\asep}[0pt][0pt]{\rule[0pt]{0.8pt}{3.5\asep}}}} 
\providecommand{\vu}{ 
    \text{\raisebox{-0.5\asep}[0pt][0pt]{\rule[0pt]{0.8pt}{\baselineskip}}}} 
\providecommand{\vm}{ 
    \text{\raisebox{-\asep}[0pt][0pt]{\rule[0pt]{0.8pt}{\baselineskip}}}}

\newtheorem*{behaupt}{Behauptung}

\newcommand{\fallfac}[2]{{#1}^{\underline{#2}}}

\newcommand{\risefac}[2]{{#1}^{\overline{#2}}}

\newcommand{\dif}{\mathop{}\!\mathrm{d}}

\newcommand{\linf}[1]{\lim_{#1 \to \infty}}
\newcommand{\gdw}{\Leftrightarrow}

\renewcommand{\arraystretch}{1.3}

\title{Optimierung}
\subtitle{Blatt 9 Hausaufgaben}
\author{
	Lennart Braun (6523742, Gruppe 6) \\
    Alexander Timmermann (6524072, Gruppe 5)
}
\date{zum 15. Dezember 2014}

\begin{document}
\maketitle

\begin{enumerate}[label=\bfseries\arabic*.]
    \item % 1.
        \begin{enumerate}
            \item
                Gegeben ist folgendes LP-Problem.
                \begin{equation}
                    \begin{gathered}
                        \text{maximiere }
                        2x_1 +x_2 \\
                        \text{unter den Nebenbedingungen} \\
                        \begin{array}{rrcr}
                            -x_1 & +2x_2 & \leq & 3 \\
                            & x_2 & \leq & 2 \\
                            x_1 & & \leq & 5 \\
                        \end{array} \\
                        x_1, x_2 \geq 0
                    \end{gathered}
                \end{equation}

                Eingangsdaten:
                \begin{gather}
                    A =
                    \begin{pmatrix}
                        -1 & 2 & 1 & 0 & 0 \\
                         0 & 1 & 0 & 1 & 0 \\
                         1 & 0 & 0 & 0 & 1 \\
                    \end{pmatrix}
                    \quad
                    b =
                    \begin{pmatrix}
                        3 \\ 2 \\ 5
                    \end{pmatrix}
                    \quad
                    c^T =
                    \begin{pmatrix}
                        2 & 1 & 0 & 0 & 0
                    \end{pmatrix}
                \end{gather}

                Initialisierung:
                \begin{gather}
                    x_B^* =
                    \begin{pmatrix}
                        x_3^* \\ x_4^* \\ x_5^*
                    \end{pmatrix}
                    =
                    \begin{pmatrix}
                        3 \\ 2 \\ 5
                    \end{pmatrix}
                    \quad
                    B =
                    \begin{pmatrix}
                        1 & 0 & 0 \\
                        0 & 1 & 0 \\
                        0 & 0 & 1 \\
                    \end{pmatrix}
                \end{gather}

                \begin{enumerate}[1.]
                    \item Iteration
                        \begin{enumerate}[1.]
                            \item Schritt (Lösung von $y^TB = c_B^T$)
                                \begin{equation}
                                    y^T
                                    \begin{pmatrix}
                                        1 & 0 & 0 \\
                                        0 & 1 & 0 \\
                                        0 & 0 & 1 \\
                                    \end{pmatrix}
                                    =
                                    \begin{pmatrix}
                                        0 & 0 & 0
                                    \end{pmatrix}
                                    \quad\Rightarrow\quad
                                    y^T =
                                    \begin{pmatrix}
                                        0 & 0 & 0
                                    \end{pmatrix}
                                \end{equation}
                                
                            \item Schritt (Eingangsvariable und -spalte)
                                \begin{equation}
                                    A_N =
                                    \begin{pmatrix}
                                        -1 & 2 \\
                                        0 & 1 \\
                                        1 & 0 \\
                                    \end{pmatrix}
                                \end{equation}
                                \begin{align}
                                    y^TA_N &=
                                    \begin{pmatrix}
                                        0 & 0
                                    \end{pmatrix} \\
                                    c^TA_N &=
                                    \begin{pmatrix}
                                        2 & 1
                                    \end{pmatrix}
                                \end{align}
                                Wir wählen $x_1$ als Eingangsvariable und
                                \begin{equation}
                                    a =
                                    \begin{pmatrix}
                                        -1 \\ 0 \\ 1
                                    \end{pmatrix}
                                    \text{ .}   
                                \end{equation}

                            \item Schritt (Lösung von $Bd = a$)
                                \begin{equation}
                                    \begin{pmatrix}
                                        1 & 0 & 0 \\
                                        0 & 1 & 0 \\
                                        0 & 0 & 1 \\
                                    \end{pmatrix}
                                    d
                                    =
                                    \begin{pmatrix}
                                        -1 \\ 0 \\ 1
                                    \end{pmatrix}
                                    \quad\Rightarrow\quad
                                    d =
                                    \begin{pmatrix}
                                        -1 \\ 0 \\ 1
                                    \end{pmatrix}
                                \end{equation}

                            \item Schritt (Ausgangsvariable)
                                \begin{equation}
                                    \begin{split}
                                        x_B^* - td &\geq 0 \\
                                        \begin{pmatrix}
                                            3 \\ 2 \\ 5
                                        \end{pmatrix}
                                        - t
                                        \begin{pmatrix}
                                            -1 \\ 0 \\ 1
                                        \end{pmatrix}
                                        &\geq 0
                                        \quad\Rightarrow\quad
                                        t \leq 5
                                    \end{split}
                                \end{equation}
                                $t = 5$
                                \begin{equation}
                                    \begin{pmatrix}
                                        3 \\ 2 \\ 5
                                    \end{pmatrix}
                                    - t
                                    \begin{pmatrix}
                                        -1 \\ 0 \\ 1
                                    \end{pmatrix}
                                    =
                                    \begin{pmatrix}
                                        8 \\ 2 \\ 0
                                    \end{pmatrix}
                                \end{equation}
                                Wir wählen $x_5$ als Ausgangsvariable.

                            \item Schritt (Update)
                                \begin{gather}
                                    x_B^* =
                                    \begin{pmatrix}
                                        x_3^* \\ x_4^* \\ x_1^*
                                    \end{pmatrix}
                                    =
                                    \begin{pmatrix}
                                        8 \\ 2 \\ 5
                                    \end{pmatrix}
                                    \quad
                                    B =
                                    \begin{pmatrix}
                                        1 & 0 & -1 \\
                                        0 & 1 & 0 \\
                                        0 & 0 & 1 \\
                                    \end{pmatrix}
                                \end{gather}

                        \end{enumerate}

                    \item Iteration
                        \begin{enumerate}[1.]
                            \item Schritt (Lösung von $y^TB = c_B^T$)
                                \begin{equation}
                                    y^T
                                    \begin{pmatrix}
                                        1 & 0 & -1 \\
                                        0 & 1 & 0 \\
                                        0 & 0 & 1 \\
                                    \end{pmatrix}
                                    =
                                    \begin{pmatrix}
                                        0 & 0 & 2
                                    \end{pmatrix}
                                    \quad\Rightarrow\quad
                                    y^T =
                                    \begin{pmatrix}
                                        0 & 0 & 2
                                    \end{pmatrix}
                                \end{equation}
                                
                            \item Schritt (Eingangsvariable und -spalte)
                                \begin{equation}
                                    A_N =
                                    \begin{pmatrix}
                                        2 & 0 \\
                                        1 & 0 \\
                                        0 & 1 \\
                                    \end{pmatrix}
                                \end{equation}
                                \begin{align}
                                    y^TA_N &=
                                    \begin{pmatrix}
                                        0 & 2
                                    \end{pmatrix} \\
                                    c^TA_N &=
                                    \begin{pmatrix}
                                        1 & 0
                                    \end{pmatrix}
                                \end{align}
                                Wir wählen $x_2$ als Eingangsvariable und
                                \begin{equation}
                                    a =
                                    \begin{pmatrix}
                                        2 \\ 1 \\ 0
                                    \end{pmatrix}
                                    \text{ .}   
                                \end{equation}

                            \item Schritt (Lösung von $Bd = a$)
                                \begin{equation}
                                    \begin{pmatrix}
                                        1 & 0 & -1 \\
                                        0 & 1 & 0 \\
                                        0 & 0 & 1 \\
                                    \end{pmatrix}
                                    d
                                    =
                                    \begin{pmatrix}
                                        2 \\ 1 \\ 0
                                    \end{pmatrix}
                                    \quad\Rightarrow\quad
                                    d =
                                    \begin{pmatrix}
                                        2 \\ 1 \\ 0
                                    \end{pmatrix}
                                \end{equation}

                            \item Schritt (Ausgangsvariable)
                                \begin{equation}
                                    \begin{split}
                                        x_B^* - td &\geq 0 \\
                                        \begin{pmatrix}
                                            3 \\ 2 \\ 5
                                        \end{pmatrix}
                                        - t
                                        \begin{pmatrix}
                                            2 \\ 1 \\ 0
                                        \end{pmatrix}
                                        &\geq 0
                                        \quad\Rightarrow\quad
                                        t \leq 2
                                    \end{split}
                                \end{equation}
                                $t = 2$
                                \begin{equation}
                                    \begin{pmatrix}
                                        3 \\ 2 \\ 5
                                    \end{pmatrix}
                                    - t
                                    \begin{pmatrix}
                                        2 \\ 1 \\ 0
                                    \end{pmatrix}
                                    =
                                    \begin{pmatrix}
                                        4 \\ 0 \\ 5
                                    \end{pmatrix}
                                \end{equation}
                                Wir wählen $x_5$ als Ausgangsvariable.

                            \item Schritt (Update)
                                \begin{gather}
                                    x_B^* =
                                    \begin{pmatrix}
                                        x_3^* \\ x_2^* \\ x_1^*
                                    \end{pmatrix}
                                    =
                                    \begin{pmatrix}
                                        4 \\ 2 \\ 5
                                    \end{pmatrix}
                                    \quad
                                    B =
                                    \begin{pmatrix}
                                        1 & 2 & -1 \\
                                        0 & 1 & 0 \\
                                        0 & 0 & 1 \\
                                    \end{pmatrix}
                                \end{gather}

                        \end{enumerate}

                    \item Iteration
                        \begin{enumerate}[1.]
                            \item Schritt (Lösung von $y^TB = c_B^T$)
                                \begin{equation}
                                    y^T
                                    \begin{pmatrix}
                                        1 & 2 & -1 \\
                                        0 & 1 & 0 \\
                                        0 & 0 & 1 \\
                                    \end{pmatrix}
                                    =
                                    \begin{pmatrix}
                                        0 & 1 & 2
                                    \end{pmatrix}
                                    \quad\Rightarrow\quad
                                    y^T =
                                    \begin{pmatrix}
                                        0 & 1 & 2
                                    \end{pmatrix}
                                \end{equation}
                                
                            \item Schritt (Eingangsvariable und -spalte)
                                \begin{equation}
                                    A_N =
                                    \begin{pmatrix}
                                        0 & 0 \\
                                        1 & 0 \\
                                        0 & 1 \\
                                    \end{pmatrix}
                                \end{equation}
                                \begin{align}
                                    y^TA_N &=
                                    \begin{pmatrix}
                                        1 & 2
                                    \end{pmatrix} \\
                                    c^TA_N &=
                                    \begin{pmatrix}
                                        0 & 0
                                    \end{pmatrix}
                                \end{align}
                                Da $y^TA_N$ komponentenweise größer als $c^TA_N$
                                ist, haben wir mit $x^*$ die optimale Lösung für
                                das LP-Problem gefunden.
                                \begin{equation}
                                    x^* =
                                    \begin{pmatrix}
                                        5 \\ 2 \\ 4 \\ 0 \\ 0
                                    \end{pmatrix}
                                \end{equation}
                                Das optimale Ergebnis lautet
                                \begin{equation}
                                    z = c^T \cdot x^* = 10 + 2 = 12
                                \end{equation}
                                
                        \end{enumerate}

                \end{enumerate}
                

            \item
                Gegeben ist folgendes LP-Problem.
                \begin{equation}
                    \begin{gathered}
                        \text{maximiere }
                        -4x_1 +x_2 +6x_3 \\
                        \text{unter den Nebenbedingungen} \\
                        \begin{array}{rrrcr}
                            -2x_1 & -x_2 & +3x_3 & \leq & 2 \\
                            x_1 & +2x_2 & & \leq & 5 \\
                            -x_1 & & +x_3 & \leq & 2 \\
                        \end{array} \\
                        x_1, x_2, x_3 \geq 0
                    \end{gathered}
                \end{equation}

                Eingangsdaten:
                \begin{gather}
                    A =
                    \begin{pmatrix}
                        -2 & -1 & 3 & 1 & 0 & 0 \\
                         1 &  2 & 0 & 0 & 1 & 0 \\
                        -1 &  0 & 1 & 0 & 0 & 1 \\
                    \end{pmatrix}
                    \quad
                    b =
                    \begin{pmatrix}
                        2 \\ 5 \\ 2
                    \end{pmatrix}
                    \quad
                    c^T =
                    \begin{pmatrix}
                        -4 & 1 & 6 & 0 & 0 & 0
                    \end{pmatrix}
                \end{gather}

                Initialisierung:
                \begin{gather}
                    x_B^* =
                    \begin{pmatrix}
                        x_4^* \\ x_5^* \\ x_6^*
                    \end{pmatrix}
                    =
                    \begin{pmatrix}
                        2 \\ 5 \\ 2
                    \end{pmatrix}
                    \quad
                    B =
                    \begin{pmatrix}
                        1 & 0 & 0 \\
                        0 & 1 & 0 \\
                        0 & 0 & 1 \\
                    \end{pmatrix}
                \end{gather}

                \begin{enumerate}[1.]
                    \item Iteration
                        \begin{enumerate}[1.]
                            \item Schritt (Lösung von $y^TB = c_B^T$)
                                \begin{equation}
                                    y^T
                                    \begin{pmatrix}
                                        1 & 0 & 0 \\
                                        0 & 1 & 0 \\
                                        0 & 0 & 1 \\
                                    \end{pmatrix}
                                    =
                                    \begin{pmatrix}
                                        0 & 0 & 0
                                    \end{pmatrix}
                                    \quad\Rightarrow\quad
                                    y^T =
                                    \begin{pmatrix}
                                        0 & 0 & 0
                                    \end{pmatrix}
                                \end{equation}
                                
                            \item Schritt (Eingangsvariable und -spalte)
                                \begin{equation}
                                    A_N =
                                    \begin{pmatrix}
                                        -2 & -1 &  3 \\
                                         1 &  2 &  0 \\
                                        -1 &  0 &  1 \\
                                    \end{pmatrix}
                                \end{equation}
                                \begin{align}
                                    y^TA_N &=
                                    \begin{pmatrix}
                                        0 & 0 & 0
                                    \end{pmatrix} \\
                                    c^TA_N &=
                                    \begin{pmatrix}
                                        -4 & 1 & 6
                                    \end{pmatrix}
                                \end{align}
                                Wir wählen $x_3$ als Eingangsvariable und
                                \begin{equation}
                                    a =
                                    \begin{pmatrix}
                                        3 \\ 0 \\ 1
                                    \end{pmatrix}
                                    \text{ .}   
                                \end{equation}

                            \item Schritt (Lösung von $Bd = a$)
                                \begin{equation}
                                    \begin{pmatrix}
                                        1 & 0 & 0 \\
                                        0 & 1 & 0 \\
                                        0 & 0 & 1 \\
                                    \end{pmatrix}
                                    d
                                    =
                                    \begin{pmatrix}
                                        3 \\ 0 \\ 1
                                    \end{pmatrix}
                                    \quad\Rightarrow\quad
                                    d =
                                    \begin{pmatrix}
                                        3 \\ 0 \\ 1
                                    \end{pmatrix}
                                \end{equation}

                            \item Schritt (Ausgangsvariable)
                                \begin{equation}
                                    \begin{split}
                                        x_B^* - td &\geq 0 \\
                                        \begin{pmatrix}
                                            2 \\ 5 \\ 2
                                        \end{pmatrix}
                                        - t
                                        \begin{pmatrix}
                                            3 \\ 0 \\ 1
                                        \end{pmatrix}
                                        &\geq 0
                                        \quad\Rightarrow\quad
                                        t \leq \frac{2}{3}
                                    \end{split}
                                \end{equation}
                                $t = \frac{2}{3}$
                                \begin{equation}
                                    \begin{pmatrix}
                                        2 \\ 5 \\ 2
                                    \end{pmatrix}
                                    - t
                                    \begin{pmatrix}
                                        3 \\ 0 \\ 1
                                    \end{pmatrix}
                                    =
                                    \begin{pmatrix}
                                        0 \\ 5 \\ \frac{4}{3}
                                    \end{pmatrix}
                                \end{equation}
                                Wir wählen $x_4$ als Ausgangsvariable.

                            \item Schritt (Update)
                                \begin{gather}
                                    x_B^* =
                                    \begin{pmatrix}
                                         x_3^* \\ x_5^* \\ x_6^*
                                    \end{pmatrix}
                                    =
                                    \begin{pmatrix}
                                        \frac{2}{3} \\ 5 \\ \frac{4}{3}
                                    \end{pmatrix}
                                    \quad
                                    B =
                                    \begin{pmatrix}
                                        3 & 0 & 0 \\
                                        0 & 1 & 0 \\
                                        1 & 0 & 1 \\
                                    \end{pmatrix}
                                \end{gather}

                        \end{enumerate}

                    \item Iteration
                        \begin{enumerate}[1.]
                            \item Schritt (Lösung von $y^TB = c_B^T$)
                                \begin{equation}
                                    y^T
                                    \begin{pmatrix}
                                        3 & 0 & 0 \\
                                        0 & 1 & 0 \\
                                        1 & 0 & 1 \\
                                    \end{pmatrix}
                                    =
                                    \begin{pmatrix}
                                        6 & 0 & 0
                                    \end{pmatrix}
                                    \quad\Rightarrow\quad
                                    y^T =
                                    \begin{pmatrix}
                                        2 & 0 & 0
                                    \end{pmatrix}
                                \end{equation}
                                
                            \item Schritt (Eingangsvariable und -spalte)
                                \begin{equation}
                                    A_N =
                                    \begin{pmatrix}
                                        -2 & -1 &  1 \\
                                         1 &  2 &  0 \\
                                        -1 &  0 &  0 \\
                                    \end{pmatrix}
                                \end{equation}
                                \begin{align}
                                    y^TA_N &=
                                    \begin{pmatrix}
                                        -4 & -2 & 2
                                    \end{pmatrix} \\
                                    c^TA_N &=
                                    \begin{pmatrix}
                                        -4 & 1 & 0
                                    \end{pmatrix}
                                \end{align}
                                Wir wählen $x_2$ als Eingangsvariable und
                                \begin{equation}
                                    a =
                                    \begin{pmatrix}
                                        -1 \\ 2 \\ 0
                                    \end{pmatrix}
                                    \text{ .}   
                                \end{equation}

                            \item Schritt (Lösung von $Bd = a$)
                                \begin{equation}
                                    \begin{pmatrix}
                                        3 & 0 & 0 \\
                                        0 & 1 & 0 \\
                                        1 & 0 & 1 \\
                                    \end{pmatrix}
                                    d
                                    =
                                    \begin{pmatrix}
                                        -1 \\ 2 \\ 0
                                    \end{pmatrix}
                                    \quad\Rightarrow\quad
                                    d =
                                    \begin{pmatrix}
                                        -\frac{1}{3} \\ 2 \\ \frac{1}{3}
                                    \end{pmatrix}
                                \end{equation}

                            \item Schritt (Ausgangsvariable)
                                \begin{equation}
                                    \begin{split}
                                        x_B^* - td &\geq 0 \\
                                        \begin{pmatrix}
                                            \frac{2}{3} \\ 5 \\ \frac{4}{3}
                                        \end{pmatrix}
                                        - t
                                        \begin{pmatrix}
                                            -\frac{1}{3} \\ 2 \\ \frac{1}{3}
                                        \end{pmatrix}
                                        &\geq 0
                                        \quad\Rightarrow\quad
                                        t \leq \frac{5}{2}
                                    \end{split}
                                \end{equation}
                                $t = \frac{5}{2}$
                                \begin{equation}
                                    \begin{pmatrix}
                                        \frac{2}{3} \\ 5 \\ \frac{4}{3}
                                    \end{pmatrix}
                                    - t
                                    \begin{pmatrix}
                                        -\frac{1}{3} \\ 2 \\ \frac{1}{3}
                                    \end{pmatrix}
                                    =
                                    \begin{pmatrix}
                                        \frac{3}{2} \\ 0 \\ \frac{1}{2}
                                    \end{pmatrix}
                                \end{equation}
                                Wir wählen $x_5$ als Ausgangsvariable.

                            \item Schritt (Update)
                                \begin{gather}
                                    x_B^* =
                                    \begin{pmatrix}
                                         x_3^* \\ x_2^* \\ x_6^*
                                    \end{pmatrix}
                                    =
                                    \begin{pmatrix}
                                        \frac{3}{2} \\ \frac{5}{2} \\ \frac{1}{2}
                                    \end{pmatrix}
                                    \quad
                                    B =
                                    \begin{pmatrix}
                                        3 & -1 & 0 \\
                                        0 &  2 & 0 \\
                                        1 &  0 & 1 \\
                                    \end{pmatrix}
                                \end{gather}

                        \end{enumerate}

                    \item Iteration
                        \begin{enumerate}[1.]
                            \item Schritt (Lösung von $y^TB = c_B^T$)
                                \begin{equation}
                                    y^T
                                    \begin{pmatrix}
                                        3 & -1 & 0 \\
                                        0 &  2 & 0 \\
                                        1 &  0 & 1 \\
                                    \end{pmatrix}
                                    =
                                    \begin{pmatrix}
                                        6 & 1 & 0
                                    \end{pmatrix}
                                    \quad\Rightarrow\quad
                                    y^T =
                                    \begin{pmatrix}
                                        2 & \frac{2}{3} & 0
                                    \end{pmatrix}
                                \end{equation}
                                
                            \item Schritt (Eingangsvariable und -spalte)
                                \begin{equation}
                                    A_N =
                                    \begin{pmatrix}
                                        -2 & 1 &  0 \\
                                         1 & 0 &  1 \\
                                        -1 & 0 &  0 \\
                                    \end{pmatrix}
                                \end{equation}
                                \begin{align}
                                    y^TA_N &=
                                    \begin{pmatrix}
                                        -\frac{1}{2} & 2 & \frac{3}{2}
                                    \end{pmatrix} \\
                                    c^TA_N &=
                                    \begin{pmatrix}
                                        -4 & 0 & 0
                                    \end{pmatrix}
                                \end{align}
                                Da $y^TA_N$ komponentenweise größer als $c^TA_N$
                                ist, haben wir mit $x^*$ die optimale Lösung für
                                das LP-Problem gefunden.
                                \begin{equation}
                                    x^* =
                                    \begin{pmatrix}
                                        0 \\ \frac{5}{2} \\ \frac{3}{2} \\ 0 \\ 0 \\ \frac{1}{2}
                                    \end{pmatrix}
                                \end{equation}
                                Das optimale Ergebnis lautet
                                \begin{equation}
                                    z = c^T \cdot x^* = \frac{5}{2} + 9 = \frac{23}{2} = 11.5
                                \end{equation}

                        \end{enumerate}

                \end{enumerate}

        \end{enumerate}

    \item % 2.
        Gegeben ist folgendes LP-Problem.
        \begin{equation}
            \begin{gathered}
                \text{maximiere }
                3x_1 -11x_2 +x_3 -9x_4 \\
                \text{unter den Nebenbedingungen} \\
                \begin{array}{rrrrcr}
                    x_1 & -7x_2 & -x_3 & -3x_4 & \leq & 1 \\
                    x_1 & +3x_2 & +x_3 & +x_4 & \leq & 3 \\
                \end{array} \\
                x_1, x_2, x_3, x_4 \geq 0
            \end{gathered}
        \end{equation}

        Eingangsdaten:
        \begin{gather}
            A =
            \begin{pmatrix}
                1 & -7 & -1 & -3 & 1 & 0 \\
                1 &  3 &  1 &  1 & 0 & 1 \\
            \end{pmatrix}
            \quad
            b =
            \begin{pmatrix}
                1 \\ 3
            \end{pmatrix}
            \quad
            c^T =
            \begin{pmatrix}
                3 & -11 & 1 & -9 & 0 & 0
            \end{pmatrix}
        \end{gather}

        Initialisierung:
        \begin{gather}
            x_B^* =
            \begin{pmatrix}
                x_5^* \\ x_6^*
            \end{pmatrix}
            =
            \begin{pmatrix}
                1 & 3
            \end{pmatrix}
            \quad
            B =
            \begin{pmatrix}
                1 & 0 \\
                0 & 1 \\
            \end{pmatrix}
        \end{gather}

        \begin{enumerate}[1.]
            \item Iteration
                \begin{enumerate}[1.]
                    \item Schritt (Lösung von $y^TB = c_B^T$)
                        \begin{equation}
                            y^T
                            \begin{pmatrix}
                                1 & 0 \\
                                0 & 1 \\
                            \end{pmatrix}
                            =
                            \begin{pmatrix}
                                0 & 0
                            \end{pmatrix}
                            \quad\Rightarrow\quad
                            y^T =
                            \begin{pmatrix}
                                0 & 0
                            \end{pmatrix}
                        \end{equation}
                        
                    \item Schritt (Eingangsvariable und -spalte)
                        \begin{equation}
                            A_N =
                            \begin{pmatrix}
                                1 & -7 & -1 & -3 \\
                                1 &  3 &  1 &  1 \\
                            \end{pmatrix}
                        \end{equation}
                        \begin{align}
                            y^TA_N &=
                            \begin{pmatrix}
                                0 & 0 & 0 & 0 \\
                            \end{pmatrix} \\
                            c^TA_N &=
                            \begin{pmatrix}
                                3 & -11 & 1 & -9
                            \end{pmatrix}
                        \end{align}
                        Wir wählen $x_1$ als Eingangsvariable und
                        \begin{equation}
                            a =
                            \begin{pmatrix}
                                1 \\ 1
                            \end{pmatrix}
                            \text{ .}   
                        \end{equation}

                    \item Schritt (Lösung von $Bd = a$)
                        \begin{equation}
                            \begin{pmatrix}
                                1 & 0 \\
                                0 & 1 \\
                            \end{pmatrix}
                            d
                            =
                            \begin{pmatrix}
                                1 \\ 1
                            \end{pmatrix}
                            \quad\Rightarrow\quad
                            d =
                            \begin{pmatrix}
                                1 \\ 1
                            \end{pmatrix}
                        \end{equation}

                    \item Schritt (Ausgangsvariable)
                        \begin{equation}
                            \begin{split}
                                x_B^* - td &\geq 0 \\
                                \begin{pmatrix}
                                    1 \\ 3
                                \end{pmatrix}
                                - t
                                \begin{pmatrix}
                                    1 \\ 1
                                \end{pmatrix}
                                &\geq 0
                                \quad\Rightarrow\quad
                                t \leq 1
                            \end{split}
                        \end{equation}
                        $t = 1$
                        \begin{equation}
                            \begin{pmatrix}
                                1 \\ 3
                            \end{pmatrix}
                            - t
                            \begin{pmatrix}
                                1 \\ 1
                            \end{pmatrix}
                            =
                            \begin{pmatrix}
                                0 \\ 2
                            \end{pmatrix}
                        \end{equation}
                        Wir wählen $x_5$ als Ausgangsvariable.

                    \item Schritt (Update)
                        \begin{gather}
                            x_B^* =
                            \begin{pmatrix}
                                x_1^* \\ x_6^*
                            \end{pmatrix}
                            =
                            \begin{pmatrix}
                                1 \\ 2
                            \end{pmatrix}
                            \quad
                            B =
                            \begin{pmatrix}
                                1 & 0 \\
                                1 & 1 \\
                            \end{pmatrix}
                        \end{gather}

                \end{enumerate}

            \item Iteration
                \begin{enumerate}[1.]
                    \item Schritt (Lösung von $y^TB = c_B^T$)
                        \begin{equation}
                            y^T
                            \begin{pmatrix}
                                1 & 0 \\
                                1 & 1 \\
                            \end{pmatrix}
                            =
                            \begin{pmatrix}
                                3 & 0
                            \end{pmatrix}
                            \quad\Rightarrow\quad
                            y^T =
                            \begin{pmatrix}
                                3 & 0
                            \end{pmatrix}
                        \end{equation}
                        
                    \item Schritt (Eingangsvariable und -spalte)
                        \begin{equation}
                            A_N =
                            \begin{pmatrix}
                                -7 & -1 & -3 & 1 \\
                                 3 &  1 &  1 & 0 \\
                            \end{pmatrix}
                        \end{equation}
                        \begin{align}
                            y^TA_N &=
                            \begin{pmatrix}
                                -21 & -3 & -9 & 3
                            \end{pmatrix} \\
                            c^TA_N &=
                            \begin{pmatrix}
                                -11 & 1 & -9 & 0
                            \end{pmatrix}
                        \end{align}
                        Wir wählen $x_2$ als Eingangsvariable und
                        \begin{equation}
                            a =
                            \begin{pmatrix}
                                -7 \\ 3
                            \end{pmatrix}
                            \text{ .}   
                        \end{equation}

                    \item Schritt (Lösung von $Bd = a$)
                        \begin{equation}
                            \begin{pmatrix}
                                1 & 0 \\
                                1 & 1 \\
                            \end{pmatrix}
                            d
                            =
                            \begin{pmatrix}
                                -7 \\ 3
                            \end{pmatrix}
                            \quad\Rightarrow\quad
                            d =
                            \begin{pmatrix}
                                -7 \\ 10
                            \end{pmatrix}
                        \end{equation}

                    \item Schritt (Ausgangsvariable)
                        \begin{equation}
                            \begin{split}
                                x_B^* - td &\geq 0 \\
                                \begin{pmatrix}
                                    1 \\ 2
                                \end{pmatrix}
                                - t
                                \begin{pmatrix}
                                    -7 \\ 10
                                \end{pmatrix}
                                &\geq 0
                                \quad\Rightarrow\quad
                                t \leq \frac{12}{5}
                            \end{split}
                        \end{equation}
                        $t = \frac{12}{5}$
                        \begin{equation}
                            \begin{pmatrix}
                                1 \\ 2
                            \end{pmatrix}
                            - t
                            \begin{pmatrix}
                                -7 \\ 10
                            \end{pmatrix}
                            =
                            \begin{pmatrix}
                                \frac{12}{5} \\ 0
                            \end{pmatrix}
                        \end{equation}
                        Wir wählen $x_6$ als Ausgangsvariable.

                    \item Schritt (Update)
                        \begin{gather}
                            x_B^* =
                            \begin{pmatrix}
                                x_1^* \\ x_2^*
                            \end{pmatrix}
                            =
                            \begin{pmatrix}
                                \frac{12}{5} \\ \frac{1}{5}
                            \end{pmatrix}
                            \quad
                            B =
                            \begin{pmatrix}
                                1 & -7 \\
                                1 &  3 \\
                            \end{pmatrix}
                        \end{gather}

                \end{enumerate}

            \item Iteration
                \begin{enumerate}[1.]
                    \item Schritt (Lösung von $y^TB = c_B^T$)
                        \begin{equation}
                            y^T
                            \begin{pmatrix}
                                1 & -7 \\
                                1 &  3 \\
                            \end{pmatrix}
                            =
                            \begin{pmatrix}
                                3 & -11
                            \end{pmatrix}
                            \quad\Rightarrow\quad
                            y^T =
                            \begin{pmatrix}
                                2 & 1
                            \end{pmatrix}
                        \end{equation}
                        
                    \item Schritt (Eingangsvariable und -spalte)
                        \begin{equation}
                            A_N =
                            \begin{pmatrix}
                                -1 & -3 & 1 & 0 \\
                                 1 &  1 & 0 & 1 \\
                            \end{pmatrix}
                        \end{equation}
                        \begin{align}
                            y^TA_N &=
                            \begin{pmatrix}
                                -1 & -5 & 2 & 1
                            \end{pmatrix} \\
                            c^TA_N &=
                            \begin{pmatrix}
                                1 & -9 & 0 & 0
                            \end{pmatrix}
                        \end{align}
                        Wir wählen $x_3$ als Eingangsvariable und
                        \begin{equation}
                            a =
                            \begin{pmatrix}
                                -1 & 1
                            \end{pmatrix}
                            \text{ .}   
                        \end{equation}

                    \item Schritt (Lösung von $Bd = a$)
                        \begin{equation}
                            \begin{pmatrix}
                                1 & -7 \\
                                1 &  3 \\
                            \end{pmatrix}
                            d
                            =
                            \begin{pmatrix}
                                -1 & 1
                            \end{pmatrix}
                            \quad\Rightarrow\quad
                            d =
                            \begin{pmatrix}
                                \frac{2}{5} \\ \frac{1}{5}
                            \end{pmatrix}
                        \end{equation}

                    \item Schritt (Ausgangsvariable)
                        \begin{equation}
                            \begin{split}
                                x_B^* - td &\geq 0 \\
                                \begin{pmatrix}
                                    \frac{12}{5} \\ \frac{1}{5}
                                \end{pmatrix}
                                - t
                                \begin{pmatrix}
                                    \frac{2}{5} \\ \frac{1}{5}
                                \end{pmatrix}
                                &\geq 0
                                \quad\Rightarrow\quad
                                t \leq 1
                            \end{split}
                        \end{equation}
                        $t = 1$
                        \begin{equation}
                            \begin{pmatrix}
                                \frac{12}{5} \\ \frac{1}{5}
                            \end{pmatrix}
                            - t
                            \begin{pmatrix}
                                \frac{2}{5} \\ \frac{1}{5}
                            \end{pmatrix}
                            =
                            \begin{pmatrix}
                                2 \\ 0
                            \end{pmatrix}
                        \end{equation}
                        Wir wählen $x_2$ als Ausgangsvariable.

                    \item Schritt (Update)
                        \begin{gather}
                            x_B^* =
                            \begin{pmatrix}
                                x_1^* \\ x_3^*
                            \end{pmatrix}
                            =
                            \begin{pmatrix}
                                2 \\ 1
                            \end{pmatrix}
                            \quad
                            B =
                            \begin{pmatrix}
                                1 & -1 \\
                                1 &  1 \\
                            \end{pmatrix}
                        \end{gather}

                \end{enumerate}

            \item Iteration
                \begin{enumerate}[1.]
                    \item Schritt (Lösung von $y^TB = c_B^T$)
                        \begin{equation}
                            y^T
                            \begin{pmatrix}
                                1 & -1 \\
                                1 &  1 \\
                            \end{pmatrix}
                            =
                            \begin{pmatrix}
                                3 & 9
                            \end{pmatrix}
                            \quad\Rightarrow\quad
                            y^T =
                            \begin{pmatrix}
                                -3 & 6
                            \end{pmatrix}
                        \end{equation}
                        
                    \item Schritt (Eingangsvariable und -spalte)
                        \begin{equation}
                            A_N =
                            \begin{pmatrix}
                                -7 & -3 & 1 & 0 \\
                                 3 &  1 & 0 & 1 \\
                            \end{pmatrix}
                        \end{equation}
                        \begin{align}
                            y^TA_N &=
                            \begin{pmatrix}
                                39 & 15 & 3 & 6
                            \end{pmatrix} \\
                            c^TA_N &=
                            \begin{pmatrix}
                                -11 & -9 & 0 & 0
                            \end{pmatrix}
                        \end{align}
                        Da $y^TA_N$ komponentenweise größer als $c^TA_N$
                        ist, haben wir mit $x^*$ die optimale Lösung für
                        das LP-Problem gefunden.
                        \begin{equation}
                            x^* =
                            \begin{pmatrix}
                                2 \\ 0 \\ 1 \\ 0 \\ 0 \\ 0
                            \end{pmatrix}
                        \end{equation}
                        Das optimale Ergebnis lautet
                        \begin{equation}
                            z = c^T \cdot x^* = 6 + 1 = 7
                        \end{equation}

                \end{enumerate}

        \end{enumerate}


\end{enumerate}


\end{document}
