\documentclass[a4paper]{scrartcl}

% font/encoding packages
\usepackage[utf8]{inputenc}
\usepackage[T1]{fontenc}
\usepackage{lmodern}
\usepackage[ngerman]{babel}

\usepackage[top=1.1in, bottom=1.5in]{geometry}

\usepackage{amsmath, amssymb, amsfonts, amsthm, mathtools}
\usepackage{stmaryrd}
\usepackage{marvosym}
\setcounter{MaxMatrixCols}{13}
\allowdisplaybreaks
\usepackage[output-decimal-marker={,}]{siunitx}
\usepackage{graphicx}
\usepackage{subcaption}
\usepackage{caption}
\usepackage{venndiagram}
\usepackage[shortlabels]{enumitem}
\usepackage{polynom}
\polyset{style=C, div=:,vars=x}
\usepackage[section]{placeins}
\usepackage{tikz}
\usetikzlibrary{tikzmark}
\usetikzlibrary{calc}
\usetikzlibrary{arrows}
\usetikzlibrary{arrows.meta}
\usetikzlibrary{decorations}
\usepackage{pgfplots}
\pgfplotsset{compat=1.5.1}
\usepgfplotslibrary{fillbetween}
\usepackage{gauss}
\usepackage{systeme}
\usepackage{cancel}

\tikzset{
    declare function={Floor(\x)=round(\x-0.5);}
}

% Matrizen
\newdimen\asep 
\asep=0.25\baselineskip 

\providecommand{\vo}{ 
    \text{\raisebox{-\asep}[0pt][0pt]{\rule[0pt]{0.8pt}{3.5\asep}}}} 
\providecommand{\vu}{ 
    \text{\raisebox{-0.5\asep}[0pt][0pt]{\rule[0pt]{0.8pt}{\baselineskip}}}} 
\providecommand{\vm}{ 
    \text{\raisebox{-\asep}[0pt][0pt]{\rule[0pt]{0.8pt}{\baselineskip}}}}

\newtheorem*{behaupt}{Behauptung}

\newcommand{\fallfac}[2]{{#1}^{\underline{#2}}}

\newcommand{\risefac}[2]{{#1}^{\overline{#2}}}

\newcommand{\dif}{\mathop{}\!\mathrm{d}}

\newcommand{\linf}[1]{\lim_{#1 \to \infty}}
\newcommand{\gdw}{\Leftrightarrow}

\DeclareSIUnit\calorie{cal}

\renewcommand{\arraystretch}{1.3}

\title{Optimierung}
\subtitle{Blatt 12 Hausaufgaben}
\author{
	Lennart Braun (6523742, Gruppe 6) \\
    Alexander Timmermann (6524072, Gruppe 5)
}
\date{zum 19. Januar 2015}

\begin{document}
\maketitle

\begin{enumerate}[label=\bfseries\arabic*.]
    \item
        \begin{enumerate}
            \item \hfill \\
                \begin{table}[h]
                    \centering
                    \begin{tabular}{c||c|c|c|c|c|c|c|c||l}
                        & $s$ & $a$ & $b$ & $c$ & $d$ & $e$ & $f$ & $g$ & $S$
                        \\ \hline \hline
                        0 & $\underline{0}$ & $1$ $s$ & $\infty$ & $\infty$ & $4$ $s$ & $8$ $s$ & $\infty$ & $\infty$ & $\{ s \}$
                        \\ \hline
                        1 & $\underline{0}$ & $\underline{1}$ $s$ & $3$ $a$ & $\infty$ & $4$ $s$ & $7$ $a$ & $7$ $a$ & $\infty$ & $\{ s, a \}$
                        \\ \hline
                        2 & $\underline{0}$ & $\underline{1}$ $s$ & $\underline{3}$ $a$ & $4$ $b$ & $4$ $s$ & $7$ $a$ & $5$ $b$ & $\infty$ & $\{ s, a, b \}$
                        \\ \hline
                        3 & $\underline{0}$ & $\underline{1}$ $s$ & $\underline{3}$ $a$ & $\underline{4}$ $b$ & $4$ $s$ & $7$ $a$ & $5$ $b$ & $8$ $c$ & $\{ s, a, b, c \}$
                        \\ \hline
                        4 & $\underline{0}$ & $\underline{1}$ $s$ & $\underline{3}$ $a$ & $\underline{4}$ $b$ & $\underline{4}$ $s$ & $7$ $a$ & $5$ $b$ & $8$ $c$ & $\{ s, a, b, c, d \}$
                        \\ \hline
                        5 & $\underline{0}$ & $\underline{1}$ $s$ & $\underline{3}$ $a$ & $\underline{4}$ $b$ & $\underline{4}$ $s$ & $7$ $a$ & $\underline{5}$ $b$ & $8$ $c$ & $\{ s, a, b, c, d, f \}$
                        \\ \hline
                        6 & $\underline{0}$ & $\underline{1}$ $s$ & $\underline{3}$ $a$ & $\underline{4}$ $b$ & $\underline{4}$ $s$ & $\underline{6}$ $f$ & $\underline{5}$ $b$ & $6$ $f$ & $\{ s, a, b, c, d, f, e \}$
                        \\ \hline
                        6 & $\underline{0}$ & $\underline{1}$ $s$ & $\underline{3}$ $a$ & $\underline{4}$ $b$ & $\underline{4}$ $s$ & $\underline{6}$ $f$ & $\underline{5}$ $b$ & $\underline{6}$ $f$ & $\{ s, a, b, c, d, f, e, g \}$
                    \end{tabular}
                    \caption{Ausführung von Dijkstras Algorithmus auf $G$}
                    \label{tab:dijkstra}
                \end{table}
                \begin{figure}[h]
                    \centering
                    \begin{tikzpicture}[
                            auto,
                            scale=2.5,
                        ]
                        \tikzstyle{vertex}=[
                            circle,
                            fill,
                        ]
                        \tikzstyle{edge}=[
                            ->,
                            >=stealth',
                        ]
                        \node [vertex, label=above:$s$] (s) at (0,1) {};
                        \node [vertex, label=above:$a$] (a) at (1,1) {};
                        \node [vertex, label=above:$b$] (b) at (2,1) {};
                        \node [vertex, label=above:$c$] (c) at (3,1) {};
                        \node [vertex, label=below:$d$] (d) at (0,0) {};
                        \node [vertex, label=below:$e$] (e) at (1,0) {};
                        \node [vertex, label=below:$f$] (f) at (2,0) {};
                        \node [vertex, label=below:$g$] (g) at (3,0) {};

                        \draw [edge] (s) to node {$1$} (a);
                        \draw [edge] (a) to node {$2$} (b);
                        \draw [edge] (b) to node {$1$} (c);
                        \draw [edge] (s) to node {$4$} (d);
                        \draw [edge] (b) to node {$2$} (f);
                        \draw [edge] (f) to node {$1$} (e);
                        \draw [edge] (f) to node {$1$} (g);
                    \end{tikzpicture}
                    \caption{kürzeste-Pfade-Baum für $G$ und $s$}
                    \label{fig:dijkstra}
                \end{figure}

            \item
                Es reicht ein Gegenbeispiel, um zu zeigen, dass die angegebene
                Strategie nicht immer die optimale Lösung liefert.
                
                Sei ein Rucksack mit einem Volumen von \SI{5}{\litre} gegeben
                und die Gegenstände in Tabelle \ref{tab:knapsack} stehen zur
                Verfügung.
                \begin{table}
                    \centering
                    \begin{tabular}{r|rrr}
                        $i$ & $w_i$ & $v_i$ & $\frac{w_i}{v_i}$ \\ \hline
                        1 & 6 \EUR & \SI{1}{\litre} & 6 $\frac{\text{\EUR}}{\si{\litre}}$ \\
                        2 & 10 \EUR & \SI{2}{\litre} & 5 $\frac{\text{\EUR}}{\si{\litre}}$ \\
                        3 & 12 \EUR & \SI{3}{\litre} & 4 $\frac{\text{\EUR}}{\si{\litre}}$
                    \end{tabular}
                    \caption{Verfügbare Gegenstände}
                    \label{tab:knapsack}
                \end{table}
                Nach der Strategie würden die Gegenstände 1 und 2 mit einem Wert
                von zusammen 16 {\EUR} ausgewählt werden.
                Allerdings besitzen die Gegenstände 2 und 3 zusammen einen Wert
                von 22 {\EUR} und passen auch in den Rucksack.
                Die Strategie liefert also nicht immer ein optimales Ergebnis.

        \end{enumerate}

    \item
        \begin{enumerate}
            \item
                Währen der Ausführung von Kruskals Algorithmus werden folgende
                Kanten zum MST hinzugefügt.
                \begin{enumerate}[(i)]
                    \item
                        $\{a,e\}$, $\{d,h\}$ und $\{e,f\}$ in beliebiger
                        Reihenfolge

                    \item
                        entweder $\{b,e\}$ oder $\{b,f\}$

                    \item
                        $\{f,g\}$ und $\{g,h\}$ in beliebiger Reihenfolge

                    \item
                        entweder $\{c,d\}$ oder $\{c,g\}$

                \end{enumerate}

                Dabei können die MSTs in den Abbildungen \ref{fig:mst-1} bis
                \ref{fig:mst-4} entstehen.
                \begin{figure}[h]
                    \centering
                    \begin{tikzpicture}[
                            auto,
                            scale=2.5,
                        ]
                        \tikzstyle{vertex}=[
                            circle,
                            fill,
                        ]
                        \tikzstyle{edge}=[
                        ]
                        \node [vertex, label=above:$a$] (a) at (0,1) {};
                        \node [vertex, label=above:$b$] (b) at (1,1) {};
                        \node [vertex, label=above:$c$] (c) at (2,1) {};
                        \node [vertex, label=above:$d$] (d) at (3,1) {};
                        \node [vertex, label=below:$e$] (e) at (0,0) {};
                        \node [vertex, label=below:$f$] (f) at (1,0) {};
                        \node [vertex, label=below:$g$] (g) at (2,0) {};
                        \node [vertex, label=below:$h$] (h) at (3,0) {};

                        % \draw [edge] (a) to node {$6$} (b);
                        \draw [edge] (a) to node {$1$} (e);
                        % \draw [edge] (b) to node {$5$} (c);
                        \draw [edge] (b) to node {$2$} (e);
                        % \draw [edge] (b) to node {$2$} (f);
                        \draw [edge] (c) to node {$4$} (d);
                        % \draw [edge] (c) to node {$5$} (f);
                        % \draw [edge] (c) to node {$4$} (g);
                        % \draw [edge] (d) to node {$4$} (g);
                        \draw [edge] (d) to node {$1$} (h);
                        \draw [edge] (e) to node {$1$} (f);
                        \draw [edge] (f) to node {$3$} (g);
                        \draw [edge] (g) to node {$3$} (h);
                    \end{tikzpicture}
                    \caption{Ein möglicher MST des angegebenen Graphen}
                    \label{fig:mst-1}
                \end{figure}
                \begin{figure}[h]
                    \centering
                    \begin{tikzpicture}[
                            auto,
                            scale=2.5,
                        ]
                        \tikzstyle{vertex}=[
                            circle,
                            fill,
                        ]
                        \tikzstyle{edge}=[
                        ]
                        \node [vertex, label=above:$a$] (a) at (0,1) {};
                        \node [vertex, label=above:$b$] (b) at (1,1) {};
                        \node [vertex, label=above:$c$] (c) at (2,1) {};
                        \node [vertex, label=above:$d$] (d) at (3,1) {};
                        \node [vertex, label=below:$e$] (e) at (0,0) {};
                        \node [vertex, label=below:$f$] (f) at (1,0) {};
                        \node [vertex, label=below:$g$] (g) at (2,0) {};
                        \node [vertex, label=below:$h$] (h) at (3,0) {};

                        % \draw [edge] (a) to node {$6$} (b);
                        \draw [edge] (a) to node {$1$} (e);
                        % \draw [edge] (b) to node {$5$} (c);
                        \draw [edge] (b) to node {$2$} (e);
                        % \draw [edge] (b) to node {$2$} (f);
                        % \draw [edge] (c) to node {$4$} (d);
                        % \draw [edge] (c) to node {$5$} (f);
                        \draw [edge] (c) to node {$4$} (g);
                        % \draw [edge] (d) to node {$4$} (g);
                        \draw [edge] (d) to node {$1$} (h);
                        \draw [edge] (e) to node {$1$} (f);
                        \draw [edge] (f) to node {$3$} (g);
                        \draw [edge] (g) to node {$3$} (h);
                    \end{tikzpicture}
                    \caption{Ein möglicher MST des angegebenen Graphen}
                    \label{fig:mst-2}
                \end{figure}
                \begin{figure}[h]
                    \centering
                    \begin{tikzpicture}[
                            auto,
                            scale=2.5,
                        ]
                        \tikzstyle{vertex}=[
                            circle,
                            fill,
                        ]
                        \tikzstyle{edge}=[
                        ]
                        \node [vertex, label=above:$a$] (a) at (0,1) {};
                        \node [vertex, label=above:$b$] (b) at (1,1) {};
                        \node [vertex, label=above:$c$] (c) at (2,1) {};
                        \node [vertex, label=above:$d$] (d) at (3,1) {};
                        \node [vertex, label=below:$e$] (e) at (0,0) {};
                        \node [vertex, label=below:$f$] (f) at (1,0) {};
                        \node [vertex, label=below:$g$] (g) at (2,0) {};
                        \node [vertex, label=below:$h$] (h) at (3,0) {};

                        % \draw [edge] (a) to node {$6$} (b);
                        \draw [edge] (a) to node {$1$} (e);
                        % \draw [edge] (b) to node {$5$} (c);
                        % \draw [edge] (b) to node {$2$} (e);
                        \draw [edge] (b) to node {$2$} (f);
                        \draw [edge] (c) to node {$4$} (d);
                        % \draw [edge] (c) to node {$5$} (f);
                        % \draw [edge] (c) to node {$4$} (g);
                        % \draw [edge] (d) to node {$4$} (g);
                        \draw [edge] (d) to node {$1$} (h);
                        \draw [edge] (e) to node {$1$} (f);
                        \draw [edge] (f) to node {$3$} (g);
                        \draw [edge] (g) to node {$3$} (h);
                    \end{tikzpicture}
                    \caption{Ein möglicher MST des angegebenen Graphen}
                    \label{fig:mst-3}
                \end{figure}
                \begin{figure}[h]
                    \centering
                    \begin{tikzpicture}[
                            auto,
                            scale=2.5,
                        ]
                        \tikzstyle{vertex}=[
                            circle,
                            fill,
                        ]
                        \tikzstyle{edge}=[
                        ]
                        \node [vertex, label=above:$a$] (a) at (0,1) {};
                        \node [vertex, label=above:$b$] (b) at (1,1) {};
                        \node [vertex, label=above:$c$] (c) at (2,1) {};
                        \node [vertex, label=above:$d$] (d) at (3,1) {};
                        \node [vertex, label=below:$e$] (e) at (0,0) {};
                        \node [vertex, label=below:$f$] (f) at (1,0) {};
                        \node [vertex, label=below:$g$] (g) at (2,0) {};
                        \node [vertex, label=below:$h$] (h) at (3,0) {};

                        % \draw [edge] (a) to node {$6$} (b);
                        \draw [edge] (a) to node {$1$} (e);
                        % \draw [edge] (b) to node {$5$} (c);
                        % \draw [edge] (b) to node {$2$} (e);
                        \draw [edge] (b) to node {$2$} (f);
                        % \draw [edge] (c) to node {$4$} (d);
                        % \draw [edge] (c) to node {$5$} (f);
                        \draw [edge] (c) to node {$4$} (g);
                        % \draw [edge] (d) to node {$4$} (g);
                        \draw [edge] (d) to node {$1$} (h);
                        \draw [edge] (e) to node {$1$} (f);
                        \draw [edge] (f) to node {$3$} (g);
                        \draw [edge] (g) to node {$3$} (h);
                    \end{tikzpicture}
                    \caption{Ein möglicher MST des angegebenen Graphen}
                    \label{fig:mst-4}
                \end{figure}

            \item
                \begin{enumerate}[(i)]
                    \item Algorithmus von Prim (mit Startknoten $a$) \\
                        Reihenfolge der hinzugefügten Kanten
                        \begin{equation}
                            \{ a, j \},
                            \{ i, j \},
                            \{ h, i \},
                            \{ b, h \},
                            \{ g, h \},
                            \{ c, h \},
                            \{ c, d \},
                            \{ d, e \},
                            \{ e, f \}
                        \end{equation}

                    \item Algorithmus von Kruskal \\
                        Reihenfolge der hinzugefügten Kanten
                        \begin{equation}
                            \{ b, h \},
                            \{ d, e \},
                            \{ g, h \},
                            \{ h, i \},
                            \{ a, j \},
                            \{ c, d \},
                            \{ c, h \},
                            \{ i, j \},
                            \{ e, f \}
                        \end{equation}

                    \item Reverse-Delete-Algorithmus \\
                        Reihenfolge der entfernen Kanten
                        \begin{equation}
                            \{ e, g \},
                            \{ d, g \},
                            \{ d, h \},
                            \{ f, g \},
                            \{ b, i \},
                            \{ a, b \},
                            \{ a, i \},
                            \{ b, c \}
                        \end{equation}

                \end{enumerate}

                Die Ausführung der Algorithmen führt in diesem Fall jeweils zu
                dem MST in Abbildung \ref{fig:mst}.
                \begin{figure}[h]
                    \centering
                    \begin{tikzpicture}[
                            auto,
                            scale=2.5,
                        ]
                        \tikzstyle{vertex}=[
                            circle,
                            fill,
                        ]
                        \tikzstyle{edge}=[
                        ]
                        \node [vertex, label=above:$a$] (a) at (4,1) {};
                        \node [vertex, label=above:$b$] (b) at (3,1) {};
                        \node [vertex, label=above:$c$] (c) at (2,1) {};
                        \node [vertex, label=above:$d$] (d) at (1,1) {};
                        \node [vertex, label=above:$e$] (e) at (0,1) {};
                        \node [vertex, label=below:$j$] (j) at (4,0) {};
                        \node [vertex, label=below:$i$] (i) at (3,0) {};
                        \node [vertex, label=below:$h$] (h) at (2,0) {};
                        \node [vertex, label=below:$g$] (g) at (1,0) {};
                        \node [vertex, label=below:$f$] (f) at (0,0) {};

                        % \draw [edge] (a) to node {$3$} (b);
                        % \draw [edge] (a) to node {$3$} (i);
                        \draw [edge] (a) to node {$2$} (j);
                        % \draw [edge] (b) to node {$3$} (c);
                        \draw [edge] (b) to node {$1$} (h);
                        % \draw [edge] (b) to node {$4$} (i);
                        \draw [edge] (c) to node {$2$} (d);
                        \draw [edge] (c) to node {$2$} (h);
                        \draw [edge] (d) to node {$1$} (e);
                        % \draw [edge] (d) to node {$6$} (g);
                        % \draw [edge] (d) to node {$6$} (h);
                        \draw [edge] (e) to node {$4$} (f);
                        % \draw [edge] (e) to node {$8$} (g);
                        % \draw [edge] (f) to node {$5$} (g);
                        \draw [edge] (g) to node {$1$} (h);
                        \draw [edge] (h) to node {$1$} (i);
                        \draw [edge] (i) to node {$2$} (j);
                    \end{tikzpicture}
                    \caption{MST von $G$}
                    \label{fig:mst}
                \end{figure}

        \end{enumerate}

\end{enumerate}

\end{document}
