\documentclass[a4paper]{scrartcl}

% font/encoding packages
\usepackage[utf8]{inputenc}
\usepackage[T1]{fontenc}
\usepackage{lmodern}
\usepackage[ngerman]{babel}

\usepackage{amsmath, amssymb, amsfonts, amsthm, mathtools}
\setcounter{MaxMatrixCols}{13}
\allowdisplaybreaks
\usepackage[output-decimal-marker={,}]{siunitx}
\usepackage{graphicx}
\usepackage{subcaption}
\usepackage{caption}
\usepackage{venndiagram}
\usepackage{enumerate}
\usepackage{polynom}
\polyset{style=C, div=:,vars=x}
\usepackage[section]{placeins}
\usepackage{tikz}
\usetikzlibrary{tikzmark}
\usetikzlibrary{calc}
\usetikzlibrary{arrows}
\usetikzlibrary{arrows.meta}
\usepackage{pgfplots}
\pgfplotsset{compat=1.5.1}
\usepgfplotslibrary{fillbetween}
\usepackage{gauss}
\usepackage{systeme}
\usepackage{cancel}

\tikzset{
    declare function={Floor(\x)=round(\x-0.5);}
}

% Matrizen
\newdimen\asep 
\asep=0.25\baselineskip 

\providecommand{\vo}{ 
    \text{\raisebox{-\asep}[0pt][0pt]{\rule[0pt]{0.8pt}{3.5\asep}}}} 
\providecommand{\vu}{ 
    \text{\raisebox{-0.5\asep}[0pt][0pt]{\rule[0pt]{0.8pt}{\baselineskip}}}} 
\providecommand{\vm}{ 
    \text{\raisebox{-\asep}[0pt][0pt]{\rule[0pt]{0.8pt}{\baselineskip}}}}

\newtheorem*{behaupt}{Behauptung}

\newcommand{\fallfac}[2]{{#1}^{\underline{#2}}}

\newcommand{\risefac}[2]{{#1}^{\overline{#2}}}

\newcommand{\dif}{\mathop{}\!\mathrm{d}}

\newcommand{\linf}[1]{\lim_{#1 \to \infty}}
\newcommand{\gdw}{\Leftrightarrow}

\renewcommand{\arraystretch}{1.3}

\title{Optimierung}
\subtitle{Blatt 3 Hausaufgaben}
\author{
	Lennart Braun (6523742, Gruppe 6) \\
    Alexander Timmermann (6524072, Gruppe 5)
}
\date{zum 3. November 2014}

\begin{document}
\maketitle

\begin{enumerate}
    \item % 1.
        \begin{enumerate}
            \item
                Starttableau
                \begin{equation}
                    \begin{array}{rcrrr}
                        x_3 & = & \frac{1}{2} & -x_1 & +3x_2 \\
                        x_4 & = & 3 & -x_1 & +x_2 \\
                        x_5 & = & 1 & +2x_1 & -\frac{1}{3}x_2 \\
                        \hline
                        z   & = &   & x_1 & +4x_2 \\
                    \end{array}
                \end{equation}
                Eingangsvariable: $x_2$ \\
                Ausgangsvariable: $x_5$ \\

                1. Iteration
                \begin{equation}
                    \begin{array}{rcrrr}
                        x_2 & = & 3 & +6x_1 & -3x_5 \\
                        x_3 & = & \frac{19}{2} & +17x_1 & -9x_5 \\
                        x_4 & = & 6 & +5x_1 & -3x_5 \\
                        \hline
                        z   & = & 12 & +25x_1 & -12x_5 \\
                    \end{array}
                \end{equation}
                Das LP-Problem ist unbeschränkt, da $x_1$ unbegrenzt erhöht
                werden kann.

                Lösung:
                \begin{equation}
                    \begin{split}
                        x_1 &= t \quad \text{mit } t \in \mathbb{R}, t \geq 0 \\
                        x_2 &= 6t + 3 \\
                        x_3 &= 17t + \frac{19}{2} \\
                        x_4 &= 5t + 6 \\
                        x_5 &= 0 \\
                        z   &= 25t + 12
                    \end{split}
                \end{equation}
                Lösung als Halbgerade $g$ des $\mathbb{R}^2$ in Parameterform:
                \begin{equation}
                    g\colon
                    \begin{pmatrix}
                        x_1 \\ x_2
                    \end{pmatrix}
                    =
                    \begin{pmatrix}
                        1 \\ 6
                    \end{pmatrix}
                    \cdot t +
                    \begin{pmatrix}
                        0 \\ 3
                    \end{pmatrix}
                \end{equation}

            \item
                \begin{center}
                    \begin{tikzpicture}
                        \begin{axis}[
                            axis lines=middle,
                            axis equal,
                            xlabel=$x_1$,
                            ylabel=$x_2$,
                            width=1.0\textwidth,
                            xmin = 0, xmax = 10,
                            ymin = 0, ymax = 10,
                            x axis line style={name path=xaxis}
                        ]
                            \addplot
                            [name path=a, samples=200, domain=0:10]
                            {1/3*x -1/6};
                            \addplot
                            [name path=b, samples=200, domain=0:10]
                            {x-3};
                            \addplot
                            [name path=c, samples=200, domain=0:10]
                            {6*x+3};
                            \addplot
                            [name path=h, samples=200, domain=0:10, color=white]
                            {10};
                            \node at (axis cs:0.7,9) {$g$};
                            \addplot[gray!50] fill between[of=c and xaxis, soft clip={domain=0:0.5+0.01}];
                            \addplot[gray!50] fill between[of=c and a, soft clip={domain=0.5-0.01:7/6+0.01}];
                            \addplot[gray!50] fill between[of=h and a, soft clip={domain=7/6-0.01:4.25+0.01}];
                            \addplot[gray!50] fill between[of=h and b, soft clip={domain=4.25-0.01:10}];
                        \end{axis}
                    \end{tikzpicture}
                \end{center}
                Würde man die graphische Methode anwenden, so könnte man die
                Gerade, welche sich aus der Zielfunktion ergibt, beliebig weit
                verschieben, ohne dass eine optimale Lösung gefunden wird.

            \item
                \begin{equation}
                    \begin{gathered}
                        \begin{split}
                            z &= 50 \\
                            \gdw 25t + 12 &= 50 \\
                            \gdw 25t &= 38 \\
                            \gdw t &= \frac{38}{25} \\
                        \end{split} \\
                        \Rightarrow \left( \frac{38}{25}, \frac{303}{25} \right)
                    \end{gathered}
                \end{equation}

                \begin{equation}
                    \begin{gathered}
                        \begin{split}
                            z &= 200 \\
                            \gdw 25t + 12 &= 200 \\
                            \gdw 25t &= 188 \\
                            \gdw t &= \frac{188}{25} \\
                        \end{split} \\
                        \Rightarrow \left( \frac{188}{25}, \frac{1203}{25} \right)
                    \end{gathered}
                \end{equation}

                \begin{equation}
                    \begin{gathered}
                        \begin{split}
                            z &= 1000 \\
                            \gdw 25t + 12 &= 1000 \\
                            \gdw 25t &= 988 \\
                            \gdw t &= \frac{988}{25} \\
                        \end{split} \\
                        \Rightarrow \left( \frac{988}{25}, \frac{6003}{25} \right)
                    \end{gathered}
                \end{equation}
                

        \end{enumerate}

    \item % 2.
        \begin{enumerate}
            \item
                \begin{equation}
                    \begin{gathered}
                        \text{maximiere }
                        5x_1 +11x_2 \\
                        \text{unter den Nebenbedingungen} \\
                        \begin{array}{rrcr}
                            -x_1 & +x_2  & \leq & 9 \\
                            8x_1 & -2x_2 & \leq & 3 \\
                            -x_1 & -x_2  & \leq & -2
                        \end{array} \\
                        x_1, x_2 \geq 0
                    \end{gathered}
                \end{equation}
                Hilfsproblem
                \begin{equation}
                    \begin{gathered}
                        \text{maximiere }
                        -x_0 \\
                        \text{unter den Nebenbedingungen} \\
                        \begin{array}{rrrcr}
                            -x_1 & +x_2 & -x_0 & \leq 9 \\
                            8x_1 & -2x_2 & -x_0 & \leq 3 \\
                            -x_1 & -x_2 & -x_0 & \leq -2
                        \end{array} \\
                        x_0, x_1, x_2 \geq 0
                    \end{gathered}
                \end{equation}
                
                unzulässiges Tableau des Hilfsproblems
                \begin{equation}
                    \begin{array}{rcrrrr}
                        x_3 & = & 9 & +x_1 & -x_2 & +x_0 \\
                        x_4 & = & 3 & -8x_1 & +2x_2 & +x_0 \\
                        x_5 & = & -2 & +x_1 & +x_2 & +x_0 \\
                        \hline
                        w   & = & & & & -x_0
                    \end{array}
                \end{equation}
                Eingangsvariable: $x_0$ \\
                Ausgangsvariable: $x_5$ \\

                Starttableau des Hilfsproblems
                \begin{equation}
                    \begin{array}{rcrrrr}
                        x_0 & = & 2 & -x_1 & -x_2 & +x_5 \\
                        x_3 & = & 11 &  & -2x_2 & +x_5 \\
                        x_4 & = & 5 & -9x_1 & +x_2 & +x_5 \\
                        \hline
                        w   & = & -2 & +x_1 & +x_2 & -x_5
                    \end{array}
                \end{equation}
                Eingangsvariable: $x_1$ \\
                Ausgangsvariable: $x_4$ \\

                1. Iteration
                \begin{equation}
                    \begin{array}{rcrrrr}
                        x_1 & = & \frac{5}{9} & +\frac{1}{9}x_2 & +\frac{1}{9}x_5 & -\frac{1}{9}x_4 \\
                        x_0 & = & \frac{13}{9} & -\frac{10}{9}x_2 & +\frac{8}{9}x_5 & +\frac{1}{9}x_4 \\
                        x_3 & = & 11 & -2x_2 & +x_5 &  \\
                        \hline
                        w   & = & -\frac{13}{9} & +\frac{10}{9}x_2 & -\frac{8}{9}x_5 & -\frac{1}{9}x_4 
                    \end{array}
                \end{equation}
                Eingangsvariable: $x_2$ \\
                Ausgangsvariable: $x_0$ \\

                2. Iteration
                \begin{equation}
                    \begin{array}{rcrrrr}
                        x_2 & = & \frac{13}{10} & +\frac{4}{5}x_5 & +\frac{1}{10}x_4 & -\frac{9}{10}x_0 \\
                        x_1 & = & \frac{7}{10} & +\frac{11}{5}x_5 & +\frac{1}{10}x_4 & -\frac{1}{10}x_0 \\
                        x_3 & = & \frac{42}{5} & -\frac{3}{5}x_5 & -\frac{1}{5}x_4 & +\frac{9}{5}x_0 \\
                        \hline
                        w   & = &  &  &  & -x_0 
                    \end{array}
                \end{equation}
                Optimale Lösung:
                \begin{equation}
                    \begin{split}
                        x_1 &= \frac{7}{10} \\
                        x_2 &= \frac{13}{10} \\
                        x_3 &= \frac{42}{5} \\
                        x_4 &= 0 \\
                        x_5 &= 0 \\
                        x_0 &= 0 \\
                        w   &= 0
                    \end{split}
                \end{equation}

                Starttableau für die 2. Phase
                \begin{equation}
                    \begin{array}{rcrrrr}
                        x_2 & = & \frac{13}{10} & +\frac{4}{5}x_5 & +\frac{1}{10}x_4 \\
                        x_1 & = & \frac{7}{10} & +\frac{11}{5}x_5 & +\frac{1}{10}x_4 \\
                        x_3 & = & \frac{42}{5} & -\frac{3}{5}x_5 & -\frac{1}{5}x_4 \\
                        \hline
                        z   & = & \frac{89}{5} & +\frac{49}{5}x_5 & +\frac{3}{5}x_4
                    \end{array}
                \end{equation}

            \item

        \end{enumerate}


\end{enumerate}


\end{document}
