\documentclass[a4paper]{scrartcl}

% font/encoding packages
\usepackage[utf8]{inputenc}
\usepackage[T1]{fontenc}
\usepackage{lmodern}
\usepackage[ngerman]{babel}

\usepackage{amsmath, amssymb, amsfonts, amsthm, mathtools}
\setcounter{MaxMatrixCols}{13}
\allowdisplaybreaks
\usepackage[output-decimal-marker={,}]{siunitx}
\usepackage{graphicx}
\usepackage{subcaption}
\usepackage{caption}
\usepackage{venndiagram}
\usepackage{enumerate}
\usepackage{polynom}
\polyset{style=C, div=:,vars=x}
\usepackage[section]{placeins}
\usepackage{tikz}
\usetikzlibrary{arrows}
\usetikzlibrary{arrows.meta}
\usepackage{pgfplots}
\pgfplotsset{compat=1.5.1}
\usepgfplotslibrary{fillbetween}
\usepackage{gauss}
\usepackage{systeme}
\usepackage{cancel}

\tikzset{
    declare function={Floor(\x)=round(\x-0.5);}
}

% Matrizen
\newdimen\asep 
\asep=0.25\baselineskip 

\providecommand{\vo}{ 
    \text{\raisebox{-\asep}[0pt][0pt]{\rule[0pt]{0.8pt}{3.5\asep}}}} 
\providecommand{\vu}{ 
    \text{\raisebox{-0.5\asep}[0pt][0pt]{\rule[0pt]{0.8pt}{\baselineskip}}}} 
\providecommand{\vm}{ 
    \text{\raisebox{-\asep}[0pt][0pt]{\rule[0pt]{0.8pt}{\baselineskip}}}}

\newtheorem*{behaupt}{Behauptung}

\newcommand{\fallfac}[2]{{#1}^{\underline{#2}}}

\newcommand{\risefac}[2]{{#1}^{\overline{#2}}}

\newcommand{\dif}{\mathop{}\!\mathrm{d}}

\newcommand{\linf}[1]{\lim_{#1 \to \infty}}
\newcommand{\gdw}{\Leftrightarrow}

\title{Optimierung}
\subtitle{Blatt 1 Hausaufgaben}
\author{
	Lennart Braun (6523742) \\
}
\date{zum 20. Oktober 2014}

\begin{document}
\maketitle

\begin{enumerate}
    \item % 1.
        \begin{enumerate}
            \item Probleme in Standardform
                \begin{enumerate}[(i)]
                    \item
                        \begin{equation}
                            \begin{gathered}
                                \text{maximiere } -x_1 + x_2 + x_3 - 2x_4 \\
                                \text{unter den Nebenbedingungen} \\
                                \sysdelim..\systeme{%
                                    7x_1 -x_2 +x_3 \leq 2,
                                    -5x_2 +x_3 -x_4 \leq 7,
                                    5x_2 -x_3 +x_4 \leq -7,
                                    -3x_1 +x_2 +2x_3 -x_4 \leq -3
                                } \\
                                x_1, x_2, x_3, x_4 \geq 0
                            \end{gathered}
                        \end{equation}
                        
                    \item
                        \begin{equation}
                            \begin{gathered}
                                \text{maximiere } x_1 - x_2 - x_3' + x_3'' + 2x_4 \\
                                \text{unter den Nebenbedingungen} \\
                                \syssubstitute{a{x_3'}b{x_3''}}
                                \sysdelim..\systeme[x_1x_2abx_4]{%
                                    -7x_1 +x_2 -4a +4b \leq  2,
                                    3x_1 -x_2 -2a +2b +x_4 \leq 3,
                                    x_2 -2x_4 \leq 7,
                                    -x_2 +2x_4 \leq -7
                                } \\
                                x_1, x_2, x_3', x_3'', x_4 \geq 0
                            \end{gathered}
                        \end{equation}
                \end{enumerate}
            \item
                Die Nebenbedingungen lassen sich in Ungleichungen des Formates
                $x_2 \leq ax_1 + b$ bzw. $x_2 \geq ax_1 + b$ überführen, die
                Halbebenen beschreiben.
                \begin{equation}
                    \begin{alignedat}{7}
                        5x_1 -2x_2 &\leq& 10 & \quad \Leftrightarrow & \quad x_2 &\geq& \frac{5}{2}x_1 -5 \\
                        -2x_1 +2x_2 &\leq& 5 & \quad \Leftrightarrow & \quad x_2 &\leq& x_1 +\frac{5}{2} \\
                        x_1 +x_2 &\leq& 7 & \quad \Leftrightarrow & \quad x_2 &\leq& -x_1 +7
                    \end{alignedat}
                \end{equation}
                Die Zielfunktion wird umgeformt und $z$ soweit erhöht, dass
                alle Bedingungen eingehalten werden.
                \begin{equation}
                    5x_1 + 3x_2 = z \Leftrightarrow 3x_2 = -\frac{5}{3}x_1 + \frac{z}{3}
                \end{equation}
                \begin{center}
                    \begin{tikzpicture}
                        \begin{axis}[
                            axis lines=middle,
                            axis equal,
                            xlabel=$x_1$,
                            ylabel=$x_2$,
                            %y label style={at={(0.35,1)}},
                            width=1.0\textwidth,
                            xmin = 0, xmax = 4,
                            ymin = 0, ymax = 8,
                            x axis line style={name path=xaxis}
                        ]
                            \addplot
                            [name path=a, samples=200,]
                            {-5+5/2*x};
                            \addplot
                            [name path=b, samples=200,]
                            {5/2+x};
                            \addplot
                            [name path=c, samples=200,]
                            {7-x};
                            \addplot[gray!50] fill between[of=b and xaxis, soft clip={domain=0:2+0.01}];
                            \addplot[gray!50] fill between[of=a and b, soft clip={domain=2-0.01:9/4+0.01}];
                            \addplot[gray!50] fill between[of=a and c, soft clip={domain=9/4-0.01:24/7}];
                            \addplot
                            [name path=d1, samples=200, style=dashed]
                            {15/3-5/3*x};
                            \addplot
                            [name path=d2, samples=200, style=dashed]
                            {195/7/3-5/3*x};
                            \addplot
                            [->, > = {Latex[scale=1.5]}, name path=d3, samples=200, domain=2.7:4.5]
                            {3/5*x-1};
                            \addplot[mark=*] coordinates {(24/7,25/7)};
                        \end{axis}
                    \end{tikzpicture}
                \end{center}
                Das Maximum ist bei $x_1 = \frac{24}{7}$ und $x_2 = \frac{25}{7}$ erreicht.
        \end{enumerate}

    \item % 2.
        \begin{enumerate}
            \item
                \begin{equation}
                    \begin{gathered}
                        \text{minimiere }
                        67W +120K +100H +90F +97B +124N +98S +62M \\
                        \text{unter den Nebenbedingungen} \\
                        \sysdelim..\systeme[WKHFBNSM]{%
                            8W +25K +30H +22F +3B +8N +6S \geq 75,
                            1W +35K +8H +1F +33N +13S +98M \geq 90,
                            54W +42B +4N +63S \geq 300,
                            S \leq 80
                        } \\
                        W, K, H, F, B, N, S, M \geq 0
                    \end{gathered}
                \end{equation}
                in Standardform
                \begin{equation}
                    \begin{gathered}
                        \text{maximiere }
                        -67W -120K -100H -90F -97B -124N -98S -62M \\
                        \text{unter den Nebenbedingungen} \\
                        \sysdelim..\systeme[WKHFBNSM]{%
                            -8W -25K -30H -22F -3B -8N -6S \leq -75,
                            -1W -35K -8H -1F -33N -13S -98M \leq -90,
                            -54W -42B -4N -63S \leq -300,
                            S \leq 80
                        } \\
                        W, K, H, F, B, N, S, M \geq 0
                    \end{gathered}
                \end{equation}
            \item
                \begin{equation}
                    \begin{gathered}
                        \text{minimiere }
                        21T +16K +371S +346M +884O \\
                        \text{unter den Nebenbedingungen} \\
                        \syssubstitute{.,}
                        \sysdelim..\systeme[TKSMO]{%
                            0.85T +1.62K +12.78S +8.39M \geq 15,
                            0.33T +0.2K +1.58S +1.39M +100O \geq 2,
                            0.33T +0.2K +1.58S +1.39M +100O \leq 6,
                            4.64T +2.37K +74.69S +80.7M \geq 4,
                            9T +8K +7S +508.2M \leq 100,
                            -T +K +S -M -O \leq 0
                        } \\
                        T, K, S, M, O \geq 0
                    \end{gathered}
                \end{equation}
                in Standardform
                \begin{equation}
                    \begin{gathered}
                        \text{maximiere }
                        -21T -16K -371S -346M -884O \\
                        \text{unter den Nebenbedingungen} \\
                        \syssubstitute{.,}
                        \sysdelim..\systeme[TKSMO]{%
                            -0.85T -1.62K -12.78S -8.39M \leq -15,
                            -0.33T -0.2K -1.58S -1.39M -100O \leq -2,
                            0.33T +0.2K +1.58S +1.39M +100O \leq 6,
                            -4.64T -2.37K -74.69S -80.7M \leq -4,
                            9T +8K +7S +508.2M \leq 100,
                            -T +K +S -M -O \leq 0
                        } \\
                        T, K, S, M, O \geq 0
                    \end{gathered}
                \end{equation}
        \end{enumerate}

\end{enumerate}


\end{document}
