\documentclass[a4paper]{scrartcl}

% font/encoding packages
\usepackage[utf8]{inputenc}
\usepackage[T1]{fontenc}
\usepackage{lmodern}
\usepackage[ngerman]{babel}

\usepackage[top=1.5cm,bottom=2cm,left=1.5cm,right=1.5cm]{geometry}

\usepackage{amsmath, amssymb, amsfonts, amsthm, mathtools}
\setcounter{MaxMatrixCols}{13}
\allowdisplaybreaks
\usepackage[output-decimal-marker={,}]{siunitx}
\usepackage{graphicx}
\usepackage{subcaption}
\usepackage{caption}
\usepackage{venndiagram}
\usepackage[shortlabels]{enumitem}
\usepackage{polynom}
\polyset{style=C, div=:,vars=x}
\usepackage[section]{placeins}
\usepackage{tikz}
\usetikzlibrary{tikzmark}
\usetikzlibrary{calc}
\usetikzlibrary{arrows}
\usetikzlibrary{arrows.meta}
\usepackage{pgfplots}
\pgfplotsset{compat=1.5.1}
\usepgfplotslibrary{fillbetween}
\usepackage{gauss}
\usepackage{systeme}
\usepackage{cancel}

\tikzset{
    declare function={Floor(\x)=round(\x-0.5);}
}

% Matrizen
\newdimen\asep 
\asep=0.25\baselineskip 

\providecommand{\vo}{ 
    \text{\raisebox{-\asep}[0pt][0pt]{\rule[0pt]{0.8pt}{3.5\asep}}}} 
\providecommand{\vu}{ 
    \text{\raisebox{-0.5\asep}[0pt][0pt]{\rule[0pt]{0.8pt}{\baselineskip}}}} 
\providecommand{\vm}{ 
    \text{\raisebox{-\asep}[0pt][0pt]{\rule[0pt]{0.8pt}{\baselineskip}}}}

\newtheorem*{behaupt}{Behauptung}

\newcommand{\fallfac}[2]{{#1}^{\underline{#2}}}

\newcommand{\risefac}[2]{{#1}^{\overline{#2}}}

\newcommand{\dif}{\mathop{}\!\mathrm{d}}

\newcommand{\linf}[1]{\lim_{#1 \to \infty}}
\newcommand{\gdw}{\Leftrightarrow}

\renewcommand{\arraystretch}{1.3}

\title{Optimierung}
\subtitle{Blatt 5 Hausaufgaben}
\author{
	Lennart Braun (6523742, Gruppe 6) \\
    Alexander Timmermann (6524072, Gruppe 5)
}
\date{zum 17. November 2014}

\begin{document}
\maketitle

\begin{enumerate}[label=\bfseries\arabic*.]
    \item % 1.
        \begin{enumerate}
            \item
                \begin{equation}
                    \begin{gathered}
                        \text{maximiere } x_{01} +x_{02} +x_{03} +x_{04} \\
                        \text{unter den Nebenbedingungen} \\
                        \begin{array}{rrrrrrrrrrrrrrcr}
                            % x_{01} & x_{02} & x_{03} & x_{04} & x_{16} & x_{21} & x_{25} & x_{26} & x_{34} & x_{35} & x_{47} & x_{57} & x_{65} & x_{67} \\
                            x_{01} &  &  &  & -x_{16} & +x_{21} &  &  &  &  &  &  &  &  & = & 0 \\
                            & x_{02} &  &  &  & -x_{21} & -x_{25} & -x_{26} &  &  &  &  &  &  & = & 0 \\
                            &  & x_{03} &  &  &  &  &  & -x_{34} & -x_{35} &  &  &  &  & = & 0 \\
                            &  &  & x_{04} &  &  &  &  & +x_{34} &  & -x_{47} &  &  &  & = & 0 \\
                            &  &  &  &  &  & x_{25} &  &  & +x_{35} &  & -x_{57} & +x_{65} &  & = & 0 \\
                            &  &  &  & x_{16} &  &  & +x_{26} &  &  &  &  & -x_{65} & -x_{67} & = & 0 \\
                        \end{array} \\
                        \begin{array}{rcrcr}
                            0 & \leq & x_{01} & \leq & 7 \\
                            0 & \leq & x_{02} & \leq & 1 \\
                            0 & \leq & x_{03} & \leq & 3 \\
                            0 & \leq & x_{04} & \leq & 2 \\
                            0 & \leq & x_{16} & \leq & 3 \\
                            0 & \leq & x_{21} & \leq & 4 \\
                            0 & \leq & x_{25} & \leq & 5 \\
                            0 & \leq & x_{26} & \leq & 2 \\
                            0 & \leq & x_{34} & \leq & 5 \\
                            0 & \leq & x_{35} & \leq & 4 \\
                            0 & \leq & x_{47} & \leq & 5 \\
                            0 & \leq & x_{57} & \leq & 2 \\
                            0 & \leq & x_{65} & \leq & 8 \\
                            0 & \leq & x_{67} & \leq & 3
                        \end{array}
                    \end{gathered}
                \end{equation}

            \item

            \item

        \end{enumerate}

    \item % 2.
        \begin{enumerate}
            \item

            \item

        \end{enumerate}

\end{enumerate}


\end{document}
