\documentclass[a4paper]{scrartcl}

% font/encoding packages
\usepackage[utf8]{inputenc}
\usepackage[T1]{fontenc}
\usepackage{lmodern}
\usepackage[ngerman]{babel}

\usepackage[top=2.5cm,bottom=3cm,left=1.5cm,right=1.5cm]{geometry}

\usepackage{amsmath, amssymb, amsfonts, amsthm, mathtools}
\usepackage{marvosym}
\setcounter{MaxMatrixCols}{13}
\allowdisplaybreaks
\usepackage[output-decimal-marker={,}]{siunitx}
\usepackage{graphicx}
\usepackage{subcaption}
\usepackage{caption}
\usepackage{venndiagram}
\usepackage[shortlabels]{enumitem}
\usepackage{polynom}
\polyset{style=C, div=:,vars=x}
\usepackage[section]{placeins}
\usepackage{tikz}
\usetikzlibrary{tikzmark}
\usetikzlibrary{calc}
\usetikzlibrary{arrows}
\usetikzlibrary{arrows.meta}
\usepackage{pgfplots}
\pgfplotsset{compat=1.5.1}
\usepgfplotslibrary{fillbetween}
\usepackage{gauss}
\usepackage{systeme}
\usepackage{cancel}

\tikzset{
    declare function={Floor(\x)=round(\x-0.5);}
}

% Matrizen
\newdimen\asep 
\asep=0.25\baselineskip 

\providecommand{\vo}{ 
    \text{\raisebox{-\asep}[0pt][0pt]{\rule[0pt]{0.8pt}{3.5\asep}}}} 
\providecommand{\vu}{ 
    \text{\raisebox{-0.5\asep}[0pt][0pt]{\rule[0pt]{0.8pt}{\baselineskip}}}} 
\providecommand{\vm}{ 
    \text{\raisebox{-\asep}[0pt][0pt]{\rule[0pt]{0.8pt}{\baselineskip}}}}

\newtheorem*{behaupt}{Behauptung}

\newcommand{\fallfac}[2]{{#1}^{\underline{#2}}}

\newcommand{\risefac}[2]{{#1}^{\overline{#2}}}

\newcommand{\dif}{\mathop{}\!\mathrm{d}}

\newcommand{\linf}[1]{\lim_{#1 \to \infty}}
\newcommand{\gdw}{\Leftrightarrow}

\renewcommand{\arraystretch}{1.3}

\title{Optimierung}
\subtitle{Blatt 5 Hausaufgaben}
\author{
	Lennart Braun (6523742, Gruppe 6) \\
    Alexander Timmermann (6524072, Gruppe 5)
}
\date{zum 17. November 2014}

\begin{document}
\maketitle

\begin{enumerate}[label=\bfseries\arabic*.]
    \item % 1.
        \begin{enumerate}
            \item Die Aufgabe, einen Flusses maximaler Stärke zu finden, als
                LP-Problem.
                \begin{equation}
                    \begin{gathered}
                        \text{maximiere } x_{01} +x_{02} +x_{03} +x_{04} \\
                        \text{unter den Nebenbedingungen} \\
                        \begin{array}{rrrrrrrrrrrrrrcr}
                            % x_{01} & x_{02} & x_{03} & x_{04} & x_{16} & x_{21} & x_{25} & x_{26} & x_{34} & x_{35} & x_{47} & x_{57} & x_{65} & x_{67} \\
                            x_{01} &  &  &  & -x_{16} & +x_{21} &  &  &  &  &  &  &  &  & = & 0 \\
                            & x_{02} &  &  &  & -x_{21} & -x_{25} & -x_{26} &  &  &  &  &  &  & = & 0 \\
                            &  & x_{03} &  &  &  &  &  & -x_{34} & -x_{35} &  &  &  &  & = & 0 \\
                            &  &  & x_{04} &  &  &  &  & +x_{34} &  & -x_{47} &  &  &  & = & 0 \\
                            &  &  &  &  &  & x_{25} &  &  & +x_{35} &  & -x_{57} & +x_{65} &  & = & 0 \\
                            &  &  &  & x_{16} &  &  & +x_{26} &  &  &  &  & -x_{65} & -x_{67} & = & 0 \\
                        \end{array} \\
                        \begin{array}{rcrcr}
                            0 & \leq & x_{01} & \leq & 7 \\
                            0 & \leq & x_{02} & \leq & 1 \\
                            0 & \leq & x_{03} & \leq & 3 \\
                            0 & \leq & x_{04} & \leq & 2 \\
                            0 & \leq & x_{16} & \leq & 3 \\
                            0 & \leq & x_{21} & \leq & 4 \\
                            0 & \leq & x_{25} & \leq & 5 \\
                            0 & \leq & x_{26} & \leq & 2 \\
                            0 & \leq & x_{34} & \leq & 5 \\
                            0 & \leq & x_{35} & \leq & 4 \\
                            0 & \leq & x_{47} & \leq & 5 \\
                            0 & \leq & x_{57} & \leq & 2 \\
                            0 & \leq & x_{65} & \leq & 8 \\
                            0 & \leq & x_{67} & \leq & 3
                        \end{array}
                    \end{gathered}
                \end{equation}

            \item Die Aufgabe, einen kostenminimalen Fluss der Stärke 6 zu
                finden, als LP-Problem.
                \begin{equation}
                    \begin{gathered}
                        \text{minimiere} \\
                        5x_{01} +4x_{02} +3x_{03} +3x_{05} +x_{06} +6x_{12} +3x_{16} +7x_{23} +5x_{24} +5x_{35} +3x_{45} +4x_{46} +2x_{56} \\
                        \text{unter den Nebenbedingungen} \\
                        \begin{array}{rrrrrrrrrrrrrcr}
                            % x_{01} & x_{02} & x_{03} & x_{05} & x_{06} & x_{12} & x_{16} & x_{23} & x_{24} & x_{35} & x_{45} & x_{46} & x_{56} \\
                            x_{01} & +x_{02} & +x_{03} & +x_{05} & +x_{06} & & & & & & & & & = & 6 \\
                            x_{01} & & & & & -x_{12} & -x_{16} & & & & & & & = & 0 \\
                             & x_{02} & & & & +x_{12} & & -x_{23} & -x_{24} & & & & & = & 0 \\
                             & & x_{03} & & & & & +x_{23} & & -x_{35} & & & & = & 0 \\
                             & & & & & & & & x_{24} & & -x_{45} & -x_{46} & & = & 0 \\
                             & & & x_{05} & & & & & & +x_{35} & +x_{45} & & -x_{56} & = & 0
                        \end{array} \\
                        \begin{array}{rcrcr}
                            0 & \leq & x_{01} & \leq & 3 \\
                            0 & \leq & x_{02} & \leq & 4 \\
                            0 & \leq & x_{03} & \leq & 5 \\
                            0 & \leq & x_{05} & \leq & 4 \\
                            0 & \leq & x_{06} & \leq & 3 \\
                            0 & \leq & x_{12} & \leq & 6 \\
                            0 & \leq & x_{16} & \leq & 1 \\
                            0 & \leq & x_{23} & \leq & 1 \\
                            0 & \leq & x_{24} & \leq & 4 \\
                            0 & \leq & x_{35} & \leq & 2 \\
                            0 & \leq & x_{45} & \leq & 1 \\
                            0 & \leq & x_{46} & \leq & 7 \\
                            0 & \leq & x_{56} & \leq & 4
                        \end{array}
                    \end{gathered}
                \end{equation}

            \item
                \begin{equation}
                    \begin{gathered}
                        \text{minimiere } \sum_{i=1}^5 \sum_{j=1}^3 c_{ij}x_{ij} \\
                        \text{unter den Nebenbedingungen} \\
                        \sum_{i=1}^5 \sum_{j=1}^3 x_{ij} = 3 \\
                        x_{i1} +x_{i2} +x_{i3} \leq 1 \text{ für alle } i = 1, \dots, 5 \\
                        x_{ij} \in \{ 0, 1 \} \text{ für alle } i = 1, \dots, 5 \text{ und } j = 1, \dots, 3
                    \end{gathered}
                \end{equation}
                

        \end{enumerate}

    \item % 2.
        \begin{enumerate}
            \item Sei $g$ der Gewinn in Euro.
                \begin{equation}
                    g = 6A_1 +14A_2 +9A_3 +6B_1 +15B_2 +11B_3 +10C_1 +16C_2 +13C_3
                \end{equation}
                Für den angegebenen Produktionsplan beträgt der Gewinn
                \begin{equation}
                    \begin{split}
                        g &= 6 \cdot 145 + 14 \cdot 35 + 9 \cdot 20 + 6 \cdot 25
                            + 15 \cdot 35 + 11 \cdot 55 + 10 \cdot 80 + 16 \cdot 140 + 13 \cdot 20 \\
                        &= 6120 \text{\,\EUR.}
                    \end{split}
                \end{equation}

            \item
                $X_1$: in regulärer Zeit veredelt. \\
                $X_2$: in Überstunden veredelt.

                \begin{equation}
                    \begin{gathered}
                        \text{maximiere} \\
                        \begin{gathered}
                            6(200 -A_1 -A_2) + 14A_1 +9A_2 \\
                            + 6(115 -B_1 -B_2) + 15B_1 +11B_2 \\
                            + 10(240 -C_1 -C_2) + 16C_1 +13C_2 \\
                            = 8A_1 +3A_2 +9B_1 +5B_2 +6C_1 +3C_2 +4290
                        \end{gathered} \\
                        \text{unter den Nebenbedingungen} \\
                        \begin{array}{rrrrrrcr}
                            A_1 & +A_2 & & & & &  \leq & 200 \\
                            & & B_1 & +B_2 & & &  \leq & 115 \\
                            & & & & C_1 & +C_2 &  \leq & 240 \\
                            A_1 & & +B_1 & & +C_1 & &  \leq & 210 \\
                            & A_2 & & +B_2 & & +C_2 &  \leq & 125 \\
                        \end{array} \\
                        A_1, A_2, B_1, B_2, C_1, C_2 \geq 0
                    \end{gathered}
                \end{equation}

        \end{enumerate}

\end{enumerate}


\end{document}
